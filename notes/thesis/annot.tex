\documentclass{article}
\usepackage[czech,english]{babel} % For Czech/English text
\usepackage[utf8]{inputenc}
\usepackage{ifthen}

% Define a boolean variable for language choice
% Set to \newboolean{useenglish} for English
% or \newboolean{useczech} for Czech

\newboolean{useenglish}
\setboolean{useenglish}{false} % <-- change to false for Czech version

\begin{document}

\ifthenelse{\boolean{useenglish}}{
    \selectlanguage{english}
    \section*{Annotation}
    This paper explores the use of conversational AI for intentionally disrupting coreference relations in dialogue.
    The goal is to assess whether such an approach can produce useful data for linguistic research.
    The chatbot is designed to generate ambiguous anaphora that cannot be easily linked to prior entities.
    The resulting data is verified through annotation, with sufficient inter-annotator agreement achieved.
    Findings indicate that success varies with the complexity of the stimuli, suggesting improvements for conversation design.
}{
    \selectlanguage{czech}
    \section*{Anotace}
    Tato práce se zabývá možností využití konverzační umělé inteligence pro cílené narušování koreferenčních vztahů v textu.
    Cílem je ověřit, zda lze tímto způsobem generovat vhodná data pro lingvistický výzkum.
    Chatbot je navržen tak, aby používal nejednoznačné anafory, které nelze snadno přiřadit předchozím entitám.
    Data jsou následně ověřena anotací, přičemž shoda mezi anotátory byla dostatečná.
    Výsledky ukazují, že úspěšnost závisí na složitosti podnětů a naznačují možnosti zlepšení návrhu konverzace.
}

\end{document}
