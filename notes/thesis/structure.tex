\documentclass[12pt]{report}
\title{Coherence disruptions in human–chatbot interaction: towards quantitative approach to conversation}
\date{}
\author{Albert Maršík}

\bibliographystyle{alpha}
\usepackage[style=verbose]{biblatex}
\usepackage{dirtytalk}
\usepackage[utf8]{inputenc}
\usepackage[T1]{fontenc}
\usepackage{enumitem}
\usepackage{geometry}
\usepackage{setspace}
\usepackage{array}
\addbibresource{references.bib}

\geometry{a4paper, margin=1in}

\newcommand{\utterance}[3]{
    \textbf{#1} #2 \\
    \textit{#3}
}


\begin{document}

   % TODO
   % write experiment chapter
   % write conclusion and further steps
   % add annotations to full data
   %
   %
   % touch up methodology
   % split data chapter
   % touch up theory
   % touch up theory with bublitz


\maketitle
\clearpage
\tableofcontents
\clearpage



\chapter*{Introduction}
\addcontentsline{toc}{chapter}{Introduction}

\par
Recently, there has been a breakthrough in the way we interact with machines \footcite{sharma2024exploring}.
We can now instruct a computer using natural language \footcite{hendrix1982natural}.
Besides making existing technology an extra step accessible,
new ways to use technology appear.
Being able to simply talk to a machine and have it respond
can help overcome longstanding challenges
such as notably accessing a knowledge base via semantic search \footcite{makela2005survey}.
Until recently a knowledge base would usually be accessed via fulltext,
meaning we would only be able to find
information of which we knew a part of the formal encoding.
Today, we can search for information simply by asking questions,
including follow up ones
all thanks to natural language computer interface.
\par
The promise of much practical usage of the current wave of generative AI is ambitious
and only brings its fruit slowly, perhaps slower, than was expected
\footcite{bloomberg2024openai1}\footcite{reuters2024openai}.
There is talk of a "plateau" in development of the technology powering
the current cutting edge inventions \footcite{ritter2024ai}.
That besides the current day issues of cutting edge AI like
high electricity consumption \footcite{ritchie2024ai} and
unpredictable and broad societal impact \footcite{hagerty2019global}\footcite{baldassarre2023social}

That being said, in context of conversation research,
this development in technology promises to make things possible that previously were not.
With a partial control of what happens in the conversation and a decent certainty, that
our system will simulate human–human conversation to the user,
new kind of conversational data is in reach –
logs of the human–bot interaction, that could be categorized based on

\begin{itemize}
    \item
    which researcher controlled stimulus and
    \item
    which participant reaction to given stimulus
\end{itemize}

they contain.

\par
In the 1960s the relatively recent emergence and adoption of telephone technology
allowed for recording and transcribing authentic conversational data.
This advancement took place thanks to developement in technology
which is reminiscent of the current day situation.
While human–bot conversational data is arguably less authentic than telephone conversation transcripts,
experimental approach can be taken while the human element is present.

\par
This papers intention is to provide a debate on a metaresearch question –
is using generative AI a viable methodology for conversation research?
This is done by attempting to develop that very methodology.
Proceeding we operate in a frontier,
our first steps should be establishing data backed baseline knowledge
and assessing possible lines of research.

\par
Understanding what makes
the unraveling text of a conversation a coherent one
can be approached via obtaining
conversational data containing coherence disruptions.
This used to be difficult in the past () but
can be done using the discussed technology –
it has the capability of conversing in a way
that is found generally acceptable by humans
and can drift away from the coherent interaction
if appropriately instruced to do so.

\par
The data this paper seeks to elicitate and analyse are
actual human–chatbot exchanges containing moments which
have the potential to be problematic for
the human participant to process and follow up on.
The boundary between what a coherent and an incoherent conversation is blurred.
It is in no way a binary property of the text of the conversation ().
The goal is therefore to touch on the gradual divide between them.

.. () Bublitz \footcite{bublitz1999disturbed}

\par
While chatbots are evaluated for how natural and error free their way of conversing is,
human–human conversation is rarely flawless as
errors happen and
conversational coherence gets temporarily disrupted
In case of human–human communication, disruptions can however be cured easily ().

\par
In human–bot communication, disruptions could derail a conversation completely,
leaving the bot, who would only rely on surface level textual clues, in the dark. \footcite{mctear2020conversational}
This has become rare with generative AI.
Even though it brings a set of its own problems like
frequently lacking factuality or
the difficulty to handle data responsibly,
the cutting edge technology powered conversation systems are

\begin{itemize}
\item
   better capable of understanding and producing relevant answers

\item
   able to return to their conversational point of departure
\end{itemize}

\par
Human–bot communication is often single–purpose.
Companies and institutions deploy voice applications to interact with customers and clients,
so there is usually a goal to be achieved.
The coherence in each moment of such conversation can then be described based on whether
the goal is currently coming closer to being achieved with ease.
Another common frame for a human–bot interaction is an open–domain conversation,
also known as chit–chat or smalltalk ().
Since there is no global goal to achieve,
the coherence of such interaction is dictated by
a different set of factors ()[https://aclanthology.org/2021.acl-long.136/]

\par
Some factors that influence coherence in conversational texts,
whether in human–human or human–bot exchanges,
have been extensively studied.
\par
Namely:

%––
\par{\textbf{Politeness}
Brown and Levinson’s work on politeness strategies describes social alignment in smooth interactions \footcite{brown1987politeness}. Politeness strategies, such as using polite language, offering options, or softening potentially face–threatening comments, help to create a comfortable communicative environment. These strategies align with social norms, which people interpret as markers of respect, consideration, or even trust. A failure to employ these politeness strategies, or using them inconsistently, can disrupt conversational coherence. For example, blunt or overly direct responses may be perceived as abrupt or rude, diverting the conversation's flow or causing discomfort. In such cases, the breakdown of polite norms can lead participants to question intent, hindering effective and smooth communication.


\par{\textbf{Speech acts}}
Following Austin and Searle's speech act theories, communication rely on expressing clear intentions and meanings that help build mutual understanding \footcite{austin1962how} \footcite{searle1969speech}. When speakers convey intentions explicitly through statements, questions, requests, or assertions, it signals to listeners the purpose and direction of the conversation. Effective communication strategies help maintain coherence by ensuring each contribution builds logically on the last. On the other hand, unclear intentions or ambiguous phrasing can create misunderstandings, disrupting the conversation's flow. Misalignment or mixed signals – such as using sarcasm without cues or making indirect requests without context— can leave listeners uncertain about how to respond, leading to off–track or irrelevant contributions and possibly creating need to address the communication to regain understanding.

\par{\textbf{Conversational Maxims}}
Grice’s conversational maxims are fundamental to coherent dialogue \footcite{grice1975logic}.
They suggest that participants should:

\begin{itemize}
\item
provide truthful information (Quality)
\item
neither too much nor too little (Quantity)
\item
remain on–topic (Relevance)
\item
communicate in an orderly, clear manner (Manner)
\end{itemize}

These maxims encourage effective exchange by setting a standard for contributions that are informative, truthful, relevant, and unambiguous. When violated, such as by giving excessive detail, omitting important context, or straying from the topic, coherence suffers. For instance, irrelevant tangents or over–detailed explanations may confuse the listener as to what is the main focal point of conversation in that moment. This misalignment can leave participants uncertain about the conversation’s direction, ultimately diminishing coherence and the effectiveness of communication.

\par{\textbf{Sequence Structure}}
The work of Schegloff and Sacks on sequence structure and turn–taking emphasizes that ordered interactions support predictability and continuity in dialogue \footcite{Schegloff_2007}. Turn–taking conventions — where participants follow an implicit sequence of speaking and responding — help maintain the flow by structuring the conversation in a logical order. This sequence structure allows both parties to anticipate when to listen and when to speak, contributing to a well–paced, cohesive exchange. However, interruptions, abrupt changes in topic, or skipping expected responses can disrupt this sequence, introducing unpredictability that can confuse participants. These interruptions fragment coherence by shifting the conversation away from expected responses or structured flow, often leaving gaps in understanding or causing conversational breakdowns.

Disturbed coherence fill me in .. Bublitz Lenk ()
hearer knows best
disturbed coherence – partial coherence
topic

\par{\textbf{Message and Topic}}
Interactional linguistics underscores that consistency in message and topic preserves continuity in conversation
\footcite{CouperKuhlenSelting2017}.
When speakers stick to a shared topic or make gradual, clear shifts, coherence is
maintained because participants know what to expect.
Frequent or abrupt topic shifts, however, or sending unclear or conflicting messages, can create disjointed exchanges.
For instance, introducing a new topic without closure on the previous one can confuse listeners,
leading to a scattered or fragmented interaction.

\par
While all of the mentioned areas unveil much about the way conversation works,
rarely do they concern themselves with the textual dimension of conversation.
Most of the mentioned authors (with the notable exception of those operating within the interactional linguistics framework) could hardly be described as linguists,
though their works significantly inform linguistics.

\par
The lack of a true interpersonal dimension in human–chatbot communication
allows to focus solely on the elements in conversational text,
that make it cohesive and coherent or rather
those that have the potential to prevent it from being that.
The key concepts discussed in this paper are
two closely related topics:

Coreference realized by anaphore and topic – what the text is about.


% _____________________

\chapter{Theoretical foundations}

% Hrbáček, Halliday, Roberts, Givón, Nedoluzhko, Izotopie (Daneš a další)

\section{Textual dimension of conversation}
\par
    The following concepts will be explored individually, in relation to one another and in relation to conversation:
    text, coherence, cohesion, coreference, anaphora, cataphora, endophora, exophora, topic, entity, and association.
    While the presented exploration draws on existing literature,
    it seeks to establish an independent and sustainable framework,
    rather than strictly adhering to established interpretations.

\subsubsection{Text}
\par
    Text, in its broadest sense, refers to any form of communication that conveys meaning through a combination of signs, symbols, or language \footcite[p.~7]{hrbacek1994}\footcite{hjelmslev2016}.
    These semiotic structures can take various forms, including written, spoken, visual, or even non–verbal modes of expression \footcite[p.~13]{barthes1977image}.
    A text can be as simple as a single sentence or as complex as a novel, and it can exist across different mediums, from books and articles to advertisements and digital content.
    What defines a text is its ability to convey a coherent message or idea, often intended for interpretation by an audience or an adressee.
    Texts can serve a wide range of purposes, including storytelling, instruction, persuasion, or simply recording information.
    Typically text is a structure that is
    linguistic, produced and percieved as intentional and coherent.

\par
    The text of a conversation is specific because it is multiproducer.
    Another example of a multiproducer text
    would be a sequence of commercial signs on a busy street.
    It is the spatial juxtaposition of the signs and temporal juxtaposition of utterances,
    that make them a text.

    \par
    Another property of a conversation text is it is negotiated ().
    This is given by its multiproducer and temporal nature.
    Other types of text which are also negotiated are relatively rare.
    There are occurences of debates which take place in written text,
    whether they are press columns or academic articles, which
    interact explicitly with each other, making them a negotiation.
    Such press discourse could however be considered
    a sequence of text units rather than a single temporarily juxtaposed text.
    This perspective could hardly be defended in regards to conversation, because
    its tight temporal coupling and cohesion,
    making conversation a unique phenomena.

\subsubsection{Coherence}
\par
    Coherence refers to the logical connections and consistent relationships that
    make a text easy to follow and possible to understand \footcite[p.~83]{givón2020coherence}\footcite[p.~9]{hrbacek1994}.
    It is achieved when the ideas, sentences, and paragraphs within a text are linked together in a meaningful way,
    allowing the reader to grasp the author's message without confusion.
    Coherence often depends on the use of transitions, the logical flow of arguments, and the proper sequencing of information.
    It ensures that each part of the text contributes to the overall meaning, creating a unified whole \footcite[p.~28]{hrbacek1994}.
    Incoherent text can be difficult or impossible to understand, even if the individual sentences are grammatically correct \footcite[p.~30]{hrbacek1994}.
    It is a property of the whole text, but
    textual elements can be pointed out that contribute to or diminish the given texts coherence.
    Those elements are however not referred to as 'coherence elements'.

\par
    Coherence is a cognitive phenomenon \footcite{Roberts01101993} because
    it involves the mental processes of interpreting, organizing, and understanding information.
    When reading a text,
    coherence arises not only from the structure and linguistic cues provided by the author but
    also from the reader's ability to
    make connections between ideas based on prior knowledge, expectations, and context.
    This cognitive interaction between the text and the reader’s mind is what makes the content understandable.

\par
    In conversation, coherence becomes even more complex,
    as multiple participants are simultaneously contributing to and interpreting the flow of information.
    Each individual brings their own perspective and understanding to the interaction,
    which requires constant negotiation to maintain coherence.
    Misunderstandings, different backgrounds, and interruptions
    can disrupt the coherence of a conversation,
    making it a more dynamic and fragile process compared to written text.

\begin{itemize}
\item
whether a written text is coherent depends mostly on the reader () [bublitz]
\item
whether a conversation text is coherent depends on\\ an ongoing negotiation
\end{itemize}

\par
    Coherence is a scalar property rather than a binary one ().
    It is also tricky to measure.
    This paper seeks to explore one possible approach of
    declaring different levels of coherence disruptions
    and observing the acceptance rates in participants
    and corelation between them.

%–––
\subsubsection{Cohesion}
\par
    While coherence refers to the interpretative quality of a text,
    wherein the ideas form a logical and meaningful whole
    cohesion,focuses on the structural relations
    within a text, achieved through grammatical and lexical links.
    It should be seen as an umbrella term
    covering specific relations within
    the structure of the text,
    where cohesive elements can be directly pointed out.
    While coherent text does not necesarily need to be cohesive,
    cohesive elements often support it.
    A coherent text tends to be at least somewhat cohesive.

\par
    Halliday and Hasan \footcite{Halliday76cohesion} developed
    a detailed framework of cohesion, which includes
    endophoric references,
    relating parts of the text to each other, and
    exophoric references, which point outside the text \footcite[p.~31]{Halliday76cohesion}.
    Endophoric cohesion covers aspects
    like anaphoric references and cataphoric references \footcite{hajicovasgall2003} \footcite{loaiciga-etal-2022-anaphoric}.
    Exophoric references, however, rely on shared context beyond the text itself,
    requiring readers to use prior knowledge.
    Following concepts can be considered cohesive elements.

\subsubsection{Cataphore and Exophore}
    %–––
\par
    In Halliday and Hasan's framework,
    cohesion in language is achieved through various devices that connect
    different parts of a text, forming a unified whole.
    They classify cohesive ties as
    references, substitutive forms, ellipsis, and connectors, with
    anaphoric references being one of the primary ways texts achieve cohesion \footcite[p.~68]{Halliday76cohesion}.
    When a text element cannot be mapped to a preceding referent,
    Halliday and Hasan suggest that cohesion is maintained through
    shared situational understanding, making the reference exophoric.
    Cataphoric references, though less common, involve
    elements that look forward in the text,
    showing intentionality by the author but contributing to
    cohesion primarily through the eventual resolution of the forward–pointing referent.

\par
    In conversation if a seemingly anaphoric text element is not succesfully mapped
    to a preceding textual coreferent
    the reference can still be understood, because shared context.
    Such element reaches out of the text with its reference, making it an exophoric one
    Cataphore is a related phenomena –
    a reference which points forward in the text.
    Such occurence is relatively rare in written text and even more so in conversation.
    In fact it is somehow pointless to account for cataphore in a multi–producer text.
    A cataphore denotes an authors intention to reveal
    the nature of a referent explicitly after first mentioning them.
    In conversation, where multiple contributors cocreate given text,
    and mutual understanding and agreement is the measure of
    how coherent the produced text is, later realisation of a vague reference
    does not contribute to how coherent it is.
    Regardless, in case of a cataphore,
    only the referent is a cohesive element,
    not the cataphore,
    as it ties back to the previous text, creating bonds across large textual units.

\subsubsection{Anaphore, Endophore and Coreference}
\par
    A common cohesive text element is an anaphore \footcite{Nedoluzhko2010}.
    It is a reference inside the text pointing back to a previously mentioned entity.
    Often it is realised via personal pronouns.
    Though there are other ways for anaphore to realise.
    %–––
    In Czech, anaphoric references often rely on grammatical gender and number,
    making participial endings essential for identifying the referent.
    For instance, when a gramatically masculine entity is mentioned,
    later references might use a participle in the masculine form, such as šel ("he went"),
    connecting back to it without repeating the noun or using a demonstrative.
    Demonstratives, such as ten ("that") or tento ("this one"),
    also frequently serve anaphoric functions, guiding the reader to a previously mentioned subject.
    Temporal and locative adverbs, such as tam ("there") and tehdy ("then"),
    also contribute cohesion by indirectly referencing time and place details introduced earlier in the text.
    These anaphoric elements strengthen textual coherence by reducing redundancy and maintaining flow.
    The reader identifies coreferential links through these markers,
    following the cohesive threads without needing explicit repetitions.

\par
    An anaphoric element is by definition also endophoric.
    It points inside the text it appears in.
    By definition an anaphoric element has a referent, which occures earlier in the text.
    These two elements are then coreferent.
    As such they also share an identical exophoric reference – they point outside of the text.

\par
    In conversation, many aspects of which are subject to negotiation,
    also specific coreference relations can be questioned \footcite{loaiciga2021reference}.
    The reference realised by one communication participant may be unclear to the other
    resulting in a repair request coming from another particiapant.
    In conversation analysis,
    %–––
    Sacks’s concept of repair traditionally addresses
    misunderstandings related to intentions and actions,
    loosely drawing on frameworks like Austin’s and Searle’s speech act theories.
    From this perspective, repairs often target interpretative gaps about
    what a speaker intends to do with their utterance.
    However, viewed from a broader, more abstract level,
    what is called repair triggers can extend beyond intentions alone,
    encompassing issues on the textual level as well.
    For instance, an nonassignable anaphora —
    a reference that lacks a clear antecedent —
    may lead to a repair request,
    thereby showing how textual ambiguities prompt interactional responses.
    This approach expands the causes of repair in conversation,
    integrating elements of reference and interactional misalignment,
    where a structural aspect of the language itself can
    become a repairable issue in the communicative exchange.

\subsubsection{Topic}
\par
    Topic is what a text is about.
    That makes topic very complicated to define.
    Among others, some issues with topic and annotating it in text are:

\begin{itemize}
\item
A text can and typically does cover multiple topics
\item
Different framing will produce different topic annotations of text
\item
The span of a topic section can be impossible to delimit within text.
\item
Topic annotation is by its nature always more text,
            so even it can be annotated for topic.
            making topic annotations recursive.
            One cannot therefore achieve a definite topic description of a text.
\end{itemize}

\par
    Despite all these complications,
    topic cannot be skipped in conversation research
    as it is deeply intertwined with the concepts mentioned above.
    Topic progressions across text are realised via anaphore and association
    and tightly interact with coherence.
    An sufficient amount of time has to be spent on a given topic unit,
    enough information has to be said about a given topic
    in order for it to be possible to move on or add another one in the conversation.
    Closure has to be provided in order for a topic to be done.
    Transitioning from one topic to another has a potential to disrupt coherence,
    if the association between the topics is too distant.
    A divergence in topic has to be justified.

\subsubsection{Association}
\par
    Association is a textual realization of an isotopic relation \footcite{koblizek2015}.
    By their exophoric properties, referents exist in a semantic web of relationships.
    Similarly to coherence, associative relationships are a cognitive phenomena.
    They come to exist when they are percieved.
    While association is a cohesive element it is difficult to formalize
    the way it can and has been done with anaphoric text relations.
    It is however a major factor in a coherence of text as
    in some cases a text can only rely on association in its coherence.

\subsubsection{Entity}
\par
    An entity is an exophoric referent, descriptions of objects, people, events etc \footcite{entities}.
    Words or text elements which can be reffered to by an anaphore will be called entities.
    Since a phrase containing an anaphore typically adds more information about the referent
    the new information must be semantically compatible,
    in other words association has to be possible between the referent and the added information.
    Entity also has to do with topic.
    In text topic can be represented by a single or multiple entities.
    Coreferent words will be regarded as a single entity.
    It can serve to partially map a topic distance in the texts chronology.

\section{Interactional dimension of conversation}
\par
A conversation text is produced by multiple producers.
This complicates things:

\begin{itemize}

    \item
    Conversation is an interactive process, distinct from static text, which is created collaboratively.

    \item
    Conversational content is continuously negotiated by participants, who continuously adapt one another.

\end{itemize}

Due to its temporal and cooperative nature, conversation allows for:
\begin{itemize}

    \item
    Overlaps in speech,

    \item
    Swift corrections of minor errors,

    \item
    Multiple layers of perspective, including:
        \begin{itemize}

            \item
            Each participant’s personal viewpoint,

            \item
            Each participant’s perception of others’ viewpoints,

            \item
            Each participant’s understanding of the shared conversation as it’s being co–created.
        \end{itemize}
\end{itemize}

        each of these perspectives can desynchronize resulting in misunderstandings.
        Humans however are excellent at correcting misunderstandings
        this is because under regular circumstances, people cooperate.

\par
The Cooperation Principle, introduced by philosopher H.P. Grice,
suggests that participants in a conversation typically work together to achieve effective communication.
Grice proposed that, to ensure this cooperation, speakers follow four conversational maxims.
In practice, people may not always follow these maxims
but they do so in ways that still rely on shared expectations of cooperation.
Even when misunderstandings arise,
humans naturally engage in conversational repair,
using their social intuition and mutual cooperativity to clarify intetion and realign perspectives.

\par
Contemporary conversation research can be understood to draw from conversation analysis.
%–––
Modern conversation research traces its roots to conversation analysis,
a field pioneered by sociologists Harvey Sacks and Emanuel Schegloff in the 1960s.
They sought to understand the structure and
social rules of everyday interactions,
focusing on the patterns and norms that govern turn–taking and response.
Thanks to recordings of phonecalls,
transcripts could be qualitatively analyzed.
This research has lead to coining new terminology.

    \subsubsection{Adjacency pair}
    \par
    Adjacency pairs describe sequences of two related utterances by different speakers \footcite[p.~188]{Sacks1992}. These pairs are characterized by their predictable and reciprocal nature, where the first part sets up the expectation for a specific type of response. Common examples include greetings('Hi' → 'Hello'), questions and answers ('What time is it?' → '3 PM'), or offers and acceptances/declines ('Would you like some coffee?' → 'Yes, please' or 'No, thank you').

    \subsubsection{Sequence structure}
    \par
    Sequence structure refers to the organization of conversational turns into coherent patterns or sequences. It describes how interactions are shaped by predictable structures, such as adjacency pairs. These sequences provide order and meaning to conversations, guiding participants in understanding when and how to respond. Schegloff \footcite{Schegloff1990} emphasized that sequence structure is central to the social organization of talk, as it allows participants to manage and negotiate interaction effectively.

    \subsubsection{Topic shading}
    \par
    Topic shading, as discussed by Sacks \footcite{topicshading}, refers to the subtle way in which a conversation naturally shifts from one topic to another while maintaining coherence. Instead of abruptly changing the subject, speakers introduce a related idea or concept, gradually steering the discussion in a new direction. This process allows for smooth transitions in dialogue, helping participants maintain engagement and avoid confusion.

    \subsubsection{Dis/preferred answers}
    \par
    Preferred answers, according to Sacks \footcite[p.~410]{Sacks1992}, are responses in conversations that align with social norms and expectations, making interactions smoother and more cooperative. In conversation analysis, preferred answers typically follow the format or intent of the preceding question or statement . They contrast with "dispreferred" answers, which might include refusals or disagreements and often require additional explanation or mitigation to maintain social harmony.

    \subsubsection{Conversational repair}
    \par
    Conversational repairs refer to how participants address and resolve problems in understanding, hearing, or speaking during interactions \footcite{sacksRepair}.
    These issues, can occur at any point in a conversation. Repairs are classified into self–repair, where the speaker corrects their own error, and other–repair, where a different participant addresses the issue. They can further be classified into self–initiated repair and other–initiated repair.

\par
%–––
As a descendant of conversation analysis interaction linguistics has emerged,
building on its insights to examine language use in social contexts.
It broadens the focus to study not only verbal exchanges but also
multimodal cues like gestures, gaze, and intonation,
analyzing how these elements contribute to meaning.
Interaction linguistics aims to understand the dynamic aspects of conversations,
such as how topics shift and how sequences of speech acts unfold,
reflecting the fluid nature of human communication.

\section{Disruptions in conversation coherence}
\par
While the question of what makes for a coherent text is too broad,
the answer to what makes for a coherent conversation can be somewhat easier to answer.
Because conversation participants negotiate understanding,
it is up to them, when a conversation is and is not coherent
to describe what a coherent conversation is,
it is worth pursuing the moments, when the conversation stops flowing with ease.
Such moments can be called coherence disruptions.
A coherence disruption is a complex phenomena as

\begin{itemize}
\item
it penetrates through some or all of mentioned perspectives on an on–going conversation
\item
it can't be evaluated in a binary fashion
\end{itemize}

There are different degrees to which a conversation coherence can be disrupted:
\begin{itemize}

\item
   if a participant suddenly starts speaking
    in an a way that can hardly be considered interaction
    due to its irrelevance or

\item
   if the utterance simply is not grammatical or understandable,
    while the conversation has been compromised and becomes incoherent,
    it has more to do with incoherent written text, because
    the incoherence is encapsulated on the level of a single utterance
\end{itemize}

Roberts \footcite{Roberts01101993} Discusses various types of incoherent text.
He exemplifies so called giberish as incoherent text that is absent of structural relations.
On the other hand he discusses experimental theater or literature as a type of text
which is assumed to be coherent in the sense that there is an intention behind it
but contains little to no structural relations.
Lastly he mentions a so called "schizophrenic discourse" as a speech that
is not assumed to be coherent even if it has structural relations to it.
In any case Roberts definitively states that coherence is assumed
and is therefore a receptive phenomena.
The incoherence Roberts discusses is considerably different from when
the source of incoherence stems from
    the structure of the conversational text or
    relationship between different utterances
    – this is when another participant assesses,
        they are simply speaking leading a different conversation
        perhaps with a differing intention
        or that they are conversing under differing set of circumstances
        which manifests formally in the linguistic fabric of the conversation – its text.
        All that despite everyone included being cooperative.

\subsection{Sources of incoherence in conversation}
\par
%–––
Schegloff shows how incoherence arises when
people interpret sequence structure differently,
namely in terms of which turn is seen as an answer to which previously occuring turn.
In his example, the participants misread each other’s intentions,
leading to confusion about how their turns fit together.
They each project different expectations for how the conversation should unfold,
which causes misaligned sequence structure interpretations.
When this happens,
they turn to brief metacommunication — comments about the conversation itself
to try to clarify and re–align their understanding.
Schegloff illustrates how these
efforts to "repair" the misalignment are central to
managing and resolving incoherent moments in conversation.

\par
Coherence disruptions are also discussed in literature.
%–––
Hrbáček’s approach to coherence and cohesion in text distinguishes the two concepts,
noting how they often interact but can also be independent.
He highlights that while
cohesion involves grammatical or lexical links that
make sentences flow together,
coherence relies on the logical and meaningful progression of ideas.
This means that a text could be cohesive –
using connectives, repetitions, and consistent lexical choices –
yet lack coherence if the sequence of ideas doesn’t
make logical sense or follow a clear progression.
Conversely, a text may be coherent in its narrative flow without
relying heavily on cohesive devices.
In Czech linguistics,
the distinction between téma (theme) and réma (rheme), as used by Daneš,
underlines the role of topic progression.
Hrbáček illustrates this by discussing examples where
a story progresses logically from one point to the next while
being incoherent despite being clear about its topic structure
due to never coming back to a previously mentioned topic.

\par
Two kinds of phenomena are at hand when it comes to
ways in which conversation coherence can be disrupted –
topic shifts and nonassignable anaphore.
While not unique to conversation
both take on specific forms in it worth looking at.

..()[Bublitz] disturbed coherece, unclear reference, topic drifts

\subsubsection{Topic shifts}
\par
    When conversations shift abruptly from one topic to another,
    it can create confusion for the conversation partner.
    They might find themselves trying to
    reconnect to the previous discussion or
    wondering how the new subject relates.
    This can lead to misunderstandings
    as the transition can feel jarring.

\par
    One interesting question is,
    how do we determine when a topic has run its course?
    What common traits do conversations share when a subject is truly exhausted?
    Perhaps observing transcripts could reveal repeating patterns in topic progression or sequence structure.

\par
    Moreover, what makes for a smooth transition between topics?
    Is it related to the cues participants give each other,
    or perhaps the context of the discussion?
    How do we navigate the flow of conversation and
    what indicates a natural shift versus a disruptive one?

\subsubsection{Nonassignable anaphore}
\par
    nonassignable anaphore is closely tied to topic progression.
    Currently established topic or topics help assigning anaphore and
    determining between an anaphore and an exophore.
    Even if an anaphoric device is not assignable,
    and the reference is presumably an exophoric one,
    The reason for employing this reference must be
    relevant to an established topic.
    In conversation meaning of demonstratives is to be negotiated.
    If an anaphores assignability causes confusion,
    chances are it is caused by one of the following

\begin{itemize}
\item
there are no relevant assignment candidates

    \quad
    this situation can be understood as a vague or unjustified exophore

\item
there are multiple equally relevant candidates

\item
candidate has occured in the conversation text too long ago

    \quad
    can be understood as an abrupt return to previously established topic
\end{itemize}

%–––
\subsection{What do people do about coherence disruptions?}
\par
In conversation, coherence disruptions often prompt participants to
employ strategies to maintain understanding and flow.
Schegloff suggests that people manage these disruptions through interactive repair or inference.
Interactive repair often involves
explicitly addressing misunderstandings or clarifying intentions,
often by rephrasing or asking questions.
Interactive repair refers to immediate, collaborative corrections within dialogue,
where one speaker might correct the other or themselves to enhance clarity.
Inference and pragmatic reasoning, the most seamless methods,
allow participants to fill gaps based on context and social cues,
helping conversations continue smoothly without explicit repair.

\par
Dingemanse and Enfield \footcite{DINGEMANSE202430} echoes this from a cognitive perspective,
highlighting how inference and pragmatic reasoning are particularly effective.
Participants rely on shared understanding and contextual knowledge to interpret ambiguous statements.
Together, they use both
explicit (metacommunication and repair) and
implicit (inference and reasoning)
methods work to restore coherence.

\par
It needs to be noted however that both interactive repair and reasoning are
deployed in a number of other contexts
other than conversation coherence disruption.
Inference takes place constantly \footcite{garfinkelstudies}.
Each of those moments could be hardly considered a coherence disruption.
There is however always potential for it,
particularily via unclear or nonassignable anaphore or abrupt unjustified topic shifts.
Repair and metacommunication also takes place in a mutually informed and synchonized interaction.
It is for example deployed when it is revealed
that the interaction participants intentions or opinions differ.

\par
These uses of interaction management are however
hardly possible to analyse on a textual level
since they do not cooccur with coherence disruptions.
What can be observed are – as mentioned above –
troublesome anaforic references and topic progressions.


% _____________________

\chapter{Experimental framework}

\section{What are chatbots?}
\par
A chatbot is a dialog system powered application simulating conversation with a user.
An attempt to make a machine conversate with a human user requires capturing the essence of human speech.
What the essence speech is and what capturing it means are not defined and
some definitions of what essence of speech is
require specific definitions of what capturing it means.
(Semantics and DialogueSchlangen, David2015) ..
In case of the history of chatbots,
the essence of speech is partially achieved
via mimicking it.
The intention was to have a user interact with a chatbot
that would communicate so well that
the user would be convinced
this is a another human they are talking to.
Whether just that has been achieved would be measured by a so called turing test
proposed by Alan Turing in 1950 \footcite{turing1950computing}

\par
Initial attempts at making a computer converse were rule–based \footcite[p.~43]{Sacks1992}
What that means is the content of the chatbot utterances
would be predetermined
and there would be a decision tree that would decide what to say next.
In the early days as well as often times in modern day systems
string matching would be used to analyse user input.

\par
ELIZA \footcite{weizenbaum1966eliza} is regarded as a milestone
what it was, it pretended to be a therapist
hiding behind general phrases
doctor authority

as long as interaction frame is strictly defined and
the robot has some level of authority in it
the rule–based approach can work
granted, it requires a lot of manual design
and constant maintenance
but it is used nowadays

\par
Machine learning moved things forward in many ways
Fuzzy matchin allowed for close matches
without the neccesity to predefine the exact sequence of characters.
IBM Watson uses classifiers build on examples \footcite{building_watson_2010}
structure remains rule based
reasoning about these rules is decided to a large degree by classifiers

\par
The recent breakthrough pushed another thing in the mainstream
it is now possible to generate near natural speech
this gives the possibility to just let the conversation be taken over by one answer generator
this way we lose tight control over what it does though
For some use cases
like open domain conversation or accessing knowledge base
that is not an issue.

\par
And so currently chabot interface text generators are very prevalent.
In 2024 this technology is now closer to beating the Turing test than
any other model or approach before it\footcite{jones2024peopledistinguishgpt4human}
by having 54\% of participants thinking
they are talking to a human.
While Eliza convinced 22\% of participants,
actual humans only convince 67\% of participants.

\subsubsection{Turn taking in chatbot interactions}

\par
Even if the Turing test is passed,
really fitting simulation of conversation
can only be achieved if the low–level conversation mechanisms
are simulated, like turn taking \footcite{optimizing-turn-taking}.

\par
As established in previous chapter, turn taking is a crucial aspect of conversation.
The way participants distribute who is to talk
explains exhaustively
the difference between
a structure of the text of conversation
and a single–producer text.
The mechanism of turn taking differs
between actual human conversation
and an interaction between a chatbot and a user.

Interaction between chatbot and user
typically take place in a strict fashion
where both participants,
human and virtual,
have unlimited time to come up with the next answer.
While the chatbot should be optimized to answer as fast as possible,
the user has as much time as they need until fallback.

\par
\say{Research in sociolinguistics, psycholinguistics, and
conversational analysis has revealed that
turn–taking is a mixed–initiative,
locally coordinated process, in which
a variety of verbal and nonverbal cues such as
eye gaze,
body pose,
head movements,
hand gestures,
intonation,
hesitations, and
filled pauses
play a very important role.
We continuously produce and monitor each other for
these signals and can coordinate seamlessly
at the scale of hundreds of milliseconds
across these different channels
with multiple actors.} \footcite{turntaking}

\par
People are capable of producing and picking up clues
that indicate opportunities for turn taking flawlessly.
There is a some way ahead for robots in this regard
whether it is figuring out the correct time to start speech \footcite{turntakingreview} \footcite{GRAVANO2011601}
or actually creating a system that will be able to produce such behavior \footcite{distributedturntaking} \footcite{Gervits2020Sigdial}.
This research field
has the potential to push conversation technology
closer to true conversation simulation.

\section{Convform}
\par
An exploration has been carried out using a custom tool called Convform \footcite{convform}.

At its core Convform is a computer program
which accepts a configuration, user input and context
and determines next chatbot answer.
Other than that it offers a collection of utilities
to help design and run chatbots.

\subsubsection{Participant facing chat interface}
In order to handle the inputs, convform provides a chatting environment
for the participants to interact with a chatbot.
The convform environment differs from a usual chat log
because it does not display
the entire history the conversation.
In attempt to simulate spoken conversation
it only displays the last chatbot response.
This way the participant has to rely on their memory
in taking part in the conversation.
Other than that the participant may enter their next response
and send it.
They are also instructed to end to conversation by a red button
if the chatbot behaviour is "unnatural" (nepřirozené)
After the conversation is over whether it has been ended by the user or the chatbot,
there is a questionare which
asks the participants to rate how "natural" the conversation was
and mark and comment on utterances in the now fully displayed conversation.

\subsubsection{Conversation design tool}
Lets admin user create chatbots and define their behavior
the behavior can be defined by string matching rules or prompts
it is capable of working as a statemachine or a single state
it provides a level of control over references within the design

\subsubsection{Testing and debugging of various conversation contexts}
while designing chatbots it is necessary
to be able to simulate various situations
to fine tune various possible scenarios
that might occur in the conversation.
To achieve this, there must be a way
to encode required context to convform.
The convform chatbots use a conversation status (CStatus) object
to represent their current understanding of the conversation.
It contains information about the history of the conversation
which in conjunction with the configuration file and user input
helps determine the next response.
The configuration file is static
CStatus changes automatically
User input comes from the user.
This conversation status can simulate any possible conversation context
from the chatbots perspective
For testing and debugging specific contexts, convform allows admin user
to tweak the conversation status

\subsubsection{Accesing the conversation data}
Lastly convform naturally includes a convenient way to read user interactions
and browse associated conversation status objects

\section{Conversation design in theory}

% https://ieeexplore.ieee.org/abstract/document/9447005

\par
Designing the behavior of a dialog system
is referred to as conversation design\footcite{kolosova2022} \footcite{mctear2020conversational} \footcite{cxd}.
It is not the course of any one conversation that is being designed here
but rather as many possible ways any conversation could go
for a given use case.
Conversation design as a profession is deeply connected
with the rule–based approach that has been used in ELIZA.
Maintaining all the possible utterances and
rules under which they would be uttered
in commercial dialog systems
has proven to be a responsibility large enough
to generate jobs.

A conversation designer operates between
the business logic and use case of the dialog system
the clients, customers or users interacting with the system
and the developers maintaining the system.

\subsection{Rule–based approach}

In order to be able to design a rule–based dialog system,
one needs to be able to encode the following:

    \begin{itemize}

        \item
        The possible utterances, that the dialog system can produce – "the mouth"

        \item
        Rules under which the next utterance will be chosen – "the ear"

    \end{itemize}

\par
If the conversation is supposed to be a state machnie e.g.
it needs be able to use different sets of rules
under different contexts in the conversation.
This way a dialog system can be context aware to a degree.
A conversation design of this sort
can be displayed as a diagram.
Then a way to maintain context of conversation is also necessary.
This context needs to encode rules to choose an immediate ruleset
which helps determine the next utterance.
This principle is a simplification of
how people decide what they will say next in conversation.

\subsubsection{Pros and cons}

\par
This approach to designing a dialog system
has been the standard for decades.
It offers a granular control over how a conversation should go.
In case of the state machine variant it
allows to guide the user through a relatively complex process.
It however suffers from how unpredictable the user can be.
It is up to the conversation designer to cover all the possible ways of answering
which not only is hardly possible
but also poses a neccesity to parcel the spectrum of possible answers
which can generate conflict when
a user input semantically spans across multiple determined categories.
This issue is even stronger while using the string matching approach,
because there the string literal can decide about the following dialog system answer
as if meanings and their speech representations were a one–to–one map,
which they are not.
Even if a certain meaning is included in a ruleset,
the system might not grasp the meaning and react in an incoherent way.
With the state machine the
distribution of various rules across various rulesets
requires big effort.
Extending the capabilites of a rule–based dialog system
hardly scale and tend to have regressions.
In case of dialog systems relying on user input by speech transcription
the text input processed by the system is not guaranteed to represent
what the user actually said.
In conclusion rule–based approach to conversation design
provides control over the dialog system behavior
but tends to be inflexible.

\subsection{Statistically driven approach}

Some of the issues tied to rule–based systems
are resolved using another approach.
Especially in recent years the breakthroughs in the field
of speech generation have been significant ()
allowing for letting the dialog system play a bigger role
in what is being said next.
In its simplest form,
it is possible to just let the the answer be generated "end–to–end".
The user input is sent to a model which generates an answer.
It has been convincingly shown that this technology has the capability
of reacting in a flexible way to much of what is being thrown at it ().
While not perfect ()
this technology is capable of staying on topic (), mirroring ()
and other things that make for a coherent conversation.

\subsubsection{Large language models}

The main component that is responsible for
this way of simulating conversation at this level of flexibility
are so called large language models ().
These models, powered by advanced neural networks,
have revolutionized the field of natural language processing.
Among the most influential architectures are transformers,
which enable these models to handle
vast amounts of text data and capture complex patterns of meaning, context, and grammar.
They use their training data to generate the next most probable token.

and then it takes also its output and predicts the next probable output

llms for other things than conversation

The Generative Pre–trained Transformer family of models,
exemplifies the capabilities of LLMs.

Researched in – exponential improvements over short period of time.

including ChatGPT, presented in ... ()

These systems are trained on diverse datasets containing billions of words,
allowing them to generate coherent and contextually relevant responses across various topics.
This flexibility has made them increasingly mainstream,
being integrated into tools for writing, education, customer service, and more.

Nowadays there are multiple publicly available LLM services.

Unlike traditional rule–based systems,
LLMs rely on deep learning techniques to process and predict language,
enabling them to understand nuanced queries and provide human–like responses.
This adaptability has set new standards for conversational AI,
making it a valuable resource in numerous industries.

\subsubsection{Prompt engineering techniques}

The rapid advancement of LLM technology
has outpaced research into optimal interaction strategies.
Understanding how to engage effectively
with these systems has been a developing area (),
which illustrates both their power and their novel nature.
The foundational idea is:
an LLM might perform nearly any task if prompted correctly.

The quality of the task differs significantly task to taks in reality,
but the anything is possible approach has proven to break new grounds
when leveraging LLMs' capabilities.

Over time, researchers and practitioners have developed techniques for
crafting effective prompts to optimize outputs.
The simplest approach is known as "zero–shot" prompting,
where a user poses a direct question or request without additional context.
However, zero–shot prompting may not always yield the desired depth or accuracy.
It is common for the model to "misunderstand" the assignment
and generate tokens so that it will "confuse" itself
and lead to generated answer in a completely irrelevat direction.

More sophisticated strategies include "few–shot" prompting,
where examples are provided to guide the model's response style or focus.
This way there is a reference for the structure of the answer
and there is a protection to the answer leading somewhere it is not meant to.
Since analogy is a task LLMs are doing really well in ()
framing the task as an analogy can help improve the output significantly.

Another very prevalent way that has proven to
improve the performance of LLMs is a so called
chain–of–thought (CoT) prompting.
It encourages the model to
articulate its reasoning step–by–step,
enhancing logical accuracy.
There are many ways to achieve this,
but the primary one is a few–shot approach
where a description of the logic is
explicitly described.
The model is then prompted to produce
a similar chain of thought (hence the name)
and end the answer with the sought after information.

This principle can be further improved by
chaining several LLM calls and having one
evaluate the previous one.
Such strategy has been used for the GPT o1
which has proven to surpass other models
in available metrics ().

\subsubsection{Pros and cons}

Using large language models as core component
of dialog systems brings resolution to many issues
rule–based systems introduce.

An LLM powered dialog system is flexible in understanding
the user input.
The user input is processed in a way much more sophisticated
than the shallow string–matching approach.
While the classifier approach is a lot more capable to understand,
it is still forced to choose a predefined answer, whereas
an LLM has the capability to taylor an answer for every input.
It can do this in a way that would be very hard to come up with
especially in advance with the help of a conversation designer,
leveraging the fact that LLM is primarily a text generator
and only functions as a component in a dialog system.
It can be relatively well controlled as
it can accept complex instructions as to how to behave
and these instructions can be tuned in runtime.

Systems of this sort however introduce their own set of problems.
The biggest issue are so called hallucinations ().
Factuality is a challenge for LLMs overall.
Being programs that are to output text no matter what,
there have been instances of asking them questions,
that do not have a correct answer ().
It has been shown, that LLMs have an issue
knowing that they do not know something.
Recognizing that is the case requires
an extra level of reasoning
that is an object of research as of recently ()[dark matter of ai]

Since these models are their own agents,
us humans also need them to be ethically aligned with us ().
That can prove challenging since a lot of ethical problems exist,
that do not have a simple answer.
Alignment however means that an LLM must always side with humans.
This is an ongoing research field which has been making some
troublesome observations recently with the most intelligent models available ().

Even if all the programming and training is done to the most benefit of humans,
information technology is succeptible to be broken by malicious action.
In case of LLMs we talk of so called jail breaks ().
LLMs being trained on vast amount of data,
they hold knowledge that can be illegal or unethical to spread
like steps to create explosives for example.
The typical examples of jail breaks are ways to manipulate
the LLM to give out this information
which under regular circumstance it would not give.

With all this in mind,
how much of a responsibility do we want to allow LLMS to have?
Considering the direction our civilization is going in
LLM powered technology is expected to be making decisions
that will affect people on a daily basis.
Hopefully that will not happen before
the challenges of alignment and hallucinations
are solved.

As far as low–stakes open–domain conversation simulation goes,
LLM powered chatbots are relatively safe
though there are cases of dangerous or tragic situations
especially for vulnerable individuals ().

This is why for dialog systems that are supposed to achieve
anything else on top of the conversation itself,
if they are meant to be powered by LLMs,
a regulating structure needs to be placed on top of the LLM.

\section{Conversation design in practice}

\par
Conversation design in Convform attempts
to combine elements of rule–based design with text generation.
It allows creating a purely rule–based chatbots
which analyse the user input based on string matching
and say exactly what they are prescribed to.
On the other hand it also allows to make the chatbot
understand the user input by adding it in a prompt
and answer using a generated response.
Both these approaches can be combined in various ways.
Other than that, convform also allows to predetermine
the chatbot personality for the entire conversation.
The building blocks of a convform chatbot are
states and intents which are
analogical to the previously mentioned "mouth" and "ear"
of the chatbot.

\subsubsection{State}
A state is an object which carries several pieces of information
bundled together.
At its core it contains the utterance of the chatbot
whether it is a hardcoded one or a prompt component which is to be called.
A state however also contains information about
which intens to listen to in the next user input,
which states to add automatically to the next response
and other navigation instructions like this one.
Each convform status associated with a response
can contain multiple states.
This is to make convform generate more complex answers
which can react in a flexible way.
However it also comes with a challenge to order these responses correctly
and make sure that they are not contradicting each other content–wise.
Ultimately a master prompt with multiple components
regulating itself via analysis prompts
might be a smarter direction to go.

\subsubsection{Intent}
An intent is an object representing a category of a user response.
It contains the information to determine whether
user input fits in given category and
the state or states to respond with next.
Just like in state, the information about whether the user input
corresponds with the intent can be encoded via
string–matching patterns or a prompt.
As mentioned intent is a problematic concept,
because it forces an outside logic and categorization
on user input, which might not be able to fit well
in the framework declared by the current intent set.
It is however also the necessary evil since
it is the only way for a conversation designer
to peek into what is going on in the conversation
and to direct the dialog system in the correct way.

\par
This way a convform chatbot can be created,
that will be instructed to lead from one state to another
make decisions based on intents
while being able to use any combination of
hardcoded responses and intent patterns
and descriptions of responses or user inputs used in intents.
Detailed description of how convform works can be found
in the wiki of its github repository \footnote{https://github.com/almarsk/convform/wiki}

\subsubsection{Coherence}
With support of LLM powered responses
convform can be used to simulate
an open–domain conversation with a user
and simultaneously using a combination
of intents and prompting
a convform chatbot can be created
that will act incoherently under a predefined set of conditions
allowing to create experimental stimuli.
First however, regular conversation needs to be achieved using convform.

\subsubsection{Conversation style}

To simulate conversation, it is first necessary to simulate a persona.
The persona can then have a simulated motive to conversate
which can interest the user enough to engage in interaction with the dialog system.
For rule–based systems, persona can be defined ahead of time
and it can manifest itself via the specific writing of the hardcoded responses
that the system is able to give.
With generated responses, the persona of the chatbot has to be included in the prompt.
The personality of LLMs and conversation technology more broadly
is being discussed\footcite{gpttoxicity}\footcite{robopersona}.
The general characteristics of a machine talking to a human are typically
friendliness, helpfulness and submissivity.
For conversation research with convform,
the goal is to achieve just that.
The chatbot persona needs to be friendly,
polite and curious.
It needs to be able to keep the conversation going
but not change topic too often.
It needs to be able to add a little bit of its own perspective.
\par
The conversation style e.g. the amount of participation and initiativity in conversation
is something people adapt in to their conversation counter–part.
Since developing a system that would immitate this behavior
requires additional effort
and expands scope beyond the coherence research
this paper focuses on,
this approach to conversation design
has not been taken here.
Instead two versions of conversation style
have been developed
and distributed evenly between participants.

\par
The initial conversation style used in the experiments
represents a curious and friendly chatbot
who is instructed via prompt to ask lots of follow up questions.
This tends to result in a conversation that moves forward in its topical structure
in way deemed incoherent by Hrbáček\footcite[p.~30]{hrbacek1994}.
It depends on the participants impression whether it would be perceived as
curious and initiatve or shallow and dismissive.

\par
A second version of conversation style has been introduced
to get some insights on participants acceptance and
the course of conversation itself.
This one would interleave topical questions with remarks on the topic
The intention behind this would be to slow down conversation tempo
and give the participant the opportunity to bring their own initiative.

\subsubsection{Prompting}

special prompts used to achieve things beyond regular conversation
all in czech including prompts
czech versions of prompts will be in annex

\par
Entity recognition

In order to track entities that could be referred to
a few–shot prompt was deployed that would help
keep track of which entities have been mentioned.
Since GPT4o, the model used in the experiment,
tended to consider too many things an entity,
most examples are negative and do not capture an entity.
It also contains some repetition as a result of fine–tuning the best wording.

..
(New or Old? Exploring How Pre–Trained Language Models Represent Discourse Entities)

Loose english translation of the prompt goes as follows:\\
\linespread{1.5}
{\tiny
What is an entity? a person, a thing; may not be animate;\\
 It is always a noun and most nouns are an entity in the speech;\\
 Look out! Pronouns such as "you", "he" and the like are never entities even if they represent persons.\\
 Verbs are certainly by no means entities even if they represent a person.\\
 nouns which are not entities in a speech are very generic or temporal;\\
 a subject in a sentence to which reference may be made by a personal or reference pronoun;\\
 the utterance usually has one entity, sometimes two, and rarely more. often the utterance lacks an entity altogether.\\
 The participants in the conversation are never entities in any way.\\

 example:\\
 Alice likes speedball the best.\\
 The entities in this sentence are ["Alice", "speedball"].\\
 example:\\
 We're meeting tomorrow at 2:00.\\
 This discourse lacks an entity, so the output is [].\\
 example:\\
 What do you like?\\
 This discourse lacks an entity, although there is a personal pronoun in it, the output is therefore [].\\
 example:\\
 How are you, Carl?\\
 This speech lacks an entity, because the communicator is never an entity, so the output is [].\\
 example:\\
 I just hope it doesn't get too cold.\\
 This speech lacks an entity, because the atmospheric phenomenon is never an entity, so the output is [].\\

 Consider which words are entities in the next sentence:\\

 \[user input\]\\

 Consider the meaning of each noun in a sentence and consider whether it could represent a specific entity,\\
 and not just an abstract concept.\\
 Consider all possible objects that could be mentioned in the speech\\
 and consider whether their names could be considered entities.\\
 Remember that verbs and pronouns are not entities, even if they represent persons.\\
 Remember that the conversationalists are not entities.
}

\par
anaphora in conversation
..
(Annotating anaphoric phenomena in situated dialogue) but here we try to generate it instead

to be able to create conversation designs which contain various types of anaphore,
convform first needs to be able to give a response that
has an anaphoric reference to an entity from the previous conversation in it.

The GPT4o model used for this use case
does not tend to generate sentences with anaphores in them.
Instead it will rather mirror the entity phrase.

The anaphorization prompt therefore tasks the model
to modify a generated response so
that the mirrored entity is replaced with
an anaphoric device.
For this another few–shot prompt was used to modify a just generated response.
Its loose english translation is as follows:\\

\linespread{1.5}
{\tiny
Find one main word in a sentence and swap it for a personal or reference pronoun.\\
 Leave the other topic centers as they are.\\
 Don't forget to omit words that may be part of the name phrase of the replaced word.\\
 Don't forget that when a noun is replaced by a pronoun, you often need to change the verb sequence –\\
 the verb will then often be at the end of the sentence; the order of the pronouns must be followed –\\
 the replacement pronoun will come for the reversible and the personal pronoun. You also need to correctly recognize the more important word –\\
 choose the one about which the question is.\\
 To preserve the naturalness, you sometimes need to modify a sentence\\
 to include a secondary sentence, especially if the replaced word is linked to a deverbative noun.\\

 Example1:\\
 sentence:\\
 And what will your paper be about?\\
 consideration:\\
 the replaced word will be a seminar. The word "yours" belongs to the name phrase of the replaced word.\\
 your answer:\\
 And what will it be about?\\

 Example2:\\
 sentence:\\
 how far from your house is your favorite park?\\
 consideration:\\
 the substituted word will be "park," is more important in a sentence than the word "house." the word "yours" belongs to the name phrase of the substituted word as well as the word "favorite."\\
 your answer:\\
 How far from your house is it?\\

 Example3:\\
 context:\\
 I'll go to the cafe\\
 sentence:\\
 what is your drink in the cafe?\\
 reasoning:\\
 the substituted word will be "cafe" because it is in context. because of the natural word sequence, you will need to move the substitution pronoun.\\
 your answer:\\
 What is your drink there?\\

 Example4:\\
 sentence:\\
 Do you have any tricks for ironing shirts quickly and efficiently?\\
 reasoning:\\
 the substituted word will be "shirt" and because the sentence is complex and you cannot easily move the verb, a secondary sentence will solve it.\\
 your answer:\\
 Do you have any tricks for ironing them quickly and efficiently?\\

 Example5:\\
 sentence:\\
 What track do you most enjoy racing on?\\
 your answer:\\
 Which one do you most enjoy racing on?\\
}

This prompt has been tuned to catch as many tricky cases as possible.

examples description ..


\subsection{Stimuli}

With these tools
multiple chatbots were created
that would generate conversation situations
which serve the role of experimental stimuli.
Participant reactions to these stimuli
can be then compared.
This way conversational experimental designs can be created.
There are three types of stimuli created
for the purpose of this paper.
They are shallow anaphore, deep anaphore and nonassignable anaphore.

\subsubsection{Shallow anaphore}

A shallow anaphore is a kind of anaphore
where the referent of the anaphoric device
should be relatively easy to map
as opposed to a deep anaphore.
The referent will always occur
in the preceding utterance of the participant.
This type of stimuli is relatively simple to achieve
in convform
generating a response
and using the anaphorization prompt on it afterwards.

Shallow anaphore is common in regular conversation ()
and should not pose a problem for participant to understand
in a conversation with chatbot.
It should therefore not have an impact
on the user acceptance of the chatbot
and should generally go unnoticed.
It is regardless worth using as stimuli
for a reference unproblematic case
that still requires the same kind of processing
as other more interesting stimuli.

Example: \\
Participant: I love coffee.
Chatbot (not anaphorized): What kind of coffee do you like best?
Chatbot (anaphorized): What kind of it do you like best?

\subsubsection{Deep anaphore}

A deep anaphore is a situation
where the referent of the anaphoric device
occurs several utterances ago.
The depth is not measured by number of occurences
but by number of new entities that occur
since the referent which the anaphore refers to.
Measuring depth of anaphore by the number of utterances
does not capture the dynamic nature
of topic progression in the text of the conversation.
The number of utterances does not map
on how many topics have been visited.
While the number of new entities does not
map exactly either,
it is first of all a lot closer to the topic progression
and second of all actually very close to
what is being sought after here –
how far in the conversation is an entity
still acceptable or even available to speakers.
An entity is can represent a topic
but can also be one of several entities to represent a topic
or can cover several topics at once
all depending on which way the covnersation goes.

\par
As stated earlier,
both topic and entity are difficult to define
and their annotations tend to be recursive.
A close–enough approach has been adopted in this paper.
While runtime topic annotation by an LLM
is not necessary for generating deep anaphore
and has therefore not been attempted in this paper,
entity tracking is made possible my
entity recognition prompt.

This prompt runs in parallel with the next response generation
and writes down its results in the conversation status.
A chatbot that contains the deep anaphore stimuli
chooses a participant mentioned entity
relatively early on in the conversation
(though not at the very beginning)
and then tracks new mentioned entities.
When there have been 4 new entities mentioned,
the next response generation prompt will be modified.
The modification lies in that the context of the conversation
that has so far taken place
will be cut so
that the chatbot only has access
to the conversation until the point of the mention of the entity.
Given the response generation prompt
the response will contain a question
about the mentioned entity.
Then the only thing that needs to be done
is modifying the response via the anaphorization prompt.

The trick here the participant and the chatbot differ in their perspectives
on what the conversation currently is.
The chatbot refers to something
that from the perspective of the participant
has been mentioned a while ago.

This approach is relatively imprecise and relies on luck to a certain degree.
Compared to the shallow anaphore however
it is expected to be somewhat more problematic
and perhaps cause the participant to request clarification.

Example ..

\par
There has been one issue that has arisen while developing this stimuli
that has proven to alter the character of the data in an unwanted way.
Since the chatbot has no access to the conversation that happens
between the occurence of the referred to entity and the participants present moment,
chances are the chatbots question will be on
a piece of information that has been mentioned in the meantime.
Whenever that happens the degree of participant acceptance decreases significantly
due to a topical incoherence rather than
due to struggling to mapping a deep anaphore.
This has been dealt with via providing the chatbot
with the rest of the conversation in another component of the prompt
with the instruction to avoid any of the topics mentioned there.
LLMs are known to handle negative instruction with less success than positive ones ()
but this measure seems to have mitigated the problem
as can be seen in the data attached in the annex.

\subsubsection{nonassignable anaphore}

The last type of stimuli used in this paper
is called a nonassignable anaphore.
It is a device that the participant will tend to
interpret as an anaphoric device,
typically a personal or demonstrative pronoun,
but one such that the participant will not be able
to map to any of the candidate entities
in the previous conversation text.
This stimuli is expected to lower the participant acceptance
by the greatest amount.

To make a chatbot contain this stimuli
entities are tracked to make sure
there are candidates to be considered
in case an anaphore occurs.
Once there is a sufficient number of
entities recognized in the conversation
a hardcoded response is returned instead of an LLM generated one.
The response contains a pronoun that to make sense of
participant needs to interpret as an anaphore.

Since the response containing the stimuli is hardcoded,
there is no guarantee
that is actually is incoherent with the previous conversation
and that there is no candidate to map the anaphore on to.
Though odds are high enough
every conversation that is supposed to contain this stimuli
will have to be manually checked
to confirm the required stimuli is present.
This will be the case for all the conversations regardless
because presence of stimuli is not guaranteed
for shallow and deep anaphore either.

An approach not explored in this paper is
achieving a nonassignable anaphore is also possible
via generating a response
using a prompt that instructs an LLM
to come up with a question containing an unrelated entity
avoiding all mentioned entities
and anaphorize it before showing it to the participant.
Although LLMs tend to perform worse with negative instructions ()
this could be achieved using a chain of prompts.
The notion of nonassignable anaphore
brings into scope the question of
what makes an anaphora assignable.
It is the semantic compatibility of the words around the anaphore
that determine which of the candidates the anaphore is referring back to.
The generation of the lexical surroundings of the anaphore
needs to be handled carefully
when coming up with an LLM based response.

\subsection{Ending the conversation}

While recognizing when the conversation is ending
or especially when it should not end
people rely on a set of clues
similarly to knowing when to take turns speaking ().
In open–domain conversation
like the one a convform chatbot holds with experiment participants,
the main challenge is to recognize
when there is a topic at hand
that interests the participant.
Another discipline in the realm of ending the conversation
is recognizing it is a good time to end the conversation
due to the participants lack of interest.
Conversation designs made for this paper
do not take much of this into account.
The main goal for a convform chatbot here
is to present the participant with a stimuli.
Once they manage that,
if the participant is willing to continue
the conversation continues for a hardcoded number of responses.
This leads to participants sometimes noticing
the conversation ending abruptly
and mentioning they would like to continue
in the questionnare
or even at the very end of the conversation itself.
This can potentially have effect on
the score given by the participant
and therefore brings noise into this parameter.
In terms of quantization
conversation is inherently noisy.
Since each conversation has to be manually checked,
an assessment of how to deal with this noise can be made
while and after processing the data.
A runtime topic annotation
and other prompting techniques
could potentially help make
the convform chatbot converse in such a way
that would be more aware of the general course of the conversation
perhaps giving hints about ending the conversation ahead of time
or reacting to and handling the participants hints of the same type.

\chapter{Data}

The data collected using a convform chatbot
is a transcript of the conversation
between a participant and the chatbot.
Depending on the conversation design of the given chatbot
the conversation may contain a record of
the participant being exposed to a specific situation
and their reaction to it.
Other than that the collected data contains
an information about whether the participant quit the conversation,
the participants rating of the conversation
and their comment on it.

Unfortunately it cannot be guaranteed
the required stimuli actually occurs in the conversation.
Though the probability is relatively high,
the LLM technology responsible for most answers
is nondeterministic
and the participants tend to be unpredicatble.
On many levels conversation can take an unintended direction
which can spoil the stimuli.
Whether thematic, textual or interactional,
anything can go wrong.
That is why as mentioned earlier,
each conversation needs to be visited manually
to confirm required stimuli is present.
That does not present too much of a problem
since a qualitative analysis of the data
has the potential to bring lots of
corrections to experiment design,
relevant observations or
inspiration for further steps.
It would however pose a complication
was one interested in scaling up this approach
that would have to be dealt with.



% _____________________



\section{Data collection}

\subsection{Experimental designs}

Data collection took place with the use of six chatbots.
They each represent a possible combination of a 3x2 design.
The two design variables are anaphora type and conversation style.

The anaphora types are as discussed above a shallow, deep and unassignable.
The conversation styles were two. One, inquisitive,
where the instruction part of the prompt
would always instruct the LLM to pose a question.
The other, relaxed, would interleave this question instruction with
an instruction to comment on the conversation in a nonquestion way.
The conversation style variable was deployed for purely explorative purposes.

All the chatbots had the same persona instruction except their names.
They would communicate in Czech with participants who were Czech or slovak native speakers.
The first utterance in the conversation would be the chatbots and
would be hardcoded to contain greeting and self introduction.

They would be instructed to chat for several turns and then the stimulus would come.
After the stimulus the chatbot would continue for a couple more turns and then say goodbye.
The questionnare would then appear to the participant.
That is unless the participant aborted earlier, in which case the questionnare would appear immediately.
In some cases participants simply left the user interface leaving no extra information.
Other then that, participants were asked to grade the conversation on scale 1 to 5
where 1 was most acceptable and 5 was least acceptable.

Timing of the prompt would differ based on the anaphora type.
For shallow type the chatbot simply conversates for two turns and thereafter it starts tracking entities.
At first recognized entity the prompt fires off
prompting an utterance about the exctracted entity and anaphorizing it.
With deep anaphora similar thing happens,
but the chatbot waits several more entity occurences
before also prompting an utterance about the extracted entity and anaphorizing it.
Lastly the nonassignable anaphora has the same timing method as the shallow anaphora,
but the prompt itself is hardcoded to contain an anaphoric reference which is
very unlikely to be assignable.

In the experiment the participant would be instructed to end the conversation,
if the chatbots communication wasnt natural.
The concrete Czech word used was "přirozená komunikace".

\subsection{Participants}
The data has been collected in two waves.
The pilot wave generated 50 conversations
the participants for which were recruited from the networks of the papers author.
In the attached full data
the identificators of these conversations have
the letter p for pilot appended at the end.
The second wave generated 325 conversations
the participants for which were recruited from the students of the Charles University.
They were instructed to take two conversations with a week pause in the middle.
However since for this experiment explorative in nature contrast and timing were not critical
they were free to revisit the chatbot however they liked.
There is no attempt made to indicate unique participants in the conversation as
this would not bring any observable results.
If such approach should be taken in the future,
more control has to be gained over the course of the converstaion
as is shown below.
No changes were made to the design for the second wave other then changing the structure of the qustionnare.
While in the pilot wave was tasked to make a comment on the conversation or
explain why they aborted it if they did so,
the second wave was shown the entire conversation
and was asked to mark and comment on specific chatbot responses they found odd or interesting.

\section{Data analysis}

Since the collected data is relatively complex
multiple layers of analysis need to be performed on it.
The steps of analysis will be following:

\begin{itemize}
\item{Conversation design stimuli annotation}
\item{Participant reaction quantitative analysis}
\item{General qualitative analysis}
\end{itemize}

In order to assess the datapoints for quantitative analysis
each conversation needs to be visited
to confirm or deny the presence of expected stimuli.
This preliminary step of qualitative analysis
provides feedback on whether the conversation design
that is meant to facilitate the experimental design
was succesful in doing so.
Further analysis leading up to a quantitative assessment
should only be done on conversations
where convform succesfully exposed the participant to the stimuli.
It has been observed in collected data
that sometimes an unintended stimuli
takes place instead of the intended one.
Then simply changing the label of the conversation is appropriate.
It is also necessary to filter the collected data for
noncooperative participants who make it impossible for convform
to perform the stimuli in the first place.
The participant needs to take on a role of a friendly conversator
if the experiment is to work.
If they for example attempt to take advantage of the LLMs obedience
and give it an unrelated task that involves any sort of text generation,
they need to be pronounced noncooperative.

The qualitative analysis of the participant reaction to stimuli
is mostly of explorative nature.
The convform environment lets the user to input any text
which even under the condition that they are cooperative
can be unexpected and can derail the conversation.
While these incidents are typically anecdotal
and cannot be used to make generalizations
they needs to be taken into account
as a possible participant behavior.
From the perspective of dialog system development
such cases would be considered to be edge cases
and dealing with them on the conversation design level
is typically considered lower priority.
However the conversation analysis perspective
will be very interested in all the potential paths that
a certain conversation situation can go in.
This is ultimately why any weaknes to the conversation design
does not matter too much
as all conversation should be visited anyways
in search for unique situations.

In conclusion the main reason for using
an experiment environment solution like convform
can be summed up in two points:

\begin{itemize}
\item{Partial control over what happens in the conversation}
\item{The ability to expose multiple people to a comparable conversational stimuli}
\end{itemize}

The first point has been described in detail above.
As far as the second point goes,
this approach is a step towards
a quantitative analysis of conversation.
Though conversation is multifactored
and various types of situation
always come with a different set of circumastances
if a quantitative analysis of a set of conversations
containing a certain reoccuring pattern
proves contrastive in some of the quantifiable parameters
to a set of conversations containing a different reoccuring pattern
case could be made this is caused by the observed patterns
regardless of all the noise naturally present in a conversation text.

\section{Annotation}

All 375 conversations were visited to confirm whether
the stimulus expected based on the conversation design.
Other than that the participant reaction was annotated to be either
continuation, metacommunication or aborting of the conversation.
These three types of participant reaction to stimulus
represent the amount of understanding the reference.
In simple terms, continuation means
the stimulus went under the participants radar
or has been accepted.
Metacommunication, an utterance which somehow addresses the ongoing communication,
shows the participant not understanding,
but perhaps attributing the misunderstanding to themselves or
believing the issue can be resolved.
Lastly the conversation is aborted when the participant looses trust
in the ongoing conversation being coherent now or in the future.
Then the illusion of social interaction falls
and the conversation the participant is having dissolves into
typing text into a computer program and getting text in return.
The immersivness necessary for simulating a conversation is here no longer achieved.

\subsection{Inter--annotator agreement}

To make sure that annotating the data is possible,
inter-annotator agreement has been measured on 37 conversations which gives 10% of the data.
Three annotators annotated each conversation for anaphora type and
whether the participant reaction to stimulus is metacommunicative.
Metacommunication is the only reaction type that needs manual annotation since
aborting the conversation is
observable in the transcript and marked in the data automatically
and any reaction that is not aborting and metacommunication has to be a continuation.

\subsubsection{Stimulus annotation}

For anaphora type Cohen's Kappa was calculated for each of the three annotator pairs
with following results:

\begin{itemize}[label={}]
    \item Cohen's Kappa for Annotator 1 and Annotator 2: 0.89
    \item Cohen's Kappa for Annotator 1 and Annotator 3: 0.88
    \item Cohen's Kappa for Annotator 2 and Annotator 3: 0.92
\end{itemize}

Coming to the average Pairwise Cohen's Kappa of 0.90
the inter-annotator agreement appears to be high enough
that this annotation categorization to be used and trusted.

Each of the conversations that caused disagreement is interesting in its own right:


\begin{quote}
\begin{itemize}[label={}, leftmargin=0pt, itemsep=0.5em]
    \item \textbf{conversation id: 145}
    \item ...
    \item \utterance{b:}{Jaké dárky plánuješ koupit?}{Which gifts do you intend to buy?}
    \item \utterance{h:}{ještě nevím, potřebuji pro taťku nějaký dárek a pro mamku}{don't know yet, need for dad a gift and for mom}
    \item \utterance{b:}{To může být výzva, ale určitě najdeš něco skvělého.}{That can be a challenge, but surely you'll find something great.}
    \item \utterance{h:}{jo, taky si myslím}{yeah, i also think}
    \item \utterance{b:}{Jaké koníčky má tvůj táta a mamka?}{Which hobbies does your dad a mom have?}
    \item \utterance{h:}{nevím, neznám je}{don't know, don't know them}
    \item \utterance{b:}{To je škoda, můžeš se jich zeptat.}{That is a shame, you can ask them.}
    \item \utterance{h:}{bydlí moc daleko}{they live too far}
    \item \utterance{b:}{Bot: To je těžké, ale co si zjistit jejich zájmy?}{That is tough, but how about finding out what their interests are?}
    \item \utterance{h:}{taky nevím§}{also don't know}
    \item \utterance{b:}{Možná bys mohl zjistit, co mají rádi.}{maybe you could look into, what they like}
    \item \utterance{h:}{ok}{ok}
    \item \utterance{b:}{To je dobrý nápad, jak to uděláš?}{That's a good idea, how will you do it?}
    \item \utterance{h:}{zeptáím se tety, ta určitě bude vědět}{I'll ask my aunt, she'll surely know}
    \item \utterance{b:}{Jaké mají oni koníčky?}{Which hobbies do they have?}
    \item \utterance{h:}{kdo?}{who?}
    \item ...
\end{itemize}
\end{quote}

This conversation potentially has a deep anaphora in it,
since the first occurence of the mappable entity \textit{dad and mom} is deep in the conversations history
at the stimulus time.
There are however other entites that are mappable
that are mentioned in the meantime like \textit{aunt}
or the general group of \textit{close ones} mentioned before the snippet starts.
This points at the potentially missing category of \textit{ambivalent anaphora}.
Another issue with this stimulus is also that it askes a question that is already being discussed.
This can be seen as a both topical and sequentical coherence disruption.
Why would the same question with the same reference in it be posed?
That also makes the anaphora that much more vague and leads to a metacommunication \textit{who?}.

\begin{quote}
\begin{itemize}[label={}, leftmargin=0pt, itemsep=0.5em]
\item \textbf{conversation id: 150}
\item ...
\item \utterance{h:}{Běhám 400m překážek a dělám sedmiboj}{I run 400m with obstacles and do heptathlon}
\item \utterance{b:}{To je náročné! Jak dlouho už se tomu věnuješ?}{That's demanding! How long have you been doing this?}
\item ... 7 turns (14 utterances) ...
\item \utterance{h:}{Jsem po zánětu v patě, tak snad se to nevrátil, hodně to uvolňuji a cvičím na to}{I had a heel inflammation, hopefully it won't return, I release and excersise it a lot}
\item \utterance{b:}{Jaké disciplíny tvoří on? Jakou máš mezi nimi nejoblíbenější?}{Which disciplines does it consist of? Which is you favorite among them?}
\item \utterance{h:}{Kdo?}{Who?}
\item ...
\end{itemize}
\end{quote}

Here the mappable entity \textit{heptathlon} is so far deep in the conversation that
the annotator who would not agree with calling a deep anaphora stimulus
perhaps did not even notice it or insisted that this is just too deep.
While that is technically against the annotators instruction based on the above description of stimuli
but it points at the interesting problematic of anaphora depth.

The other two conversations that had an annotation disagreement about them were IDs 151 and 153.
Both of them have an imperfect generation in them that excludes them from the stimuli containing data.

\subsubsection{Metacommunication annotation}

Inter-annotator agreement regarding the metacommunication has produced following results:

\begin{itemize}[label={}]
\item Cohen's Kappa for Annotator 1 and Annotator 2: 0.94
\item Cohen's Kappa for Annotator 1 and Annotator 3: 0.88
\item Cohen's Kappa for Annotator 2 and Annotator 3: 0.94
\end{itemize}

With the average Pairwise Cohen's Kappa coming to 0.92,
it can be considered reliable to annotate metacommunication
in the collected data.

The two conversations that had disagreement about them in terms of
presence of metacommunication in the participant reaction to stimulus
are both interesing in their own right.


\begin{quote}
\begin{itemize}[label={}, leftmargin=0pt, itemsep=0.5em]
\item \textbf{conversation id: 164}
\item ...
\item \utterance{b:}{Jakou písničku z ní máš nejraději?}{Which song from here do you like the best?}
\item \utterance{h:}{z rádia?}{from the radio?}
\item ...
\end{itemize}
\end{quote}

This reaction can be seen to contain an ellided metacommunicative device like \textit{do you mean}.
Since it is not present explicitely and the participant reaction is topically anchored in the conversation,
the case is ambivalent.

\begin{quote}
\begin{itemize}[label={}, leftmargin=0pt, itemsep=0.5em]
\item \textbf{conversation id: 173}
\item ...
\item \utterance{b:}{Tak to by mě zajímalo, jak na to pak ona zareagovala.}{Then I wonder, how she reaction to that.}
\item \utterance{h:}{Ano, pokazilo}{Yes, it went wrong}
\item ...
\end{itemize}
\end{quote}

It is unclear where the participant is coming from in their response.
For this reason, it is also unclear whether the participant reaction is a metacommunication.
There is a good chance the participant is somehow commenting on the chatbot breaking,
in which case such utterance would indeed be considered to address the ongoing communication.

\par
In conclusion the annotator agreement is
strong enough for both stimulus type and metacommunication reaction and
quantitative analysis can be performed on the measured metrics.

\chapter{Quantitative Analysis}

\section{Expectations}

In the performed quantitative analysis,
following expectations were used as reference to the found results.

For user rating the rating was expected to be ordered like so from best to worse:
shallow, deep, nonassignable.

As far as aborting the conversation goes,
The amount of aborted conversations were expected to be ordered
from least to most als:
shallow, deep, nonassignable.

The reactions were expected to map onto the stimuli like so:

\begin{quote}
\begin{tabbing}
\hspace{4cm} \= \hspace{4cm} \= \kill % set up tab stops
Shallow \> Continue \\
Deep \> Meta \\
Nonassignable \> Abort \\
\end{tabbing}
\end{quote}

As far as conversation style, the inquisitive variant was expected to have a worse score
since every next question shifts the topic forwards, which should generate a topical progression
that sooner or later becomes incoherent as Hrbáček demonstrates ().

\section{Result}

\subsection{Ratings}

The collected ratings have the following distribution:

\begin{quote}
\begin{tabbing}
\hspace{4cm} \= \hspace{4cm} \= \kill % set up tab stops
\textbf{rating} \> \textbf{times} \\
Rating 1 \> 22 \\
Rating 2 \> 62 \\
Rating 3 \> 138 \\
Rating 4 \> 37 \\
Rating 5 \> 6 \\
No rating \> 19 \\
\end{tabbing}
\end{quote}

The average rating of the collected conversations
which have been rated by participants is 2.77.
Since the missing ratings are participants who clicked off of the conversation
the absence of their rating can be interpreted as the worst rating.
Then the average rating of all the conversations is 2.97.

The instruction to the participants was to rate the conversation
using the same system as the grading in czech public school system.
That may explain the majority of ratings being a 2 or a 3.
The ratings in the collected data can only be compared against each other and
for them to give a better picture of the stimulus effect,
a more complex experiment design would need to be deployed
containing a reference conversation and its rating for each participant.

\subsection{Stimulus and reaction}

The annotation result is as following:

\begin{quote}
\begin{tabbing}
\hspace{4cm} \= \hspace{4cm} \= \kill % set up tab stops
\textbf{anaphora type} \> \textbf{times} \\
Shallow \> 104 \\
Deep \> 64 \\
Nonassignable \> 116 \\
Other \> 85
\end{tabbing}
\end{quote}

Ignoring the conversation that lack a rating the result is following:

\begin{quote}
\begin{tabbing}
\hspace{4cm} \= \hspace{4cm} \= \kill % set up tab stops
\textbf{anaphora type} \> \textbf{times} \\
Shallow \> 102 \\
Deep \> 58 \\
Nonassignable \> 105 \\
Other \> 71
\end{tabbing}
\end{quote}

There is less conversation with the deep anaphora stimulus.
This is because the deep anaphora stimulus requires the most difficult operation by the dialog system
and is therefore the most likely to fail.
The annotation result also confirms its necessity
given almost a fifth of the conversations were marked as \textit{other} and
thereby excluded from the stimulus reactions quantitative analysis.

\subsection{Design rating}

The rating for the various conversation types came out to be following:

\begin{quote}
\begin{tabbing}
\hspace{4cm} \= \hspace{4cm} \= \kill % set up tab stops
\textbf{anaphora type} \> \textbf{rating} \\
Shallow \> 2.75 \\
Deep \> 2.53 \\
Nonassignable \> 2.95 \\
\end{tabbing}
\end{quote}

Opposed to expectations shallow anaphora stimulus type has worse score than deep anaphora.
P-value from Kruskal-Wallis test being 0.02 is sufficiently low, but
when values stimulus types are compares post-hoc via Dunn test,
result come to:

\begin{quote}
\begin{tabbing}
\hspace{4cm} \= \hspace{4cm} \= \kill % set up tab stops
\textbf{anaphora types} \> \> \textbf{p-value} \\
Deep \> Nonassignable \> 0.005 \\
Deep \> Shallow \> 0.158 \\
Nonassignable \> Shallow \> 0.136 \\
\end{tabbing}
\end{quote}

The only sufficiently low p-value is between deep and nonassignable anaphora.
The results only partially correspond with the expectaions.
Number rating is however noisy as other things happen in conversation that affect it.
The chatbot contributions to the conversation outside of the stimulus
has to be under control and produce natural responses in order to really be able
to view the ratings as reflecting the stimuli
hence the high p-values.

\subsubsection{Design rating with abandonment interpretation}

Some conversations do not have a rating because participant clicked off of the experiment webpage.
Since the participants were instructed to leave the conversation if it is not \textit{natural},
these conversations can be interpreted as worse possible reaction - 5.

We can further split these into a situation where
the participant clicked off of the experiment
right after the stimulus.
This only happened in case of nonassignable anaphora design.
Its new average score comes to 3.06.

If the participant left the conversation at any time after the stimulus,
this can be because the stimulus disturbed the conversation coherence
and while it was not serious enough at the stimulus time for the participant
to end the conversation,
the coherence never recovered.
If we interpret conversations abandoned at any point after stimulus as rated with the worse possible rating,
we get following results:

\begin{quote}
\begin{tabbing}
\hspace{4cm} \= \hspace{4cm} \= \kill % set up tab stops
\textbf{anaphora type} \> \textbf{rating} \\
Shallow \> 2.80 \\
Deep \> 2.77 \\
Nonassignable \> 3.15 \\
\end{tabbing}
\end{quote}

Shallow anaphora that is expected to have the best result still gets a slightly worse rating than the deep one.
The differences between these ratings and the ones where abandoned conversations are excluded are:

\begin{quote}
\begin{tabbing}
\hspace{4cm} \= \hspace{4cm} \= \kill % set up tab stops
\textbf{anaphora type} \> \textbf{difference} \\
Shallow \> -0.04 \\
Deep \> -0.23 \\
Nonassignable \> -0.19 \\
\end{tabbing}
\end{quote}

The difference shows that the shallow anaphora had by far the least number of abandoned conversations.
Updated significance check provides a slightly different picture with
Kruskal-Wallis test p-value being a similar value, but
the post-hoc Dunn test showing more significant differences:

\begin{quote}
\begin{tabbing}
\hspace{4cm} \= \hspace{4cm} \= \kill % set up tab stops
\textbf{anaphora types} \> \> \textbf{p-value} \\
Deep \> Nonassignable \> 0.016 \\
Deep \> Shallow \> 0.653 \\
Nonassignable \> Shallow \> 0.019 \\
\end{tabbing}
\end{quote}

Here the results are more significant but still not enough for deep vs shallow anaphora.
In conclusion the shallow anaphora was not rated as expected in relation to the other stimuli, but
the difference does not seem to be significant and
in accordance with the expectations has the lowest number of abandoned conversations.
There are many factors in the conversation data and
stimulus does not seem to have a strong enough effect on the rating.
If better more seamless conversation could be simulated,
perhaps stronger effect of the stimulus on rating could be observed.
The small difference can also be attributed to the fact that participants tend to
avoid extremes in their grading.

\subsection{Reaction to stimulus}

A quantitative metric that is expected to show better the acceptability and understandability
of the various stimulus designs is the direct conversational reaction to them.
The annotated data came to the following result:

\begin{quote}
\begin{tabbing}
\hspace{4cm} \= \hspace{3cm} \= \hspace{3cm} \= \hspace{3cm} \= \kill % set up tab stops
\textbf{anaphora type} \> \textbf{continuation} \> \textbf{meta} \> \textbf{abort} \\
Shallow         \> 91 \> 8 \> 5 \\
Deep            \> 34 \> 26 \> 4 \\
Nonassignable   \> 13 \> 57 \> 46 \\
\end{tabbing}
\end{quote}

Which percentage wise gives the following:

\begin{quote}
\begin{tabbing}
\hspace{4cm} \= \hspace{3cm} \= \hspace{3cm} \= \= \hspace{3cm} \kill % set up tab stops
\textbf{anaphora type} \> \textbf{continuation} \> \textbf{meta} \> \textbf{abort} \\
Shallow         \> 87.50\% \> 7.69\% \> 4.81\% \\
Deep            \> 53.12\% \> 40.62\% \> 6.25\% \\
Nonassignable   \> 11.21\% \> 49.14\% \> 39.66\% \\
\end{tabbing}
\end{quote}

With the extremely low p-value coming to $P < 10^{-28}$
 there is no doubt
that the different stimuli have a clear effect on the participant reaction and
that the results will be replicable.
The expectations are met with the continuation being highest for shallow anaphora
and lowest for nonassignable anaphora while
the abort reaction proves to have the opposite tendency.

One unpredicted feature in the results is
deep anaphora having a relatively close percentage of continuation and meta reactions and
nonassignable anaphora sharing similar values for meta reactions and abort reactions.
This suggests there is a inner division in the data that can be described and simulated more closely.

\subsubsection{Stimulus x Reaction specific ratings}

Looking at ratings for specific groups of stimuli, results are following:

\begin{quote}
\begin{tabbing}
\hspace{4cm} \= \hspace{3cm} \= \hspace{3cm} \= \= \hspace{3cm} \kill % set up tab stops
\textbf{anaphora type} \> \textbf{continuation} \> \textbf{meta} \> \textbf{abort} \\
Shallow \> 2.73 \> 3.12 \> 2.60 \\
Deep \> 2.43 \> 2.62 \> 2.75 \\
Nonassignable \> 2.45 \> 3.04 \> 2.98 \\
\end{tabbing}
\end{quote}

Interestingly for shallow and nonassignable anaphora,
the ratings do not follow the expected course.
The unfulfiled expectation lies in the meta reaction having worse ratings than the abort reactions.
The only anaphora type the expectation were met for is the deep one.
Yet the significance of this result is low and
more control over the conversation outside of the stimulus needs to be acquired
to be able to rely on these results.
To make conclusions about the meta reaction being indicative of
a less acceptable experience in participant
would be hurried.

\subsection{Comments}

While annotating, participants comments were collected
to gain an extra metric on the reflection of the stimuli.
While the shallow anaphora was mentioned once due to a generation issue
and generally went by unnoticed,
the deep anaphora got 11 mentions, which makes for
slightly less than a fifth of conversations with this stimulus.
Finally the nonassignable anaphora was mentioned 69 times in the participant comments.
This makes over half of its occurences were commented upon.
All the comments expressed confusion about mapping the anaphora to a possible preceding referent.
This result follows the expectations.

\subsection{Conversation style}

As mentioned above, there was an attempt
to expose different participants to different conversation styles.
This came out of the neccesity to create a default way for the chatbot to communicate.
First style is the inquisitive, where the chatbot is prompted to ask questions.
An elaboration on it is a style of conversation,
where the chatbot is at one time prompted to ask questions and
at another to simply comment on the topic at hand.
There were 183 conversations collected with the chatbot communicating in an inquisitive style
and 128 conversations where the chatbot communicated in the second, more relaxed style.
The relaxed style has been expected to have a better score as was confirmed
with following result:

\begin{quote}
\begin{tabbing}
\hspace{4cm} \= \hspace{3cm} \kill % set up tab stops
\textbf{conversation style} \> \textbf{rating} \\
Inquisitive \> 2.63\\
Relaxed \> 2.45 \\
\end{tabbing}
\end{quote}

However the Kruskal-Wallis test p-value of 0.10 suggests too small effect
to consider this result replicable.
To find statistically significant effect on conversation styles
the conversation needs to be controlled in a more nuanced way by the dialog system.

The relaxed style however better rated on average also suffered from some issues
in terms of generating natural answers.
Sometimes the chatbot would simply repeat the participants response without adding anything new to it.
For example in following conversation lack of topical input lead to aborting the conversation:

\begin{quote}
\begin{itemize}[label={}, leftmargin=0pt, itemsep=0.5em]
\item \textbf{conversation id: 164}
\item ...
\item \utterance{b:}{To chápu, co bys chtěl v histologii zlepšit?}{I understand, what would you like to improve about histology}
\item \utterance{h:}{chtela bych ji nemit proste}{I would just like to not have it}
\item \utterance{b:}{To chápu, chtěla bys ji prostě nemít."}{I understand, you would just like to not have it.}
\item \textit{aborted}
\end{itemize}
\end{quote}


\subsection{Zero anaphora}

Since Czech is a prodrop language (),
it is expected that where the stimulus anaphora stands in subject position,
there will be tendency of it being dropped.
A response with zero anaphora \footcite{zero} would then be expected to be more acceptable
than an equivalent response with an explicit one.

During annotation there have been found to be only 16 conversations with zero anaphora stimulus.
This is partially due to the prompting of the stimulus,
the few-shot prompt contained no example of a zero anaphora and
partially because a lot of the stimuli contained an object positioned anaphora.

Coincidently the stimuli containing zero anaphora were split equally between shallow and deep anaphora stimuli.
The nonassignable anaphora stimuli mostly consisted of a hardcoded response that already contained the anaphora
therefore it is not a part of the zero anaphora collection of conversations.

\begin{quote}
\begin{tabbing}
\hspace{4cm} \= \hspace{3cm} \kill % set up tab stops
\textbf{reaction type} \> \textbf{\# occurences}\\
Continuation \> 14 \\
Meta \> 2 \\
Abort \> 0  \\
\end{tabbing}
\end{quote}

This result very clearly shows that the zero-anaphora stimuli were mostly going by unnoticed
and have a higher level of acceptability.
The same goes for rating with following result:

\begin{quote}
\begin{tabbing}
\hspace{4cm} \= \hspace{3cm} \kill % set up tab stops
\textbf{anaphora type} \> \textbf{rating}\\
Zero \> 2.19 \\
Nonzero \> 2.73 \\
\end{tabbing}
\end{quote}

Only shallow and deep anaphora type stimuli were used for this calculation.
The zero anaphora score is a lot better considering the range of averages
that the data has produced so far.
This shows future designs need to take zero-anaphora into account more
in order to achieve stronger contrast between stimuli.

The data contains conversations where stimulus was
inappropriately generated with a nonzero anaphora like here:

\begin{quote}
\begin{itemize}[label={}, leftmargin=0pt, itemsep=0.5em]
\item \textbf{conversation id: 347}
\item ...
\item \utterance{h:}{...hráli jsme je přes aplikaci ale já mu k vánocům koupila pravou šachovnici, takže konečně můžeme hrát normálně.}{... we would play via an app but I bought him a chessboard for christmas so we can finally play normally.}
...
\item \utterance{b:}{Jakou máte ji, klasickou nebo něco speciálního?}{Which do you have her, classic or something special?}
\item \utterance{h:}{jak to myslíš?}{what do you mean?}
\item ...
\end{itemize}
\end{quote}

The meta reaction is likely caused by the depth of the anaphora,
but the bot response would have been more likely had the anaphora not been a part of it.

In some cases zero anaphora can also be inappropriate like here:

\begin{quote}
\begin{itemize}[label={}, leftmargin=0pt, itemsep=0.5em]
\item \textbf{conversation id: 357}
\item ...
\item \utterance{h:}{Nooo teď jsme jeli kolem břevnovského kláštera, nikdy jsem tam nebyla, tak bych zajela tam.}{Well we just drove past Břevnov monastery, I never visited, so I would go there.}
\item \utterance{b:}{To zní zajímavě! Co tě na Břevnovském klášteře láká nejvíce?}{That sounds interesting! What about Břevnov monastery is the most attractive for you?}
\item \utterance{h:}{To nevím právě, překvapilo me to jak je to velký.}{I don't know, it surprised me how big it is.}
\item \utterance{b:}{Máš rád objevování nových míst a památek?}{Do you like discovering new places and monuments?}
\item \utterance{h:}{Ráda bych na to měla více prostoru.}{I would like to have more space to do it.}
\item \utterance{b:}{To chápu! Jaké místo bys chtěla objevit jako první?}{I get that! Which place would you like to discover first?}
\item \utterance{h:}{No nevím ale třeba bys mi mohl doporučit nějaký hezký místo v Česku spjato i s nějakou novou moderní architekturou, možná nějaká vyhlídka.}{I don't know but maybe you could recommend a nice place in Czechia connected with some new modern architecture, maybe a lookout tower.}
\item \utterance{b:}{Slyšela jsi o historii a zajímavostech?}{Have you heard about history and trivia?}
\item \textit{aborted}
\end{itemize}
\end{quote}

The participant commented after the conversation that the question did not fit in with the context
which is why they aborted.
That is reasonable as the response was generated based on the monastery topic from deeper in the conversation.
Perhaps if there was an explicit anaphora here, the mapping would have been easier
as dropping in a nonsubject position is not preferred in Czech.
Regardless, the hard topical shift that disrupts the coherence
caused by the chatbotwould remain in the conversation.

\chapter{Qualitative observation}

Visiting each conversation during annotation
gave a detailed insight on the nature of the data,
success of generating the correct stimuli and
creating the sought after situations.
The following observations about
the experiment design, the various stimuli and conversation coherence in general
were made.

\section{Pro-coherence perception}

Most noticeably the data shows how participants communicate with a cooperativity in mind
and assume a lot in order to maintain the conversation flow.
The quality with which the chatbot could generate responses was high enough,
that the illusion of social interaction would hold up even if minor issues did come up.

\subsection{Remaps}

In following conversation, the participant maps an anaphora on a possible candidate
even though the hardcoded response was meant to be nonassignable:

    \begin{quote}
    \begin{itemize}[label={}, leftmargin=0pt, itemsep=0.5em]
    \item \textbf{conversation id: 100}
    \item ...
    \item \utterance{h:}{mám jednu mourovatou doma}{I have a tabby one at home}
    \item \utterance{b:}{Jak se jmenuje tvá mourovatá kočka?}{What is the name of your tabby cat?}
    \item \utterance{h:}{to nevím}{I don't know}
    \item \utterance{b:}{Jak dlouho už ji máš doma?}{How long have you had it at home?}
    \item \utterance{h:}{nemám vlastně}{I don't actually}
    \item \utterance{b:}{Jak ses k ní tedy dostal?}{So how did you get it?}
    \item \utterance{h:}{ráda peču}{I like to bake}
    \item \utterance{b:}{Jaké oblíbené dobroty obvykle pečeš?}{Which treats do you usually bake?}
    \item \utterance{h:}{dorty}{cakes}
    \item \utterance{b:}{Tak to by mě zajímalo, jak na to ona potom zareagovala.}{Well I'd like to know how did she react to it afterwards.}
    \item \utterance{h:}{myslíš tu kočku?}{You mean the cat?}
    \item ...
    \end{itemize}
    \end{quote}


This conversation along with several others is the reason annotation was necessary for this type of stimulus.
The tolerance on participant side was sometimes very high, even if the anaphora was incongruent
with the assumed mapping or even if
some of the reference contained in the hardcoded stimulus does not map perfectly
like in following conversation:

    \begin{quote}
    \begin{itemize}[label={}, leftmargin=0pt, itemsep=0.5em]
    \item \textbf{conversation id: 257}
    \item ...
    \item \utterance{b:}{Stále je co zlepšovat, přemýšlel jsi o trenérovi?}{There is always room to improve, did you consider a trainer?}
    \item \utterance{h:}{na dvou hodinách už jsem byl s trenérem}{I already did two lessons with a trainer}
    \item \utterance{b:}{Tak to by mě zajímalo, jak na to ona potom zareagovala.}{Well I'd like to know how did she react to it afterwards.}
    \item \utterance{h:}{že musím lépe používat nohy, pohyb vychází z kyčle}{that I have to use my legs better, the movement start in the hip}
    \item ...
    \end{itemize}
    \end{quote}

On this regard, a general third person seems to be acceptable,
though from the annotation point of view is should be nonassignable as can be seen here:

    \begin{quote}
    \begin{itemize}[label={}, leftmargin=0pt, itemsep=0.5em]
    \item \textbf{conversation id: 36}
    \item ...
    \item \utterance{b:}{Co tě nejvíce zaujalo v této knížce?}{What did you find the most interesting in this book?}
    \item \utterance{h:}{Nádherné fotografie minerálů pod mikroskopem.}{Beautiful photos of minerals under a microscope.}
    \item \utterance{b:}{Jaké další knihy o minerálech ti doporučili?}{What other books about mineral did they recommend?}
    \item \utterance{h:}{Většinou mi nikdo nic nedoporučuje, hledám si knihy sám.}{Usually nobody recommends me anything, I look for my books alone.}
    \item ...
    \end{itemize}
    \end{quote}

In some cases the nonassignable anaphora stimulus has the potential
to be interpreted as a third person reference to the participant.
Such cases seem to have been tolerated by the participants as well
though in human-human conversation these would probably cause metacommunication.

    \begin{quote}
    \begin{itemize}[label={}, leftmargin=0pt, itemsep=0.5em]
    \item \textbf{conversation id: 276}
    \item ...
    \item \utterance{b:}{Co tě dneska tak unavilo?}{What exhausted you today?}
    \item \utterance{h:}{konverzace s klučičím pohlavím}{conversations with the male gender}
    \item \utterance{b:}{To zní náročně, o čem jste si povídali?}{That sounds demanding, what did you talk about?}
    \item \utterance{h:}{o škole}{about school}
    \item \utterance{b:}{Tak to by mě zajímalo, jak na to ona potom zareagovala.}{Well I'd like to know how did she react to it afterwards.}
    \item \utterance{h:}{docela dobře}{pretty well}
    \item ...
    \end{itemize}
    \end{quote}

In case of the deep anaphora stimulus, one of the big reasons annotation was necessary
was also the risk of a remap.
Generating the deep anaphora response, the LLM is only shown the appropriate section of the conversation.
This exposes the generated and anaphorized response
to the risk of another mapping candidate to have been mentioned in the meantime like in the following conversation:

    \begin{quote}
    \begin{itemize}[label={}, leftmargin=0pt, itemsep=0.5em]
    \item \textbf{conversation id: 259}
    \item ...
    \item \utterance{h:}{ano, mám - zaklínač 3, kingdomcome, nebo dota 2}{yes I do - Wither 3, kingdomcome or dota 2}
    \item \utterance{b:}{Zaklínač 3 má nádherně propracovaný svět a silný příběh.}{Withcer 3 has an elaborate world and a strong story.}
    \item \utterance{h:}{přesně tak, je to úplně jiný svět než ten náš, takže je to skvělá relaxace.}{exactly, it is a completely different world from ours, so it is a great relaxation.}
    \item \utterance{b:}{Do které části herního světa Zaklínače 3 se nejraději vracíš?}{To which part of the Witcher 3 world do you like to return the most?}
    \item \utterance{h:}{Jako které místo se mi tam líbilo? Novigrad je moc hezký}{As in which place there I liked? Novigrad is very pretty.}
    \item \utterance{b:}{Kterého hrdinu v ní hraješ nejraději?}{Which hero do you play most often in her? (dota 2 is feminine in Czech)}
    \item \utterance{h:}{No tak tam není moc široký výběr hrdinů, Geralta}{Well there is not a broad selection of heroes, Geralt}
    \item ...
    \end{itemize}
    \end{quote}

Since multiple videogames were mentioned,
the participant interpreted the question to be about the most recetnly mentioned one,
\textit{Witcher 3} which has also been topicalized by the chatbot.
The chatbot was prompted to ask about the \textit{dota 2} where also the question makes more sense.

Due to stimulus generation issues, sometimes the reference is misunderstood.
In following conversation the stimulus reference the the \textit{african continent}
but the participant answers in a more broad way.
They even mention this misunderstanding later in a comment.
This means during the conversation, the anaphora is nonassignable and general enough,
that the participant continues the conversation even if it entails a topic shift.

\begin{quote}
\begin{itemize}[label={}, leftmargin=0pt, itemsep=0.5em]
\item \textbf{conversation id: 243}
\item ...
\item \utterance{b:}{Dnes jsem se dozvěděla, že existuje více než 7 000 jazyků.}{Today I learned there is over 7000 languages.}
\item \utterance{h:}{to je hodně zajímavé, už jsem to taky slyšela. Mám pocit, že třetinou z nich se mluví na africkém kontinentu}{that is very interesting, I also heard about this. I think that third of them are spoken on the african continent.}
\item \utterance{b:}{Máš nějakou oblíbenou zemi nebo jazyk na něm?}{So you have a favorite country or language on him? (continent is masculine in Czech)}
\item \utterance{h:}{mám docela ráda anglii a angličtinu, co ty?}{I like england and english, what about you?}
\item ...
\end{itemize}
\end{quote}


\subsection{Pseudoanaphora}

    hardcoded phrase with intended nonassignable anaphora is sometimes assignable by association

    An anaphora which lacks an explicit candidate mapping does not necessarily need to be nonassignable.
    Such pseudoanaphora () also occurs in the collected data:

    \begin{quote}
    \begin{itemize}[label={}, leftmargin=0pt, itemsep=0.5em]
    \item \textbf{conversation id: 248}
    \item ...
    \item \utterance{h:}{Byla jsem v nemocnici za pacienty}{I went to the hospital to see the patients}
    \item \utterance{b:}{To muselo být zajímavé. Jaká byla tvoje nejcennější zkušenost?}{That must have been interesting. What was you most valuable experience?}
    \item \utterance{h:}{Je jich hodne}{There was a lot}
    \item \utterance{b:}{To zní dobrodružně, co tě tam dneska konkrétně zaujalo?}{That sounds adventurous, what specifially interested you today?}
    \item \utterance{h:}{Zvetsena slezina}{enlarged spleen}
    \item \utterance{b:}{Znamenalo to pro něj nějaké specifické komplikace?}{Did it for him entail some complications?}
    \item \utterance{h:}{No…je to priznak lymfomu}{Well... it is a symptom of lymphoma}
    \item ...
    \end{itemize}
    \end{quote}

    Here the chatbot response contains an anaphora which
    cannot be mapped onto a specific mentioned entity in the previous text
    but is understandable as there is clearly a person at hand in the situation the participant describes.
    The participant has then no issues continuing the conversation and
    moving the topic forward by answering the chatbots question.

    A spectrum could be shown of anaphoric devices
    that stems from a pseudoanaphora which is clearly related to a mentioned entity
    though lacks an explicit antecedent and
    a newly mentioned entity which then has the potential to retrospectively be pronounced a cataphora
    if it is lexically realized later on in the conversation.

\subsection{Sequential coherence}

Without directly provoking it via experiment design,
some conversations show a pro-coherence perception on the participant side
in terms of the sequental structure () of the conversation:

    \begin{quote}
    \begin{itemize}[label={}, leftmargin=0pt, itemsep=0.5em]
    \item \textbf{conversation id: 123}
    \item ...
    \item \utterance{h:}{Boj s rakovinnými buňkami znáš?}{Do you know the fight with cancer cells?}
    \item \utterance{b:}{Jak se rozsivky konkrétně podílejí na boji s ní?}{How diatoms specifically participate in the fight against it?}
    \item \utterance{h:}{Ano.}{Yes.}
    \item ...
    \end{itemize}
    \end{quote}

Here the participant answer hints at a misinterpretation of the chatbot response.
While the chatbot is instructed to ask open-ended questions to keep the conversation going,
the participant seems to understand the question in the chatbot response
as a request for confirmation which is also given.
This is likely because the chatbot question adds new topical content with its question
which allows for interpreting the question as demonstrating a level of knowledge about the topic
which the participant inquires about in the first question.

\section{Topical (in)coherence}

    %too simplified a notion
    %remaps
    %lexical surroundings - verb
    %perfect

    The deep anaphora stimulus type had the highest error rate.
    The annotation showed that only 64 conversations of this type were collected.
    That is approximately half of how many conversations were collected of the shallow and nonassignable anaphora.
    This relative failure has mainly been caused by the remaps discussed above.
    In its ideal form it looked something like this:

    \begin{quote}
    \begin{itemize}[label={}, leftmargin=0pt, itemsep=0.5em]
    \item \textbf{conversation id: 253}
    \item ...
    \item \utterance{h:}{Začínám se připravovat na zkouškové období na vejšce, protože už se kvapem blíží}{I am starting to prepare for the examp period in my university, because it is coming up quickly.}
    \item \utterance{b:}{Jaké předměty ti dají nejvíc zabrat?}{Which subjects are the hardest?}
    \item \utterance{h:}{Asi demografická analýza}{I guess demographical analysis.}
    \item \utterance{b:}{Co tě na demografické analýze nejvíc baví, nebo naopak stresuje?}{What is the most fun aspect or the most stressful about demographical analysis?}
    \item \utterance{h:}{Nejvíce mě stresuje, že je tam opravdu hodně učiva. Baví mě tabulky v excelu}{What stresses me the most is that there is really a lot of material. I like the excel tables.}
    \item \utterance{b:}{Na co se pak těšíš nejvíc, až zkouškové skončí?}{What do you look forward to the most, when the exam period is over?}
    \item \utterance{h:}{Na ten klid a žádnej stres}{The peace and no stress}
    \item \utterance{b:}{Na co ještě jiného, kromě klidu, se těšíš?}{Is there anything else other than peace you are looking forward to?}
    \item \utterance{h:}{Na to, že budu moct zase trávit víc času se svými přáteli}{To being able to spend more time with my friends again}
    \item \utterance{b:}{Jak dlouho už ji studuješ?}{How long have you been studying it?}
    \item \utterance{h:}{druhým rokem}{second year}
    \item ...
    \end{itemize}
    \end{quote}

    The chatbot question goes back to an entity that has only been mentioned several turns ago
    and the conversation topic has somewhat drifted apart since then.
    The participant would in this case be able to map the anaphora with the help of its lexical environment.
    Especially the verb of the question containing the anaphora makes or breaks the understanding.
    This is clear in following conversation, where the verb is repeated:

    \begin{quote}
    \begin{itemize}[label={}, leftmargin=0pt, itemsep=0.5em]
    \item \textbf{conversation id: 30}
    \item ...
     \item \utterance{h:}{vcera me bolela hlava}{I had a headache yesterday}
     \item \utterance{b:}{Jak dlouho už tě to trápí?}{How long has that bothered you?}
     \item \utterance{h:}{krk jen dnes a hlava dva dny}{throat only today and head two days}
     \item \utterance{b:}{Bolí tě i něco jiného, nebo jen hlava a krk?}{Does anything else hurt other than head and throat?}
     \item \utterance{h:}{nene}{no no}
     \item \utterance{b:}{Co si myslíš, že by ti mohlo pomoct?}{What do you thing could help you?}
     \item \utterance{h:}{snad léky z lékarny}{Perhaps medicine from the pharmacy}
     \item \utterance{b:}{Co jsi dělal, když tě začala ona bolet?}{What did you do when she started hurting? (note: head is feminine in Czech)}
     \item \utterance{h:}{vzala si paralen}{I took a paralen (note: paralen is a pill)}
    \item ...
    \end{itemize}
    \end{quote}

    This mapping is so obvious due to the repetition that
    the inappropriate nonzero anaphora does not distract the participant.

\subsection{Hard shift refusal}

    Various reactions to the deep anaphora stimulus type were recorded.
    The data shows sometimes mapping the anaphora does not warrant acceptance of the response.

    \begin{quote}
    \begin{itemize}[label={}, leftmargin=0pt, itemsep=0.5em]
    \item \textbf{conversation id: 114}
    \item ...
    \item \utterance{b:}{To je těžké říct, ale miluju Gaudího. Co ty?}{That is hard to say, but I love Gaudí. What about you?}
    \item \utterance{h:}{Já také, co se ti na Gaudím líbí?}{Me too, what do you like about Gaudí?}
    \item \utterance{b:}{Líbí se mi jeho kreativita a přírodní inspirace. Co tě na něm oslovuje?}{I love his creativity and inspiration in nature. What talk to you about him?}
    \item ... 2 turns (4 utterances) ...
    \item \utterance{h:}{Vinylových desek má až moc :( nenapadá tě něco jiného?}{He has way too many vinyls :( can't you think of something else?}
    \item \utterance{b:}{Jeho organické tvary a barvy mě fascinují. Jak se ti líbí ta?}{His organic shapes and color fascinate me. How to you like her?}
    \item \utterance{h:}{Cože? O čem to mluvíš?}{What? What are you talking about?}
    \item ...
    \end{itemize}
    \end{quote}

    In this conversation the stimulus suffers from a generation issue and
    causes a metacommunicative reaction
    and later is commented upon:

    \begin{quote}
    \utterance{comment:}{Chatbot nepochopil změnu tématu, pravděpodobně reagoval na romantismus, ale já se s ním chtěla bavit o dárcích. Navíc, nedokázal strukturovat otázku.}
    {Chatbot misunderstood the topic shift, probably reacted to romantism, but I want to talk about gifts. Besides he couldn't structure the question.}
    \end{quote}

    Other than the ideal case where deep anaphora supports the conversation coherence by
    following up on a previously mentioned topic,
    deep anaphora as it was programmed here would sometimes abruptly shift to an abandoned topic.

\subsection{Incomplete control over conversation}

    One other frequent reason conversations that were
    meant to contain a deep anaphora stimulus would fail
    was insufficient topical distance from the anaphorized entity.
    Since convform does not relate tracked entities to each other
    conversations like following could not be considered to contain a deep anaphora
    because the anaphorized entity would last occur too recently:

    \begin{quote}
    \begin{itemize}[label={}, leftmargin=0pt, itemsep=0.5em]
    \item \textbf{conversation id: 192}
    \item ...
    \item \utterance{b:}{Pravěk je zajímavý! Co tě na něm nejvíc fascinovalo?}{Prehistoric era is fascinating! What fascinated you most about it?}
    \item \utterance{h:}{Mamuti}{Mammoths}
    \item \utterance{b:}{Mamuti jsou fascinující! Četl jsi o jejich vyhynutí?}{Mammoths are fascinating! Did you read about their extinction?}
    \item \utterance{h:}{Ještě jsem je viděl}{I saw them still}
    \item \utterance{b:}{Opravdu? Kde jsi je viděl a jaký to byl zážitek?}{Really? Where did you see them and how was that experience?}
    \item ... 4 turns (8 utterances) ...
    \item \utterance{b:}{Mamuti a jejich adaptace na ledové podmínky mě fascinují.}{Mammoths and their adaptation to icy conditions fascinate me.}
    \item \utterance{h:}{V těch knihovnách?}{In those libraries?}
    \item \utterance{b:}{Odkud sis vybral materiály k jejich výzkumu?}{Where did you pick the materials for their research?}
    \item ...
    \end{itemize}
    \end{quote}

\subsection{Nontopicalized entities}

    One of the requirements for an entity
    to be able to be mapped onto by a deep anaphora
    might be that it has been topicalized.
    One such case is found in the data:

    \begin{quote}
    \begin{itemize}[label={}, leftmargin=0pt, itemsep=0.5em]
    \item \textbf{conversation id: 162}
    \item ...
    \item \utterance{b:}{Jaké hry máš nejraději hrát s kamarády?}{Which games do you like to play with your friends the most?}
    \item \utterance{h:}{Asi Sea of Thieves, Minecraft, nebo střílečky}{Sea of Thieves, Minecraft or shooters I guess}
    \item \utterance{b:}{Jakou hudbu nejraději posloucháš při hraní her?}{Which music do you prefer to listen to while playing these games?}
    \item \utterance{h:}{Ne vždycky u toho poslouchám hudbu, ale když ano, tak buď rap, nebo doporučení od dvou kanálů na Youtube}{I don't alway listen to music, but when I do it's either rap or recommendations of two Youtube channels}
    \item \utterance{b:}{Jaké kanály na YouTube ti dělají hudební doporučení?}{Which channels on Youtube give you recommendations?}
    \item \utterance{h:}{Asi je neznáš, ale xkito a MrSuicideSheep}{You probably don't know them but xkito and MrSuicideSheep}
    \item \utterance{b:}{Jaké stavební projekty jsi v něm nejvíc dokončil?}{Which building projects did you finish the most in him (note: Minecraft is masculine in Czech)}
    \item \textit{aborted}
    \end{itemize}
    \end{quote}

    The participant aborts the conversation after a question is asked that contains an anaphora
    which due to its lexical environment can only be mapped to \textit{minecraft} from a previous participant response
    as it is the only entity that has to do with \textit{building projects}.
    This entity however has only been in the rhema position and mentioned long ago from the perpsective of topical progression.
    Other issues contributing to the participant aborting the conversation are lack of the topic shift annotation
    and lack of closure of the interrupted topic.
    There has however also been an annotator disagreement on this conversation
    as one of the annotators considered this conversation to contain a nonassignable anaphora.
    While that was slightly against the annotator manual,
    because formally there is indeed a mapping candidate present in the conversation text,
    this supports the participants decision to stop the conversation.
    In terms of Daneš () the conversation topic progression is as follows:

    \newcommand{\topicannot}[3]{
        \textbf{#1} #2 \\
    }

    \begin{quote}
    \begin{itemize}[label={}, leftmargin=0pt, itemsep=0.5em]
    \item \textbf{topical progression of conversation id: 162}
    \item ...
    \item \topicannot{b:}{Which T1?}
    \item \topicannot{h:}{R1a, R1b , or R1c.}
    \item \topicannot{b:}{Which T2?}
    \item \topicannot{h:}{Either R2a or R2b.}
    \item \topicannot{b:}{Which T3(<-R2b)?}
    \item \topicannot{h:}{T3a and T3b.}
    \item \topicannot{b:}{Which T4(<-R1b)?}
    \item \textit{aborted}
    \end{itemize}
    \end{quote}

    Here T4 is tied to R1b which never stood in the topic position.
    More conversations of this type  need to be gathered
    to understand the role of topicalization in conversation coherence.

\subsection{Unexpected refusal}

Sometimes stimuli that were considered to be mappable and coherent
cause metacommunicative reactions - the participants misunderstood.

In the following conversation the stimulus was mappable:

\begin{quote}
\begin{itemize}[label={}, leftmargin=0pt, itemsep=0.5em]
\item \textbf{conversation id: 96}
\item ...
\item \utterance{b:}{Co by tě na tom předmětu víc zaujalo?}{What would interested you the most in that course?}
\item \utterance{h:}{kdyby ho učil pes}{If it was given by a dog}
\item \utterance{b:}{Jaké bys od psa čekal vyučovací metody?}{Which teaching methods would you expect from a dog?}
\item \utterance{h:}{štěkání v morseovce}{barking morse code}
\item \utterance{b:}{A co bys dál od psa očekával v hodině?}{And what else would you expect from a dog in the lesson?}
\item \utterance{h:}{mazlení. co ty?}{cuddling. what about you?}
\item \utterance{b:}{Jakého bys chtěl mít jako učitele?}{Which would you like to have as a teacher?}
\item \utterance{h:}{jak to myslíš?}{how do you mean?}
\item ...
\end{itemize}
\end{quote}

The question itself was however ambivalent in some way as the participant comment suggests:

\begin{quote}
\utterance{comment:}{nenapsal jaké plemeno}{didn't write which race}
\end{quote}

In following conversation, zero anaphora referring to the topic at hand
caused metacommunication:

\begin{quote}
\begin{itemize}[label={}, leftmargin=0pt, itemsep=0.5em]
\item \textbf{conversation id: 312}
\item ...
\item \utterance{h:}{Mám, jmenuje se: She is the man}{I do, it is called: She is the man}
\item \utterance{b:}{To je skvělý film! Co tě na něm nejvíc baví?}{That is a great movie! What do you like about it the most?}
\item \utterance{h:}{Je to romantická komedie, takže mě baví to, že je to spíše na oddech.}{It is a romcom so I like it more to relax.}
\item \utterance{b:}{Jaké typy máš nejraději?}{Which types do you like best?}
\item \utterance{h:}{Typy čeho?}{Types of what?}
\item \utterance{b:}{Typy filmů nebo seriálů, jaký žánr preferuješ?}{Types of movies or series, which genre do you prefer?}
\item ...
\end{itemize}
\end{quote}

Referring to a the current hypertopic via a zero anaphora has the potential to cause a spectrum of reaction.

\subsection{Unexpected acceptance}

Sometimes participants would accept a hardly or not at all assignable anaphora.

    \begin{quote}
    \begin{itemize}[label={}, leftmargin=0pt, itemsep=0.5em]
    \item \textbf{conversation id: 279}
    \item ...
    \item \utterance{b:}{Kterou koledou začínáte zkoušky?}{Which carol do you start your rehearsals with?}
    \item \utterance{h:}{Každou zkoušku začínáme jinak}{We start each rehersal differently}
    \item \utterance{b:}{Tak to by mě zajímalo, jak na to pak ona zareagovala.}{Then I wonder, how she reaction to that.}
    \item \utterance{h:}{To mě taky}{Me too}
    \item \utterance{b:}{Z jakého důvodu měníte pořadí hned po každém zkouškovém dni?}{For which reason do you change the order after every rehearsal day?}
    \item \textit{aborted}
    \end{itemize}
    \end{quote}

This conversation has a gramatically feminine mapping candidate
and somehow the chatbot remark though hardcoded makes sense to the participant.

In the following conversation the participant seems to want to continue in the conversation
regardless of the incoherent response from the chatbot and so they simply change the topic:

    \begin{quote}
    \begin{itemize}[label={}, leftmargin=0pt, itemsep=0.5em]
    \item \textbf{conversation id: 142}
    \item ...
    \item \utterance{h:}{a co jsi dělal ty? }{and what did you do?}
    \item \utterance{b:}{Dělal jsem si radost četbou a učením nových věcí. Ty máš nějaké oblíbené předměty?}{I was enjoying myself by reading and learning new things. Do you have any favorite subjects?}
    \item \utterance{h:}{biofyzikální chemii}{biophysical chemistry}
    \item \utterance{b:}{To zní zajímavě! Co tě na ní nejvíc baví?}{That sounds interesting! What do you like most about it?}
    \item \utterance{h:}{jako baví mě farmakologie}{like I enjoy pharmacology}
    \item \utterance{b:}{Tak to by mě zajímalo, jak na to pak ona zareagovala.}{Then I wonder, how she reaction to that.}
    \item \utterance{h:}{a ty studuješB }{and you studyB}
    \item ...
    \end{itemize}
    \end{quote}

\subsection{Nonfactual implication}

    The chatbot would sometimes hallucinate () information that has not been said.
    Here the chatbot implies that the participant mentioned going to work that day:

    \begin{quote}
    \begin{itemize}[label={}, leftmargin=0pt, itemsep=0.5em]
    \item \textbf{conversation id: 264p}
    \item \utterance{b:}{Ahoj, já jsem Vladimatik.}{Hi, I am Vladimatik.}
    \item \utterance{h:}{Ahojky }{Hello}
    \item \utterance{b:}{Jaký jsi měl den?}{How was your day?}
    \item \utterance{h:}{Docela to šlo. Co ty?}{It was ok. What about you?}
    \item \utterance{b:}{Taky se mám dobře, díky za optání.}{I am also well, thanks for asking.}
    \item \utterance{h:}{To je dobře }{That's good.}
    \item \utterance{b:}{Taky ráda něco podnikám po práci, co tě dneska bavilo?}{I also like to do things after work, what did you enjoy today?}
    \item \utterance{h:}{Já jsem dneska nebyla v práci a navíc mě dnes nic nebavilo…}{I wasn't at work today and besides I enjoyed nothing today...}
    \item ...
    \end{itemize}
    \end{quote}

    Implying things that were not mentioned leads to a special type of metacommunicative reaction
    which can contain negated repetition of the incorrect statement.

\section{Metacommunication}

    317 metacommunication ¨
    \begin{quote}
    \begin{itemize}[label={}, leftmargin=0pt, itemsep=0.5em]
    \item \textbf{conversation id: 317}
    \item ...
    \item \utterance{b:}{Tak to by mě zajímalo, jak na to pak ona zareagovala.}{Then I wonder, how she reaction to that.}
    \item \utterance{b:}{kdo ona?}{who she?}
    \item ...
    \end{itemize}
    \end{quote}

    270 double anaphora -> meta ¨
    \begin{quote}
    \begin{itemize}[label={}, leftmargin=0pt, itemsep=0.5em]
    \item \textbf{conversation id: 270}
    \item ...
    \item \utterance{b:}{To je skvělý plán, co nejraději v posilovně cvičíš?}{That is a great plan, what do you like to excersise in the gym?}
    \item \utterance{h:}{nejraději cvičím záda. myslím, že budu dělat cviky na záda zrovna dneska}{I like the back the most. I think I will do back excersise today}
    \item \utterance{b:}{Které konkrétně na něj mají vliv?}{Which exactly have influence on him?}
    \item \utterance{h:}{jak to myslíš, napiš mi prosím otázku líp}{what do you mean, write the question better please}
    \item ...
    \end{itemize}
    \end{quote}

    101 meta but ended shortly after, coherence would not recover ¨
    \begin{quote}
    \begin{itemize}[label={}, leftmargin=0pt, itemsep=0.5em]
    \item \textbf{conversation id: 101}
    \item ...
    \item \utterance{b:}{To jsi naprosto pravdu! Jaké knižní překvapení tě naposledy potkalo?}{You are completely right! Which book surprise did you get last?}
    \item \utterance{h:}{Asi brandon sanders, i když sci-fi vlastně vůbec nečtu}{I guess brandon sanders even though I don't actually read sci-fi at all.}
    \item \utterance{b:}{Tak to by mě zajímalo, jak na to pak ona zareagovala.}{Then I wonder, how she reaction to that.}
    \item \utterance{h:}{Kdo?}{Who}
    \item \utterance{b:}{Omlouvám se, myslel jsem na postavy z jeho knih. Co se ti na Sandersonovi líbilo?}{Sorry, I meant the characters from his books. What did you like about Sanderson?}
    \item ...
    \end{itemize}
    \end{quote}

    149 uncertain whether metacommunication ¨
    \begin{quote}
    \begin{itemize}[label={}, leftmargin=0pt, itemsep=0.5em]
    \item \textbf{conversation id: 149}
    \item ...
    \item \utterance{b:}{Co si myslíš o mediaci při rozvodu?}{What do you think about mediation during divorce?}
    \item \utterance{h:}{je to dobrý insitut}{it is a good institution}
    \item \utterance{b:}{To může pomoci vyřešit spory bez soudů. Co si myslíš o rolí mediátora?}{It can help solve conflicts without court. What do you think about the role of the mediator?}
    \item \utterance{h:}{je to velmi důležitá pozice}{it is a very important role}
    \item \utterance{b:}{Jaké další způsoby řešení sporů znáš?}{Which other ways of solving conflicts do you know?}
    \item \utterance{h:}{jakých sporů}{what conflicts}
    \item ...
    \end{itemize}
    \end{quote}

\section{Conclusion}

annotation of categories works
categorization itself purposfuly simplified
edgecases for both anaphora type and metacommunication point us in the right direction
ratings work just a bit - control over conversation
conversation style dont work - control over conversation
stimulus reaction pairs work !!
    result shows there is further split in deep and nonassignable - realtime topic analysis
zero anaphora - good
pro coherence
remaps - realtime topic analysis
pseudoanaphora - complex extratextual mapping
deep anaphora - chatbot needs to know what they are talking about
deep anaphora - refusal though understood
    topicalization plays a role
    annotation plays a role
deep anaphora acceptance depends on
whether the topic shift is annotated, accepted and
whether there is enough information to perform mapping and
whether there is no critical ambivalence
topicalization?


\chapter{Final thoughts}

\section{Evaluation of anaphora categorization}

    The categorization used in this research has been simplified for exploration.
    The immediate observation has been that both deep and nonassignable anaphora
    have two dominant participant reactions.
    The deep anaphora was had a similar number of acceptance and metacommunicative reaction.
    This begs a question whether it is possible to program and prompt two different chatbots,
    that will be capable of sharing these two types of reactions.
    In other words the question is whether it is possible to simulate an immediately acceptable
    and a confusing deep anaphora.
    The same goes for the nonassignable anaphora
    where the split was on the metacommunication and abroting the conversation.
    Here the task to split these two types up among two different chatbots is even trickier,
    because the criteria of not understanding and directly ending the conversation might depend on
    individual circumstances.
    That way the textual aspect of the conversation would have a lesser effect.

    That qualitative analysis shows also that
    at least one disruptive anaphora type has been elided from the categorization.
    In many cases the participant confusion stemmed from the ambivalence of the anaphora -
    there has been multiple mapping candidates.
    There is a hint of a more granular categorization of disruptive anaphora.

    Finally on anaphora and conversation coherence in general,
    it is not easy to just break the anaphora,
    something always comes along with it as the various factors at hand in conversation are intertwined.
    The data certainly showed, that everytime there is a anaphora related coherence disruption in conversation,
    this impacts the topical structure of the unfolding text of conversation
    which is where the incoherence ends up being perceived.

\section{Further steps}


    \subsection{Experiment design improvement}

        For future research better control over the conversation needs to be gained.
        That way the amount of noise can be decreased as there would be less sources
        of confusion or nonacceptability in the conversation text other than the stimulus.
        Participants could also be exposed to a sequence of conversations with different stimulus each time.
        That way the stimuli could be ordered a certain way and a rating development metric would be gained.
        Lastly more granular approach to the participant reaction to stimulus could be taken.
        At least one type has been ommited or rather fused with the metacommunication reaction type -
        the late aborting of the conversation.
        After the stimulus if the participant reacts in a metacommunicative way,
        the chatbot has a chance to fix the disruption.
        If it does not succeed in that, the coherence remains disrupted and
        the social interaction illusion falls due to the stimulus but not directly after the stimulus.
        For this the chatbot also needs to have access to which entity the anaphora has been referring to.

    \subsection{Stimulus generation improvement}

        Some of the challenges met during the course of the research have to do with
        the stimulus generation.
        One improvement that has been mentioned above is the chatbot being aware of what it is referring to
        with its anaphoras.
        In this version of convform, this information has not been stored anywhere.
        It should be present in the prompt in case the participant is confused and asking about it.
        The secondary reaction might show which types of referents will be acceptable or less so.

        The conversation quality needs to improve.
        The zero anaphora has to be deployed where it is appropriate to decrease the nonacceptability noise.
        Better topic progression awareness needs to be developed in convform.
        The chatbot sometimes ignores questions and asks its own which is topically and sequentially inappropriate.

        Lastly the nonassignable anaphora stimulus could be improved to generate stimulus responses via prompt.
        That would require generating an irrelevant entity and anaphorically referring to it.
        This challenge could provide more insight on associativity and pseudoanaphora if tackled succesfully.

    \subsection{Hybrid approach}

            All of the aforementioned improvements are hard to achieve with the current state of convform.
            At the time of this research it functions as a hybrid tree structure where the response is either
            hardcoded or a prompt.
            To gain more flexibility, the dialog management needs to be more prompt oriented.
            However to keep the control, the prompt needs to consist of components that self regulate based on
            logic and analysis prompts.
            Essentially with each turn metadata
            like topical and sequential status of the conversation should be collected.
            The components should then be allowed to assemble the prompt
            based on this data while following the conversation designer instruction.
            This process should allow for creating a dialog system that is aware of what is going on
            and can conditionally produces prompts and generate responses
            that are defined by the conversation designer
            while maintaining the flexibility with which state-of-the-art systems interact.

\section{New opportunities}

With a dialog system of this nature
intentionally creating the stimuli hinted at in the analysis would become possible:

    \begin{itemize}
    \item{Understanding the bimodality of deep and nonassignable anaphora}
    \item{Deep anaphora referring to a (non)topicalized entity}
    \item{Closer look at zero reference, pseudoanaphora and associativity}
    \item{Observing participant reactions to nonfactual implications}
    \item{Comparing different anaphora depths}
    \item{Topic shift with or without discourse markers}
    \item{Secondary participant reactions to repair attempts}
    \end{itemize}

\section{Conclusion}

The methodology used in present research has be experimental on two levels.
One was to see whether it can be useful for conversation research.
While there are many improvements to be made on both the experimental framework
and the stimuli, it has proven to be able to generate data that would be otherwise
impossible to get.
This comes at the price of annotation. Every conversation has to be visited to confirm
whether the required stimulus is actually present.
With improving conversation design, this might become a smaller issue
and with a certain degree of precision and a certain scale,
annotations might become obsolete.
For now and near future, the conversations will be visited.
Seeing the interactions with ones own eyes has been the goal of the research regardless,
so annotating can also be perceived as a reward rather than a price.

The other level was to expose participants to a chatbot that would subvert the course of the conversation.
This has shown to be succesful to a decent degree although here also it a lot of improvement to be made
nn both practical and theoretical aspect.
The result itself has been more or less unsurprising and the experiment served mainly to generate data
that would support this uncontroversial claim.
From here, multiple research paths can be taken in both understanding the inner workings of conversation
and attempting to simulate various situations using the dialog system.


\printbibliography

\end{document}
