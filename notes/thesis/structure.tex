\documentclass[12pt]{report}
\title{Coherence disruptions in human-chatbot interaction: towards quantitative approach to conversation}
\date{}
\author{Albert Maršík}

\bibliographystyle{alpha}
\usepackage[style=verbose]{biblatex}
\usepackage{dirtytalk}
\addbibresource{references.bib}

\begin{document}

   % methodology
   % data
   % experiment

\maketitle

\chapter*{Introduction}
\addcontentsline{toc}{chapter}{Introduction}

\par
Recently, there has been a breakthrough in the way we interact with machines \footcite{sharma2024exploring}.
We can instruct a computer using natural language \footcite{hendrix1982natural}.
Besides making existing technology an extra step accessible,
new ways to use technology appear.
Automated interaction alone can solve previously unsolvable problems,
such as notably accessing a knowledge base via semantic search \footcite{makela2005survey}.
Until recently a knowledge base would usually be accessed only via fulltext,
meaning we would only be able to find
information of which we knew a part of the formal encoding.
Today, we can search for information simply by asking questions,
all thanks to natural language computer interface.
\par
The promise of much practical usage of the current wave of generative AI is ambitious
and only brings its fruit slowly, perhaps slower, than was expected
\footcite{bloomberg2024openai1}\footcite{reuters2024openai}.
There is talk of a "plateau" in development of the technology powering the current cutting edge inventions \footcite{ritter2024ai}.
And that is not the only issue there is to the current wave of cutting edge AI. To name the most prominent ones:

\begin{itemize}

    \item
    high electricity consumption \footcite{ritchie2024ai}

    \item
    unpredictable and broad societal impact \footcite{hagerty2019global}\footcite{baldassarre2023social}
\end{itemize}

That being said, in context of conversation research,
this development in technology promises to makes things possible that previously were not.
With a partial control of what happens in the conversation and a decent certainty, that
our system will simulate human-human conversation to the user,
new kind of conversational data is in reach -
logs of the human-bot interaction, that could be categorized based on

\begin{itemize}

    \item
    which researcher controlled stimulus and

    \item
    which participant reaction to given stimulus
\end{itemize}

they contain.

\par
In the 1960s the relatively recent emergence and adoption of telephone technology
allowed for recording and transcribing authentic conversational data.
This advancement took place thanks to developement in technology
which is reminiscent of the current day situation.
While human-bot conversational data is arguably less authentic than telephone conversation transcripts,
experimental approach can be taken while the human element is present.

\par
This papers intention is to provide a debate on a metaresearch question -
is using generative AI a viable methodology for conversation research?
This is done by attempting to develop that very methodology.
Proceeding we operate in a frontier,
our first steps should be establishing data backed baseline knowledge
and assessing possible lines of research.

\par
To understand what something is,
it can be fruitful first understanding what that thing is not.
One way to understand what makes the unraveling text of a conversation
a valid one - a coherent one -
is obtaining conversational data containing coherence disruptions.
This can be done using the discussed technology -
it has the capability of conversing in a way
that is found generally acceptable by humans
and can drift away from the coherent interaction
if appropriately instruced to do so.

\par
The data this paper seeks to elicitate and analyse are
actual human-chatbot exchanges containing moments which
have the potential to be problematic for
the human participant to process and follow up on.
The line between what a coherent and an incoherent conversation is blurred.
It is in no way a binary property of the text of the conversation.
The goal is therefore to touch on the gradual divide between them.

\par
While chatbots are evaluated for how natural and error free their way of conversing is,
human-human conversation is rarely flawless:

\begin{itemize}
\item
errors happen
\item
conversational coherence gets temporarily disrupted
\end{itemize}

In case of human-human communication, disruptions can however be cured easily.

\par
It used to be (and remains so in legacy systems)
that in human-bot communication, disruptions could derail a conversation completely,
leaving the bot, who would only rely on surface level textual clues, in the dark. \footcite{mctear2020conversational}
This has become rare with generative AI.
Even though it brings a set of its own problems like
frequently lacking factuality or
the difficulty to handle data responsibly,
the cutting edge technology powered conversation systems are

\begin{itemize}

\item
   better capable of understanding and producing relevant answers

\item
   instructed to return to their conversational point of departure
\end{itemize}

\par
Human-bot communication is often single-purpose.
Companies and institutions deploy voice applications to interact with customers and clients,
so there is usually a goal to be achieved.
The coherence of such conversation can then be described based on whether
the goal has been achieved with success and ease.
Another common scenario is an open-domain conversation,
also known as chit-chat or smalltalk.

\par
Some factors that influence coherence in conversational texts,
whether in human-human or human-bot exchanges,
have been extensively studied.
\par
Namely:

%--
\par{\textbf{Politeness}
Brown and Levinson’s work on politeness strategies describes social alignment in smooth interactions \footcite{brown1987politeness}. Politeness strategies, such as using polite language, offering options, or softening potentially face-threatening comments, help to create a comfortable communicative environment. These strategies align with social norms, which people interpret as markers of respect, consideration, or even trust. A failure to employ these politeness strategies, or using them inconsistently, can disrupt conversational coherence. For example, blunt or overly direct responses may be perceived as abrupt or rude, diverting the conversation's flow or causing discomfort. In such cases, the breakdown of polite norms can lead participants to question intent, hindering effective and smooth communication.


\par{\textbf{Speech acts}}
Following Austin and Searle's speech act theories, communication rely on expressing clear intentions and meanings that help build mutual understanding \footcite{austin1962how} \footcite{searle1969speech}. When speakers convey intentions explicitly through statements, questions, requests, or assertions, it signals to listeners the purpose and direction of the conversation. Effective communication strategies help maintain coherence by ensuring each contribution builds logically on the last. On the other hand, unclear intentions or ambiguous phrasing can create misunderstandings, disrupting the conversation's flow. Misalignment or mixed signals – such as using sarcasm without cues or making indirect requests without context— can leave listeners uncertain about how to respond, leading to off-track or irrelevant contributions and possibly creating need to address the communication to regain understanding.

\par{\textbf{Conversational Maxims}}
Grice’s conversational maxims are fundamental to coherent dialogue \footcite{grice1975logic}.
They suggest that participants should:

\begin{itemize}
\item
provide truthful information (Quality)
\item
neither too much nor too little (Quantity)
\item
remain on-topic (Relevance)
\item
communicate in an orderly, clear manner (Manner)
\end{itemize}

These maxims encourage effective exchange by setting a standard for contributions that are informative, truthful, relevant, and unambiguous. When violated, such as by giving excessive detail, omitting important context, or straying from the topic, coherence suffers. For instance, irrelevant tangents or over-detailed explanations may confuse the listener as to what is the main focal point of conversation in that moment. This misalignment can leave participants uncertain about the conversation’s direction, ultimately diminishing coherence and the effectiveness of communication.

\par{\textbf{Sequence Structure}}
The work of Schegloff and Sacks on sequence structure and turn-taking emphasizes that ordered interactions support predictability and continuity in dialogue \footcite{Schegloff_2007}. Turn-taking conventions — where participants follow an implicit sequence of speaking and responding — help maintain the flow by structuring the conversation in a logical order. This sequence structure allows both parties to anticipate when to listen and when to speak, contributing to a well-paced, cohesive exchange. However, interruptions, abrupt changes in topic, or skipping expected responses can disrupt this sequence, introducing unpredictability that can confuse participants. These interruptions fragment coherence by shifting the conversation away from expected responses or structured flow, often leaving gaps in understanding or causing conversational breakdowns.

\par{\textbf{Message and Topic}}
Interactional linguistics underscores that consistency in message and topic preserves continuity in conversation
\footcite{CouperKuhlenSelting2017}. When speakers stick to a shared topic or make gradual, clear shifts, coherence is maintained because participants know what to expect. Frequent or abrupt topic shifts, however, or sending unclear or conflicting messages, can create disjointed exchanges. For instance, introducing a new topic without closure on the previous one can confuse listeners, leading to a scattered or fragmented interaction. By shifting focus unpredictably or offering unclear messages, coherence diminishes as participants lose track of the conversation’s thread, resulting in exchanges that feel scattered or incomplete.

\par
While all of the mentioned areas unveil much about the way conversation works,
rarely do they concern themselves with the textual dimension of conversation.
Most of the mentioned authors (with the notable exception of those operating within the interactional linguistics framework) would hardly be described as linguists,
though their works significantly inform linguistics.

\par
The lack of a true interpersonal dimension in human-chatbot communication
allows to focus solely on the elements in conversational text,
that make it cohesive and coherent or rather
those that have the potential to prevent it from being that.
The key concepts discussed in this paper are
two closely related topics:

Coreference realized by anaphore and topic – what the text is about.




\chapter{Theoretical background}

% Hrbáček, Halliday, Roberts, Givón, Nedoluzhko, Izotopie (Daneš a další)

\section{Textual dimension of conversation}
\par
    The following concepts will be explored individually, in relation to one another and in relation to conversation: text, coherence, cohesion, coreference, anaphora, cataphora, endophora, exophora, topic, entity, and association. While the presented exploration draws on existing literature, it seeks to establish an independent and sustainable framework, rather than strictly adhering to established interpretations.

\subsubsection{Text}
\par
    Text, in its broadest sense, refers to any form of communication that conveys meaning through a combination of signs, symbols, or language \footcite[p.~7]{hrbacek1994}\footcite{hjelmslev2016}.
    These semiotic structures can take various forms, including written, spoken, visual, or even non-verbal modes of expression \footcite[p.~13]{barthes1977image}.
    A text can be as simple as a single sentence or as complex as a novel, and it can exist across different mediums, from books and articles to advertisements and digital content.
    What defines a text is its ability to convey a coherent message or idea, often intended for interpretation by an audience or an adressee.
    Texts can serve a wide range of purposes, including storytelling, instruction, persuasion, or simply recording information.
    Typically text is a structure that is
    linguistic, produced and percieved as intentional and coherent.

\par
    The text of a conversation is specific because it is multiproducer.
    Another example of a multiproducer text
    would be a sequence of commercial signs on a busy street.
    It is the spatial juxtaposition of the signs and temporal juxtaposition of utterances,
    that make them a text.

    \par
    Another property of a conversation text is it is negotiated.
    This is given by its multiproducer and temporal nature.
    Other types of text which are also negotiated are relatively rare.
    There are occurences of debates which take place in written text,
    whether they are press columns or academic articles, which
    interact explicitly with each other, making them a negotiation.
    Such press discourse could however be considered
    a sequence of text units rather than a single temporarily juxtaposed text.
    This perspective could hardly be defended in regards to conversation, because
    its tight temporal coupling and cohesion,
    making conversation a unique phenomena.

\subsubsection{Coherence}
\par
    Coherence refers to the logical connections and consistent relationships that
    make a text easy to follow and possible to understand \footcite[p.~83]{givón2020coherence}\footcite[p.~9]{hrbacek1994}.
    It is achieved when the ideas, sentences, and paragraphs within a text are linked together in a meaningful way,
    allowing the reader to grasp the author's message without confusion.
    Coherence often depends on the use of transitions, the logical flow of arguments, and the proper sequencing of information.
    It ensures that each part of the text contributes to the overall meaning, creating a unified whole \footcite[p.~28]{hrbacek1994}.
    Incoherent text can be difficult or impossible to understand, even if the individual sentences are grammatically correct \footcite[p.~30]{hrbacek1994}.
    It is a property of the whole text, but
    textual elements can be pointed out that contribute to or diminish the given texts coherence.
    Those elements are however not referred to as 'coherence elements'.

\par
    Coherence is a cognitive phenomenon \footcite{Roberts01101993} because
    it involves the mental processes of interpreting, organizing, and understanding information.
    When reading a text,
    coherence arises not only from the structure and linguistic cues provided by the author but
    also from the reader's ability to
    make connections between ideas based on prior knowledge, expectations, and context.
    This cognitive interaction between the text and the reader’s mind is what makes the content understandable.

\par
    In conversation, coherence becomes even more complex,
    as multiple participants are simultaneously contributing to and interpreting the flow of information.
    Each individual brings their own perspective and understanding to the interaction,
    which requires constant negotiation to maintain coherence.
    Misunderstandings, different backgrounds, and interruptions
    can disrupt the coherence of a conversation,
    making it a more dynamic and fragile process compared to written text.

\begin{itemize}
\item
whether a written text is coherent depends mostly on the reader
\item
whether a conversation text is coherent depends on an ongoing\\ negotiation
\end{itemize}

\par
    Coherence is a scalar property rather than a binary one.
    It is however tricky to measure.
    This paper seeks to explore one possible approach of
    declaring different levels of coherence disruptions
    and observing the acceptance rates in participants
    and corelation between them.

%---
\subsubsection{Cohesion}
\par
    While coherence refers to the interpretative quality of a text,
    wherein the ideas form a logical and meaningful whole
    cohesion,focuses on the structural relations
    within a text, achieved through grammatical and lexical links.
    It should be seen as an umbrella term
    covering specific relations within
    the structure of the text,
    where cohesive elements can be directly pointed out.
    While coherent text does not necesarily need to be cohesive,
    cohesive elements often support it.
    A coherent text tends to be at least somewhat cohesive.

\par
    Halliday and Hasan \footcite{Halliday76cohesion} developed
    a detailed framework of cohesion, which includes
    endophoric references,
    relating parts of the text to each other, and
    exophoric references, which point outside the text \footcite[p.~31]{Halliday76cohesion}.
    Endophoric cohesion covers aspects
    like anaphoric references and cataphoric references \footcite{hajicovasgall2003} \footcite{loaiciga-etal-2022-anaphoric}.
    Exophoric references, however, rely on shared context beyond the text itself,
    requiring readers to use prior knowledge.
    Their framework highlights how elements of cohesion
    contribute to textual unity and flow, even
    if coherence based on meaning is not fully achieved.
    Following concepts can be considered cohesive elements.

\subsubsection{Cataphore and Exophore}
    %---
\par
    In Halliday and Hasan's framework,
    cohesion in language is achieved through various devices that connect
    different parts of a text, forming a unified whole.
    They classify cohesive ties as
    references, substitutive forms, ellipsis, and connectors, with
    anaphoric references being one of the primary ways texts achieve cohesion \footcite[p.~68]{Halliday76cohesion}.
    When a text element cannot be mapped to a preceding referent,
    Halliday and Hasan suggest that cohesion is maintained through
    shared situational understanding, making the reference exophoric.
    Cataphoric references, though less common, involve
    elements that look forward in the text,
    showing intentionality by the author but contributing to
    cohesion primarily through the eventual resolution of the forward-pointing referent.

\par
    In conversation if a seemingly anaphoric text element is not succesfully mapped
    to a preceding textual coreferent
    the reference can still be understood, because shared context.
    Such element reaches out of the text with its reference, making it an exophoric one
    Cataphore is a related phenomena –
    a reference which points forward in the text.
    Such occurence is relatively rare in written text and even more so in conversation.
    In fact it is somehow pointless to account for cataphore in a multi-producer text.
    A cataphore denotes an authors intention to reveal
    the nature of a referent explicitly after first mentioning them.
    In conversation, where multiple contributors cocreate given text,
    and mutual understanding and agreement is the measure of
    how coherent the produced text is, later realisation of a vague reference
    does not contribute to how coherent it is.
    Regardless, in case of a cataphore,
    only the referent is a cohesive element,
    not the cataphore,
    as it ties back to the previous text, creating bonds across large textual units.

\subsubsection{Anaphore, Endophore and Coreference}
\par
    A common cohesive text element is an anaphore \footcite{Nedoluzhko2010}.
    It is a reference inside the text pointing back to a previously mentioned entity.
    Often it is realised via personal pronouns.
    Though there are other ways for anaphore to realise.
    %---
    In Czech, anaphoric references often rely on grammatical gender and number,
    making participial endings essential for identifying the referent.
    For instance, when a gramatically masculine entity is mentioned,
    later references might use a participle in the masculine form, such as šel ("he went"),
    connecting back to it without repeating the noun or using a demonstrative.
    Demonstratives, such as ten ("that") or tento ("this one"),
    also frequently serve anaphoric functions, guiding the reader to a previously mentioned subject.
    Temporal and locative adverbs, such as tam ("there") and tehdy ("then"),
    also contribute cohesion by indirectly referencing time and place details introduced earlier in the text.
    These anaphoric elements strengthen textual coherence by reducing redundancy and maintaining flow.
    The reader identifies coreferential links through these markers,
    following the cohesive threads without needing explicit repetitions.

\par
    An anaphoric element is by definition also endophoric.
    It points inside the text it appears in.
    By definition an anaphoric element has a referent, which occures earlier in the text.
    These two elements are then coreferent.
    As such they also share an identical exophoric reference – they point outside of the text.

\par
    In conversation, many aspects of which are subject to negotiation,
    also specific coreference relations can be questioned \footcite{loaiciga2021reference}.
    The reference realised by one communication participant may be unclear to the other
    resulting in a repair request coming from another particiapant.
    In conversation analysis,
    %---
    Sacks’s concept of repair traditionally addresses
    misunderstandings related to intentions and actions,
    loosely drawing on frameworks like Austin’s and Searle’s speech act theories.
    From this perspective, repairs often target interpretative gaps about
    what a speaker intends to do with their utterance.
    However, viewed from a broader, more abstract level,
    what is called repair triggers can extend beyond intentions alone,
    encompassing issues on the textual level as well.
    For instance, an nonassignable anaphora —
    a reference that lacks a clear antecedent —
    may lead to a repair request,
    thereby showing how textual ambiguities prompt interactional responses.
    This approach expands the causes of repair in conversation,
    integrating elements of reference and interactional misalignment,
    where a structural aspect of the language itself can
    become a repairable issue in the communicative exchange.

\subsubsection{Topic}
\par
    Topic is what a text is about.
    That makes topic very complicated to define.
    Among others, some issues with topic and annotating it in text are:

\begin{itemize}
\item
A text can and typically does cover multiple topics
\item
Different framing will produce different topic annotations of text
\item
The span of a topic section can be impossible to delimit within text.
\item
Topic annotation is by its nature always more text,
            so even it can be annotated for topic.
            making topic annotations recursive.
            One cannot therefore achieve a definite topic description of a text.
\end{itemize}

\par
    Despite all these complications,
    topic cannot be ignored in conversation research
    as it is deeply intertwined with the aforementioned concepts.
    Topic progressions across text are realised via anaphore and association
    and tightly interact with coherence.
    An appropriate amount of time has to be spent on a given topic unit,
    enough information has to be said about a given topic
    in order for it to be possible to move on or add another one in the conversation.
    Closure has to be provided in order for a topic to be done.
    Transitioning from one topic to another has a potential to disrupt coherence,
    if the association between the topics is too distant.
    A divergence in topic has to be justified.

\subsubsection{Association}
\par
    Association is a textual realization of an isotopic relation \footcite{koblizek2015}.
    By their exophoric properties, referents exist in a semantic web of relationships.
    Similarly to coherence, associative relationships are a cognitive phenomena.
    They come to exist when they are percieved.
    While association is a cohesive element it is difficult to formalize
    the way it can and has been done with anaphoric text relations.
    It is however a major factor in a coherence of text as
    in some cases a text can only rely on association in its coherence.

\subsubsection{Entity}
\par
    An entity is an exophoric referent, descriptions of objects, people, events etc \footcite{entities}.
    Words or text elements which can be reffered to by an anaphore will be called entities.
    Since a phrase containing an anaphore typically adds more information about the referent
    the new information must be semantically compatible,
    in other words association has to be possible between the referent and the added information.
    Entity also has to do with topic.
    In text topic can be represented by a single or multiple entities.
    Coreferent words will be regarded as a single entity.
    It can serve to partially map a topic distance in the texts chronology.

\section{Interactional dimension of conversation}
\par
In terms of introduced background,
conversation is a text which is produced by multiple producers
This complicates things:

\begin{itemize}

    \item
    Conversation is an interactive process, distinct from static text, which is created collaboratively.

    \item
    Conversational content is continuously negotiated by participants, who continuously adapt one another.

\end{itemize}

Due to its temporal and cooperative nature, conversation allows for:
\begin{itemize}

    \item
    Overlaps in speech,

    \item
    Swift corrections of minor errors,

    \item
    Multiple layers of perspective, including:
        \begin{itemize}

            \item
            Each participant’s personal viewpoint,

            \item
            Each participant’s perception of others’ viewpoints,

            \item
            Each participant’s understanding of the shared conversation as it’s being co-created.
        \end{itemize}
\end{itemize}

        each of these perspectives can desynchronize resulting in misunderstandings.
        Humans however are excellent at correcting misunderstandings
        this is because under regular circumstances, people cooperate.

\par
The Cooperation Principle, introduced by philosopher H.P. Grice,
suggests that participants in a conversation typically work together to achieve effective communication.
Grice proposed that, to ensure this cooperation, speakers follow four conversational maxims.
In practice, people may not always follow these maxims
but they do so in ways that still rely on shared expectations of cooperation.
Even when misunderstandings arise,
humans naturally engage in conversational repair,
using their social intuition and mutual cooperativity to clarify intetion and realign perspectives.

\par
Contemporary conversation research can be understood to draw from conversation analysis.
%---
Modern conversation research traces its roots to conversation analysis,
a field pioneered by sociologists Harvey Sacks and Emanuel Schegloff in the 1960s.
They sought to understand the structure and
social rules of everyday interactions,
focusing on the patterns and norms that govern turn-taking and response.
Thanks to recordings of phonecalls,
transcripts could be qualitatively analyzed.
This research has lead to coining new terminology.

    \subsubsection{Adjacency pair}
    \par
    Adjacency pairs describe sequences of two related utterances by different speakers \footcite[p.~188]{Sacks1992}. These pairs are characterized by their predictable and reciprocal nature, where the first part sets up the expectation for a specific type of response. Common examples include greetings('Hi' → 'Hello'), questions and answers ('What time is it?' → '3 PM'), or offers and acceptances/declines ('Would you like some coffee?' → 'Yes, please' or 'No, thank you').

    \subsubsection{Sequence structure}
    \par
    Sequence structure refers to the organization of conversational turns into coherent patterns or sequences. It describes how interactions are shaped by predictable structures, such as adjacency pairs. These sequences provide order and meaning to conversations, guiding participants in understanding when and how to respond. Schegloff \footcite{Schegloff1990} emphasized that sequence structure is central to the social organization of talk, as it allows participants to manage and negotiate interaction effectively.

    \subsubsection{Topic shading}
    \par
    Topic shading, as discussed by Sacks \footcite{topicshading}, refers to the subtle way in which a conversation naturally shifts from one topic to another while maintaining coherence. Instead of abruptly changing the subject, speakers introduce a related idea or concept, gradually steering the discussion in a new direction. This process allows for smooth transitions in dialogue, helping participants maintain engagement and avoid confusion.

    \subsubsection{Dis/preferred answers}
    \par
    Preferred answers, according to Sacks \footcite[p.~410]{Sacks1992}, are responses in conversations that align with social norms and expectations, making interactions smoother and more cooperative. In conversation analysis, preferred answers typically follow the format or intent of the preceding question or statement . They contrast with "dispreferred" answers, which might include refusals or disagreements and often require additional explanation or mitigation to maintain social harmony.

    \subsubsection{Conversational repair}
    \par
    Conversational repairs refer to how participants address and resolve problems in understanding, hearing, or speaking during interactions \footcite{sacksRepair}. These issues, can occur at any point in a conversation. Repairs are classified into self-repair, where the speaker corrects their own error, and other-repair, where a different participant addresses the issue. They can further be classified into self-initiated repair and other-initiated repair.

\par
%---
As a descendant of conversation analysis interaction linguistics has emerged,
building on its insights to examine language use in social contexts.
It broadens the focus to study not only verbal exchanges but also
multimodal cues like gestures, gaze, and intonation,
analyzing how these elements contribute to meaning.
Interaction linguistics aims to understand the dynamic aspects of conversations,
such as how topics shift and how sequences of speech acts unfold,
reflecting the fluid nature of human communication.

\section{Disruptions in conversation coherence}
\par
While the question of what makes for a coherent text is too broad,
the answer to what makes for a coherent conversation can be somewhat easier to answer.
Because conversation participants negotiate understanding,
it is up to them, when a conversation is and is not coherent
to describe what a coherent conversation is,
it is worth pursuing the moments, when the conversation stops flowing with ease.
Such moments can be called coherence disruptions.
A coherence disruption is a complex phenomena as

\begin{itemize}
\item
it penetrates through some or all of mentioned perspectives on an on-going conversation
\item
it can't be evaluated in a binary fashion
\end{itemize}

There are different degrees to which a conversation coherence can be disrupted:
\begin{itemize}

\item
   if a participant suddenly starts speaking
    in an a way that can hardly be considered interaction
    due to its irrelevance or

\item
   if the utterance simply is not grammatical or understandable,
    while the conversation has been compromised and becomes incoherent,
    it has more to do with incoherent written text, because
    the incoherence is encapsulated on the level of a single utterance
\end{itemize}

Roberts \footcite{Roberts01101993} Discusses various types of incoherent text.
He exemplifies so called giberish as incoherent text that is absent of structural relations.
On the other hand he discusses experimental theater or literature as a type of text
which is assumed to be coherent in the sense that there is an intention behind it
but contains little to no structural relations.
Lastly he mentions a so called "schizophrenic discourse" as a speech that
is not assumed to be coherent even if it has structural relations to it.
In any case Roberts definitively states that coherence is assumed
and is therefore a receptive phenomena.
The incoherence Roberts discusses is considerably different from when
the source of incoherence stems from
    the structure of the conversational text or
    relationship between different utterances
    - this is when another participant assesses,
        they are simply speaking leading a different conversation
        perhaps with a differing intention
        or that they are conversing under differing set of circumstances
        which manifests formally in the linguistic fabric of the conversation - its text.
        All that despite everyone included being cooperative.

\subsection{Sources of incoherence in conversation}
\par
%---
Schegloff shows how incoherence arises when
people interpret sequence structure differently,
namely in terms of which turn is seen as an answer to which previously occuring turn.
In his example, the participants misread each other’s intentions,
leading to confusion about how their turns fit together.
They each project different expectations for how the conversation should unfold,
which causes misaligned sequence structure interpretations.
When this happens,
they turn to brief metacommunication — comments about the conversation itself
to try to clarify and re-align their understanding.
Schegloff illustrates how these
efforts to "repair" the misalignment are central to
managing and resolving incoherent moments in conversation.

\par
Coherence disruptions are also discussed in linguistic literature.
%---
Hrbáček’s approach to coherence and cohesion in text distinguishes the two concepts,
noting how they often interact but can also be independent.
He highlights that while
cohesion involves grammatical or lexical links that
make sentences flow together,
coherence relies on the logical and meaningful progression of ideas.
This means that a text could be cohesive -
using connectives, repetitions, and consistent lexical choices -
yet lack coherence if the sequence of ideas doesn’t
make logical sense or follow a clear progression.
Conversely, a text may be coherent in its narrative flow without
relying heavily on cohesive devices.
In Czech linguistics,
the distinction between téma (theme) and réma (rheme), as used by Daneš,
underlines the role of topic progression.
Hrbáček illustrates this by discussing examples where
a story progresses logically from one point to the next while
being incoherent despite being clear about its topic structure
due to never coming back to a previously mentioned topic.

\par
Two kinds of phenomena are at hand when it comes to
ways in which conversation coherence can be disrupted -
topic shifts and nonassignable anaphore.
While not unique to conversation
both take on specific forms in it worth looking at.

\subsubsection{Topic shifts}
\par
    When conversations shift abruptly from one topic to another,
    it can create confusion for the conversation partner.
    They might find themselves trying to
    reconnect to the previous discussion or
    wondering how the new subject relates.
    This can lead to misunderstandings
    as the transition can feel jarring.

\par
    One interesting question is,
    how do we determine when a topic has run its course?
    What common traits do conversations share when a subject is truly exhausted?
    Perhaps observing transcripts could reveal repeating patterns in topic progression or sequence structure.

\par
    Moreover, what makes for a smooth transition between topics?
    Is it related to the cues participants give each other,
    or perhaps the context of the discussion?
    How do we navigate the flow of conversation and
    what indicates a natural shift versus a disruptive one?

\subsubsection{Nonassignable anaphore}
\par
    nonassignable anaphore is closely tied to topic progression.
    Currently established topic or topics help assigning anaphore and
    determining between an anaphore and an exophore.
    Even if an anaphoric device is not assignable,
    and the reference is presumably an exophoric one,
    The reason for employing this reference must be
    relevant to an established topic.
    In conversation meaning of demonstratives is to be negotiated.
    If an anaphores assignability causes confusion,
    chances are it is caused by one of the following

\begin{itemize}
\item
there are no relevant assignment candidates

    \quad
    this situation can be understood as a vague or unjustified exophore

\item
there are multiple equally relevant candidates

\item
candidate has occured in the conversation text too long ago

    \quad
    can be understood as an abrupt return to previously established topic
\end{itemize}

%---
\subsection{What do people do about coherence disruptions?}
\par
In conversation, coherence disruptions often prompt participants to
employ strategies to maintain understanding and flow.
Schegloff suggests that people manage these disruptions through interactive repair or inference.
Interactive repair often involves
explicitly addressing misunderstandings or clarifying intentions,
often by rephrasing or asking questions.
Interactive repair refers to immediate, collaborative corrections within dialogue,
where one speaker might correct the other or themselves to enhance clarity.
Inference and pragmatic reasoning, the most seamless methods,
allow participants to fill gaps based on context and social cues,
helping conversations continue smoothly without explicit repair.

\par
Dingemanse and Enfield \footcite{DINGEMANSE202430} echoes this from a cognitive perspective,
highlighting how inference and pragmatic reasoning are particularly effective.
Participants rely on shared understanding and contextual knowledge to interpret ambiguous statements.
Together, they use both
explicit (metacommunication and repair) and
implicit (inference and reasoning)
methods work to restore coherence.

\par
It needs to be noted however that both interactive repair and reasoning are
deployed in a number of other contexts
other than conversation coherence disruption.
Inference takes place constantly \footcite{garfinkelstudies}.
Each of those moments could be hardly considered a coherence disruption.
There is however always potential for it,
particularily via unclear or nonassignable anaphore or abrupt unjustified topic shifts.
Repair and metacommunication also takes place in a mutually informed and synchonized interaction.
It is for example deployed when it is revealed
that the interaction participants intentions or opinions differ.

\par
These uses of interaction management are however
hardly possible to analyse on a textual level
since they do not cooccur with coherence disruptions.
What can be observed are – as mentioned above –
troublesome anaforic references and topic progressions.

\chapter{Methodology}

\section{What are chatbots?}
\par
A chatbot is a dialog system powered application simulating conversation with a user.
An attempt to make a machine conversate with a human user requires capturing the essence of human speech.
What the essence speech is and what capturing it means are not defined and
some definitions of what essence of speech is
require specific definitions of what capturing it means.
(Semantics and DialogueSchlangen, David2015) ..
In case of the history of chatbots,
the essence of speech is partially achieved
via mimicking it.
The intention was to have a user interact with a chatbot
that would communicate so well that
the user would be convinced
this is a another human they are talking to.
Whether just that has been achieved would be measured by a so called turing test
proposed by Alan Turing in 1950 \footcite{turing1950computing}

\par
Initial attempts at making a computer converse were rule-based \footcite[p.~43]{Sacks1992}
What that means is the content of the chatbot utterances
would be predetermined
and there would be a decision tree that would decide what to say next.
In the early days as well as often times in modern day systems
string matching would be used to analyse user input.

\par
ELIZA \footcite{weizenbaum1966eliza} is regarded as a milestone
what it was, it pretended to be a therapist
hiding behind general phrases
doctor authority

as long as interaction frame is strictly defined and
the robot has some level of authority in it
the rule-based approach can work
granted, it requires a lot of manual design
and constant maintenance
but it is used nowadays

\par
Machine learning moved things forward in many ways
Fuzzy matchin allowed for close matches
without the neccesity to predefine the exact sequence of characters.
IBM Watson uses classifiers build on examples \footcite{building_watson_2010}
structure remains rule based
reasoning about these rules is decided to a large degree by classifiers

\par
The recent breakthrough pushed another thing in the mainstream
it is now possible to generate near natural speech
this gives the possibility to just let the conversation be taken over by one answer generator
this way we lose tight control over what it does though
For some use cases
like open domain conversation or accessing knowledge base
that is not an issue.

\par
And so currently chabot interface text generators are very prevalent.
In 2024 this technology is now closer to beating the Turing test than
any other model or approach before it\footcite{jones2024peopledistinguishgpt4human}
by having 54\% of participants thinking
they are talking to a human.
While Eliza convinced 22\% of participants,
actual humans only convince 67\% of participants.

\subsubsection{Turn taking in chatbot interactions}

\par
Even if the Turing test is passed,
really fitting simulation of conversation
can only be achieved if the low-level conversation mechanisms
are simulated, like turn taking \footcite{optimizing-turn-taking}.

\par
As established in previous chapter, turn taking is a crucial aspect of conversation.
The way participants distribute who is to talk
explains exhaustively
the difference between
a structure of the text of conversation
and a single-producer text.
The mechanism of turn taking differs
between actual human conversation
and an interaction between a chatbot and a user.

Interaction between chatbot and user
typically take place in a strict fashion
where both participants,
human and virtual,
have unlimited time to come up with the next answer.
While the chatbot should be optimized to answer as fast as possible,
the user has as much time as they need until fallback.

\par
\say{Research in sociolinguistics, psycholinguistics, and
conversational analysis has revealed that
turn-taking is a mixed-initiative,
locally coordinated process, in which
a variety of verbal and nonverbal cues such as
eye gaze,
body pose,
head movements,
hand gestures,
intonation,
hesitations, and
filled pauses
play a very important role.
We continuously produce and monitor each other for
these signals and can coordinate seamlessly
at the scale of hundreds of milliseconds
across these different channels
with multiple actors.} \footcite{turntaking}

\par
People are capable of producing and picking up clues
that indicate opportunities for turn taking flawlessly.
There is a some way ahead for robots in this regard
whether it is figuring out the correct time to start speech \footcite{turntakingreview} \footcite{GRAVANO2011601}
or actually creating a system that will be able to produce such behavior \footcite{distributedturntaking} \footcite{Gervits2020Sigdial}.
This research field
has the potential to push conversation technology
closer to true conversation simulation.

\section{Convform}
\par
An exploration has been carried out using a custom tool called Convform \footcite{convform}.

At its core Convform is a computer program
which accepts a configuration, user input and context
and determines next chatbot answer.
Other than that it offers a collection of utilities
to help design and run chatbots.

\subsubsection{Participant facing chat interface}
In order to handle the inputs, convform provides a chatting environment
for the participants to interact with a chatbot.
The convform environment differs from a usual chat log
because it does not display
the entire history the conversation.
In attempt to simulate spoken conversation
it only displays the last chatbot response.
This way the participant has to rely on their memory
in taking part in the conversation.
Other than that the participant may enter their next response
and send it.
They are also instructed to end to conversation by a red button
if the chatbot behaviour is "unnatural" (nepřirozené)
After the conversation is over whether it has been ended by the user or the chatbot,
there is a questionare which
asks the participants to rate how "natural" the conversation was
and mark and comment on utterances in the now fully displayed conversation.

\subsubsection{Conversation design tool}
Lets admin user create chatbots and define their behavior
the behavior can be defined by string matching rules or prompts
it is capable of working as a statemachine or a single state
it provides a level of control over references within the design

\subsubsection{Testing and debugging of various conversation contexts}
while designing chatbots it is necessary
to be able to simulate various situations
to fine tune various possible scenarios
that might occur in the conversation.
To achieve this, there must be a way
to encode required context to convform.
The convform chatbots use a conversation status (CStatus) object
to represent their current understanding of the conversation.
It contains information about the history of the conversation
which in conjunction with the configuration file and user input
helps determine the next response.
The configuration file is static
CStatus changes automatically
User input comes from the user.
This conversation status can simulate any possible conversation context
from the chatbots perspective
For testing and debugging specific contexts, convform allows admin user
to tweak the conversation status

\subsubsection{Accesing the conversation data}
Lastly convform naturally includes a convenient way to read user interactions
and browse associated conversation status objects

\section{Conversation design in theory}

\par
Designing the behavior of a dialog system
is referred to as conversation design\footcite{kolosova2022} \footcite{mctear2020conversational}.
It is not the course of any one conversation that is being designed here
but rather as many possible ways any conversation could go
for a given use case.
Conversation design as a profession is deeply connected
with the rule-based approach that has been used in ELIZA.
Maintaining all the possible utterances and
rules under which they would be uttered
in commercial dialog systems
has proven to be a responsibility large enough
to generate jobs.

A conversation designer operates between
the business logic and use case of the dialog system
the clients, customers or users interacting with the system
and the developers maintaining the system.

\subsection{Rule-based approach}

In order to be able to design a rule-based dialog system,
one needs to be able to encode the following:

    \begin{itemize}

        \item
        The possible utterances, that the dialog system can produce - "the mouth"

        \item
        Rules under which the next utterance will be chosen - "the ear"

    \end{itemize}

\par
If the conversation is supposed to be a state machnie e.g.
it needs be able to use different sets of rules
under different contexts in the conversation.
This way a dialog system can be context aware to a degree.
A conversation design of this sort
can be displayed as a diagram.
Then a way to maintain context of conversation is also necessary.
This context needs to encode rules to choose an immediate ruleset
which helps determine the next utterance.
This principle is a simplification of
how people decide what they will say next in conversation.

\subsubsection{Pros and cons}

\par
This approach to designing a dialog system
has been the standard for decades.
It offers a granular control over how a conversation should go.
In case of the state machine variant it
allows to guide the user through a relatively complex process.
It however suffers from how unpredictable the user can be.
It is up to the conversation designer to cover all the possible ways of answering
which not only is hardly possible
but also poses a neccesity to parcel the spectrum of possible answers
which can generate conflict when
a user input semantically spans across multiple determined categories.
This issue is even stronger while using the string matching approach,
because there the string literal can decide about the following dialog system answer
as if meanings and their speech representations were a one-to-one map,
which they are not.
Even if a certain meaning is included in a ruleset,
the system might not grasp the meaning and react in an incoherent way.
With the state machine the
distribution of various rules across various rulesets
requires big effort.
Extending the capabilites of a rule-based dialog system
hardly scale and tend to have regressions.
In case of dialog systems relying on user input by speech transcription
the text input processed by the system is not guaranteed to represent
what the user actually said.
In conclusion rule-based approach to conversation design
provides control over the dialog system behavior
but tends to be inflexible.

\subsection{Statistically driven approach}

Some of the issues tied to rule-based systems
are resolved using another approach.
Especially in recent years the breakthroughs in the field
of speech generation have been significant ()
allowing for letting the dialog system play a bigger role
in what is being said next.
In its simplest form,
it is possible to just let the the answer be generated "end-to-end".
The user input is sent to a model which generates an answer.
It has been convincingly shown that this technology has the capability
of reacting in a flexible way to much of what is being thrown at it ().
While not perfect ()
this technology is capable of staying on topic (), mirroring ()
and other things that make for a coherent conversation.

\subsubsection{Large language models}

The main component that is responsible for
this way of simulating conversation at this level of flexibility
are so called large language models ().
These models, powered by advanced neural networks,
have revolutionized the field of natural language processing.
Among the most influential architectures are transformers,
which enable these models to handle
vast amounts of text data and capture complex patterns of meaning, context, and grammar.
The way they generate answers is they take use their training data
to generate the next most probable token.

The Generative Pre-trained Transformer family of models, including ChatGPT,
exemplifies the capabilities of LLMs.
These systems are trained on diverse datasets containing billions of words,
allowing them to generate coherent and contextually relevant responses across various topics.
This flexibility has made them increasingly mainstream,
being integrated into tools for writing, education, customer service, and more.

Unlike traditional rule-based systems,
LLMs rely on deep learning techniques to process and predict language,
enabling them to understand nuanced queries and provide human-like responses.
This adaptability has set new standards for conversational AI,
making it a valuable resource in numerous industries.

\subsubsection{Prompt engineering techniques}

The rapid advancement of LLM technology
has outpaced research into optimal interaction strategies.
Understanding how to engage effectively
with these systems has been a challenge,
reflecting both their power and their novel nature.
Despite this, a foundational idea persists:
an LLM might perform nearly any task if prompted correctly.
While this perspective becomes hard to keep if one expects it to be fulfilled exhaustively
it remains the most productive approach for leveraging LLMs' capabilities.

Over time, researchers and practitioners have developed techniques for
crafting effective prompts to optimize outputs.
The simplest approach is known as "zero-shot" prompting,
where a user poses a direct question or request without additional context.
However, zero-shot prompting may not always yield the desired depth or accuracy.

More sophisticated strategies include "few-shot" prompting,
where examples are provided to guide the model's response style or focus.
..
Chain-of-thought (CoT) prompting encourages the model to
articulate its reasoning step-by-step,
enhancing logical accuracy.
Retrieval-augmented generation (RAG)
incorporates external knowledge sources into responses,
improving factual reliability.
Finally, approaches like ReAct (Reasoning + Acting)
combine reasoning with actions to handle complex, multi-step tasks dynamically.

\subsubsection{Pros and cons}

pros -
flexibility,

creativity,

controllability

cons -
hallucinations,

jail breaks,

questionable reliability,

alignment,

how much of a responsibility do we want to allow LLMS to have?


\subsection{Hybrid approach}



\section{Conversation design in practice}

\par
conversation design in Convform ..

\subsubsection{State}

\subsubsection{Intent}

\subsubsection{Conversation style}

\par
default mode

\par
Inquisitive

\par
Relaxed

\subsubsection{Prompting}

\par
Entity recognition
(New or Old? Exploring How Pre-Trained Language Models Represent Discourse Entities)

\par
Anaphorization

(Annotating anaphoric phenomena in situated dialogue) but here we try to generate it instead


\subsection{Stimuli}

\subsubsection{shallow anaphore}

\subsubsection{deep anaphore}

\par
how its done ..

\par
topic mentioned in the meantime ..

\subsubsection{nonassignable anaphore}

\par
hardcoded ..

\par
prompting ..

\subsection{Ending the conversation}

\subsection{Data}

\par
the nature of elicited data ..
therefore to process the data ..

\subsubsection{Attempting to replicate a situation using conversation design}

\subsubsection{Quantitative analysis}

\subsubsection{Qualitative analysis}





\if false

METHODOLOGY 20
	3 ai approach
		1 llms
		1 prompt engineering techniques
		1 pros and cons
	1 hybrid approach
	conversation design hands on
	1	in general
		stimuli
	1		basic mode
				how it is done
	1			conversational style
	                (Robo-Identity: Exploring Artificial Identity and Emotion via Speech Interactions) \
					inquisitive
					relaxed
	3		prompt techniques
				anaphorize
				entity recognition
	1		shallow anaphore
	1		deep anaphore
				how its done
				asking info mentioned in the meantime
	1		nonassignable anaphore
				hardcoded
				prompting
	1	ending the conversation
	1 what the data gives
	   how to handle data
	2 learning from
	    hands on design - trying to replicate a situation
		qualitative analysis
		quantitative analysis

PILOT 10
    expectations
		rating
			from best to worse: shallow, deep, nonassignable
		abort
			from least to most : shallow, deep, nonassignable
		reactions
			shallow 	- continue
			deep		- meta
			nonassignable 	- abort
		inquisitive will be more negative
	results
		rating
			o from best to worse: shallow, deep, nonassignable
			x inquisitive scores higher
		abort
			x deep and nonassignable same amount of aborts
			o inquisitive significantly more aborts for nonassignable
			o comments confirm this
			x inquisitive slightly less aborts for deep
		reactions
			o tendency manifested
			x deep and nonassignable same amount of meta
			o deep and nonassignable mirroring continue and abort
	qualitative
		hardcoded phrase with intended nonassignable anaphore is sometimes assignable by association
		limiting context for deep anaphore tends to generate inappropriate pragmatics
		limiting context for deep anaphore can lead to anaphore being open to a remap
		inquisitivity sometimes appreciated, other times considered lack of contribution
		deep anaphore (if accepted) tends to be seen as an abrupt shift to previous topic
		nonassignable anaphore (if accepted) tends to be seen as a cataphore
		        cataphore is unimportant in conversation, it is a text phenomena
			    what is important is there is trust (or tolerance)
	further steps
		making the questionnaire more detailed
			convlog
		replicate current results
		annotator agreement on stimulus annotations
		why inquisitive conversation style seems to be more likely to be appreciated?
		what signifies appreciation / dissatisfaction in inquisitive conversation style? (router)
		at which point does deep anaphore become nonassignable?
		when does associativity help assign an otherwise nonassignable anaphore?

\fi


\printbibliography

\end{document}
