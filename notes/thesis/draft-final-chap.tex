
\section{Data collection}

\subsection{Experimental designs}

Data collection took place with the use of six chatbots.
They each represent a possible combination of a 3x2 design.
The two design variables are anaphora type and conversation style.

The anaphora types are as discussed above a shallow, deep and unassignable.
The conversation styles were two. One, inquisitive,
where the instruction part of the prompt
would always instruct the LLM to pose a question.
The other, relaxed, would interleave this question instruction with
an instruction to comment on the conversation in a nonquestion way.
The conversation style variable was deployed for purely explorative purposes.

All the chatbots had the same persona instruction except their names.
They would communicate in Czech with participants who were Czech or slovak native speakers.
The first utterance in the conversation would be the chatbots and
would be hardcoded to contain greeting and self introduction.

They would be instructed to chat for several turns and then the stimulus would come.
After the stimulus the chatbot would continue for a couple more turns and then say goodbye.
The questionnare would then appear to the participant.
That is unless the participant aborted earlier, in which case the questionnare would appear immediately.
In some cases participants simply left the user interface leaving no extra information.
Other then that, participants were asked to grade the conversation on scale 1 to 5
where 1 was most acceptable and 5 was least acceptable.

Timing of the prompt would differ based on the anaphora type.
For shallow type the chatbot simply conversates for two turns and thereafter it starts tracking entities.
At first recognized entity the prompt fires off
prompting an utterance about the exctracted entity and anaphorizing it.
With deep anaphora similar thing happens,
but the chatbot waits several more entity occurences
before also prompting an utterance about the extracted entity and anaphorizing it.
Lastly the nonassignable anaphora has the same timing method as the shallow anaphora,
but the prompt itself is hardcoded to contain an anaphoric reference which is
very unlikely to be assignable.

In the experiment the participant would be instructed to end the conversation,
if the chatbots communication wasnt natural.
The concrete Czech word used was "přirozená komunikace".

\subsection{Participants}
The data has been collected in two waves.
The pilot wave generated 50 conversations
the participants for which were recruited from the networks of the papers author.
In the attached full data
the identificators of these conversations have
the letter p for pilot appended at the end.
The second wave generated 325 conversations
the participants for which were recruited from the students of the Charles University.
They were instructed to take two conversations with a week pause in the middle.
However since for this experiment explorative in nature contrast and timing were not critical
they were free to revisit the chatbot however they liked.
There is no attempt made to indicate unique participants in the conversation as
this would not bring any observable results.
If such approach should be taken in the future,
more control has to be gained over the course of the converstaion
as is shown below.
No changes were made to the design for the second wave other then changing the structure of the qustionnare.
While in the pilot wave was tasked to make a comment on the conversation or
explain why they aborted it if they did so,
the second wave was shown the entire conversation
and was asked to mark and comment on specific chatbot responses they found odd or interesting.


\subsection{Annotation}

All 375 conversations were visited to confirm whether
the stimulus expected based on the conversation design.
Other than that the participant reaction was annotated to be either
continuation, metacommunication or aborting of the conversation.
These three types of participant reaction to stimulus
represent the amount of understanding the reference.
In simple terms, continuation means
the stimulus went under the participants radar
or has been accepted.
Metacommunication, an utterance which somehow addresses the ongoing communication,
shows the participant not understanding,
but perhaps attributing the misunderstanding to themselves or
believing the issue can be resolved.
Lastly the conversation is aborted when the participant looses trust
in the ongoing conversation being coherent now or in the future.
Then the illusion of social interaction falls
and the conversation the participant is having dissolves into
typing text into a computer program and getting text in return.
The immersivness necessary for simulating a conversation is here no longer achieved.

\subsection{Inter--annotator agreement}

To make sure that annotating the data is possible,
inter-annotator agreement has been measured on 37 conversations which gives 10% of the data.
Three annotators annotated each conversation for anaphora type and
whether the participant reaction to stimulus is metacommunicative.
Metacommunication is the only reaction type that needs manual annotation since
aborting the conversation is
observable in the transcript and marked in the data automatically
and any reaction that is not aborting and metacommunication has to be a continuation.

\subsubsection{Stimulus}

For anaphora type Cohen's Kappa was calculated for each of the three annotator pairs
with following results:

\begin{itemize}[label={}]
    \item Cohen's Kappa for Annotator 1 and Annotator 2: 0.89
    \item Cohen's Kappa for Annotator 1 and Annotator 3: 0.88
    \item Cohen's Kappa for Annotator 2 and Annotator 3: 0.92
\end{itemize}

Coming to the average Pairwise Cohen's Kappa of 0.90
the inter-annotator agreement appears to be high enough
that this annotation categorization to be used and trusted.

Each of the conversations that caused disagreement is interesting in its own right:


\begin{quote}
\begin{itemize}[label={}, leftmargin=0pt, itemsep=0.5em]
    \item \textbf{conversation id: 145}
    \item ...
    \item \utterance{b:}{Jaké dárky plánuješ koupit?}{Which gifts do you intend to buy?}
    \item \utterance{h:}{ještě nevím, potřebuji pro taťku nějaký dárek a pro mamku}{don't know yet, need for dad a gift and for mom}
    \item \utterance{b:}{To může být výzva, ale určitě najdeš něco skvělého.}{That can be a challenge, but surely you'll find something great.}
    \item \utterance{h:}{jo, taky si myslím}{yeah, i also think}
    \item \utterance{b:}{Jaké koníčky má tvůj táta a mamka?}{Which hobbies does your dad a mom have?}
    \item \utterance{h:}{nevím, neznám je}{don't know, don't know them}
    \item \utterance{b:}{To je škoda, můžeš se jich zeptat.}{That is a shame, you can ask them.}
    \item \utterance{h:}{bydlí moc daleko}{they live too far}
    \item \utterance{b:}{Bot: To je těžké, ale co si zjistit jejich zájmy?}{That is tough, but how about finding out what their interests are?}
    \item \utterance{h:}{taky nevím§}{also don't know}
    \item \utterance{b:}{Možná bys mohl zjistit, co mají rádi.}{maybe you could look into, what they like}
    \item \utterance{h:}{ok}{ok}
    \item \utterance{b:}{To je dobrý nápad, jak to uděláš?}{That's a good idea, how will you do it?}
    \item \utterance{h:}{zeptáím se tety, ta určitě bude vědět}{I'll ask my aunt, she'll surely know}
    \item \utterance{b:}{Jaké mají oni koníčky?}{Which hobbies do they have?}
    \item \utterance{h:}{kdo?}{who?}
    \item ...
\end{itemize}
\end{quote}

This conversation potentially has a deep anaphora in it,
since the first occurence of the mappable entity \textit{dad and mom} is deep in the conversations history
at the stimulus time.
There are however other entites that are mappable
that are mentioned in the meantime like \textit{aunt}
or the general group of \textit{close ones} mentioned before the snippet starts.
This points at the potentially missing category of \textit{ambivalent anaphora}.
Another issue with this stimulus is also that it askes a question that is already being discussed.
This can be seen as a both topical and sequentical coherence disruption.
Why would the same question with the same reference in it be posed?
That also makes the anaphora that much more vague and leads to a metacommunication \textit{who?}.

\begin{quote}
\begin{itemize}[label={}, leftmargin=0pt, itemsep=0.5em]
\item \textbf{conversation id: 150}
\item ...
\item \utterance{h:}{Běhám 400m překážek a dělám sedmiboj}{I run 400m with obstacles and do heptathlon}
\item \utterance{b:}{To je náročné! Jak dlouho už se tomu věnuješ?}{That's demanding! How long have you been doing this?}
\item ... 7 turns (14 utterances) ...
\item \utterance{h:}{Jsem po zánětu v patě, tak snad se to nevrátil, hodně to uvolňuji a cvičím na to}{I had a heel inflammation, hopefully it won't return, I release and excersise it a lot}
\item \utterance{b:}{Jaké disciplíny tvoří on? Jakou máš mezi nimi nejoblíbenější?}{Which disciplines does it consist of? Which is you favorite among them?}
\item \utterance{h:}{Kdo?}{Who?}
\item ...
\end{itemize}
\end{quote}

Here the mappable entity \textit{heptathlon} is so far deep in the conversation that
the annotator who would not agree with calling a deep anaphora stimulus
perhaps did not even notice it or insisted that this is just too deep.
While that is technically against the annotators instruction based on the above description of stimuli
but it points at the interesting problematic of anaphora depth.

The other two conversations that had an annotation disagreement about them were IDs 151 and 153.
Both of them have an imperfect generation in them that excludes them from the stimuli containing data.

\subsubsection{Metacommunication}

Inter-annotator agreement regarding the metacommunication has produced following results:

\begin{itemize}[label={}]
\item Cohen's Kappa for Annotator 1 and Annotator 2: 0.94
\item Cohen's Kappa for Annotator 1 and Annotator 3: 0.88
\item Cohen's Kappa for Annotator 2 and Annotator 3: 0.94
\end{itemize}

With the average Pairwise Cohen's Kappa coming to 0.92,
it can be considered reliable to annotate metacommunication
in the collected data.

The two conversations that had disagreement about them in terms of
presence of metacommunication in the participant reaction to stimulus
are both interesing in their own right.


\begin{quote}
\begin{itemize}[label={}, leftmargin=0pt, itemsep=0.5em]
\item \textbf{conversation id: 164}
\item ...
\item \utterance{b:}{Jakou písničku z ní máš nejraději?}{Which song from here do you like the best?}
\item \utterance{h:}{z rádia?}{from the radio?}
\item ...
\end{itemize}
\end{quote}

This reaction can be seen to contain an ellided metacommunicative device like \textit{do you mean}.
Since it is not present explicitely and the participant reaction is topically anchored in the conversation,
the case is ambivalent.

\begin{quote}
\begin{itemize}[label={}, leftmargin=0pt, itemsep=0.5em]
\item \textbf{conversation id: 173}
\item ...
\item \utterance{b:}{Tak to by mě zajímalo, jak na to pak ona zareagovala.}{Then I wonder, what was her reaction to that.}
\item \utterance{h:}{Ano, pokazilo}{Yes, it went wrong}
\item ...
\end{itemize}
\end{quote}

It is unclear where the participant is coming from in their response.
For this reason, it is also unclear whether the participant reaction is a metacommunication.
There is a good chance the participant is somehow commenting on the chatbot breaking,
in which case such utterance would indeed be considered to address the ongoing communication.

\par
In conclusion the annotator agreement is
strong enough for both stimulus type and metacommunication reaction and
quantitative analysis can be performed on the measured metrics.

\section{Quantitative Analysis}

\subsection{Expectations}

In the performed quantitative analysis,
following expectations were used as reference to the found results.

For user rating the rating was expected to be ordered like so from best to worse:
shallow, deep, nonassignable.

As far as aborting the conversation goes,
The amount of aborted conversations were expected to be ordered
from least to most als:
shallow, deep, nonassignable.

The reactions were expected to map onto the stimuli like so:

\begin{quote}
\begin{tabbing}
\hspace{4cm} \= \hspace{4cm} \= \kill % set up tab stops
Shallow \> Continue \\
Deep \> Meta \\
Nonassignable \> Abort \\
\end{tabbing}
\end{quote}

As far as conversation style, the inquisitive variant was expected to have a worse score
since every next question shifts the topic forwards, which should generate a topical progression
that sooner or later becomes incoherent as Hrbáček demonstrates ().

\subsection{Result}

\subsubsection{Ratings}

The collected ratings have the following distribution:

\begin{quote}
\begin{tabbing}
\hspace{4cm} \= \hspace{4cm} \= \kill % set up tab stops
\textbf{rating} \> \textbf{times} \\
Rating 1 \> 22 \\
Rating 2 \> 62 \\
Rating 3 \> 138 \\
Rating 4 \> 37 \\
Rating 5 \> 6 \\
No rating \> 19 \\
\end{tabbing}
\end{quote}

The average rating of the collected conversations
which have been rated by participants is 2.77.
Since the missing ratings are participants who clicked off of the conversation
the absence of their rating can be interpreted as the worst rating.
Then the average rating of all the conversations is 2.97.

The instruction to the participants was to rate the conversation
using the same system as the grading in czech public school system.
That may explain the majority of ratings being a 2 or a 3.
The ratings in the collected data can only be compared against each other and
for them to give a better picture of the stimulus effect,
a more complex experiment design would need to be deployed
containing a reference conversation and its rating for each participant.

\subsubsection{Stimulus and reaction}

The annotation result is as following:

\begin{quote}
\begin{tabbing}
\hspace{4cm} \= \hspace{4cm} \= \kill % set up tab stops
\textbf{anaphora type} \> \textbf{times} \\
Shallow \> 104 \\
Deep \> 64 \\
Nonassignable \> 116 \\
Other \> 85
\end{tabbing}
\end{quote}

Ignoring the conversation that lack a rating the result is following:

\begin{quote}
\begin{tabbing}
\hspace{4cm} \= \hspace{4cm} \= \kill % set up tab stops
\textbf{anaphora type} \> \textbf{times} \\
Shallow \> 102 \\
Deep \> 58 \\
Nonassignable \> 105 \\
Other \> 71
\end{tabbing}
\end{quote}

There is less conversation with the deep anaphora stimulus.
This is because the deep anaphora stimulus requires the most difficult operation by the dialog system
and is therefore the most likely to fail.
The annotation result also confirms its necessity
given almost a fifth of the conversations were marked as \textit{other} and
thereby excluded from the stimulus reactions quantitative analysis.

\subsection{Design rating}

The rating for the various conversation types came out to be following:

\begin{quote}
\begin{tabbing}
\hspace{4cm} \= \hspace{4cm} \= \kill % set up tab stops
\textbf{anaphora type} \> \textbf{rating} \\
Shallow \> 2.75 \\
Deep \> 2.53 \\
Nonassignable \> 2.95 \\
\end{tabbing}
\end{quote}

Opposed to expectations shallow anaphora stimulus type has worse score than deep anaphora.
P-value from Kruskal-Wallis test being 0.02 is sufficiently low, but
when values stimulus types are compares post-hoc via Dunn test,
result come to:

\begin{quote}
\begin{tabbing}
\hspace{4cm} \= \hspace{4cm} \= \kill % set up tab stops
\textbf{anaphora types} \> \> \textbf{p-value} \\
Deep \> Nonassignable \> 0.005 \\
Deep \> Shallow \> 0.158 \\
Nonassignable \> Shallow \> 0.136 \\
\end{tabbing}
\end{quote}

The only sufficiently low p-value is between deep and nonassignable anaphora.
The results only partially correspond with the expectaions.
Number rating is however noisy as other things happen in conversation that affect it.
The chatbot contributions to the conversation outside of the stimulus
has to be under control and produce natural responses in order to really be able
to view the ratings as reflecting the stimuli
hence the high p-values.

\subsubsection{Design rating with abandonment interpretation}

Some conversations do not have a rating because participant clicked off of the experiment webpage.
Since the participants were instructed to leave the conversation if it is not \textit{natural},
these conversations can be interpreted as worse possible reaction - 5.

We can further split these into a situation where
the participant clicked off of the experiment
right after the stimulus.
This only happened in case of nonassignable anaphora design.
Its new average score comes to 3.06.

If the participant left the conversation at any time after the stimulus,
this can be because the stimulus disturbed the conversation coherence
and while it was not serious enough at the stimulus time for the participant
to end the conversation,
the coherence never recovered.
If we interpret conversations abandoned at any point after stimulus as rated with the worse possible rating,
we get following results:

\begin{quote}
\begin{tabbing}
\hspace{4cm} \= \hspace{4cm} \= \kill % set up tab stops
\textbf{anaphora type} \> \textbf{rating} \\
Shallow \> 2.80 \\
Deep \> 2.77 \\
Nonassignable \> 3.15 \\
\end{tabbing}
\end{quote}

Shallow anaphora that is expected to have the best result still gets a slightly worse rating than the deep one.
The differences between these ratings and the ones where abandoned conversations are excluded are:

\begin{quote}
\begin{tabbing}
\hspace{4cm} \= \hspace{4cm} \= \kill % set up tab stops
\textbf{anaphora type} \> \textbf{difference} \\
Shallow \> -0.04 \\
Deep \> -0.23 \\
Nonassignable \> -0.19 \\
\end{tabbing}
\end{quote}

The difference shows that the shallow anaphora had by far the least number of abandoned conversations.
Updated significance check provides a slightly different picture with
Kruskal-Wallis test p-value being a similar value, but
the post-hoc Dunn test showing more significant differences:

\begin{quote}
\begin{tabbing}
\hspace{4cm} \= \hspace{4cm} \= \kill % set up tab stops
\textbf{anaphora types} \> \> \textbf{p-value} \\
Deep \> Nonassignable \> 0.016 \\
Deep \> Shallow \> 0.653 \\
Nonassignable \> Shallow \> 0.019 \\
\end{tabbing}
\end{quote}

Here the results are more significant but still not enough for deep vs shallow anaphora.
In conclusion the shallow anaphora was not rated as expected in relation to the other stimuli, but
the difference does not seem to be significant and
in accordance with the expectations has the lowest number of abandoned conversations.
There are many factors in the conversation data and
stimulus does not seem to have a strong enough effect on the rating.
If better more seamless conversation could be simulated,
perhaps stronger effect of the stimulus on rating could be observed.
The small difference can also be attributed to the fact that participants tend to
avoid extremes in their grading.

\subsection{Reaction to stimulus}

A quantitative metric that is expected to show better the acceptability and understandability
of the various stimulus designs is the direct conversational reaction to them.
The annotated data came to the following result:

\begin{quote}
\begin{tabbing}
\hspace{4cm} \= \hspace{3cm} \= \hspace{3cm} \= \hspace{3cm} \= \kill % set up tab stops
\textbf{anaphora type} \> \textbf{continuation} \> \textbf{meta} \> \textbf{abort} \\
Shallow         \> 91 \> 8 \> 5 \\
Deep            \> 34 \> 26 \> 4 \\
Nonassignable   \> 13 \> 57 \> 46 \\
\end{tabbing}
\end{quote}

Which percentage wise gives the following:

\begin{quote}
\begin{tabbing}
\hspace{4cm} \= \hspace{3cm} \= \hspace{3cm} \= \= \hspace{3cm} \kill % set up tab stops
\textbf{anaphora type} \> \textbf{continuation} \> \textbf{meta} \> \textbf{abort} \\
Shallow         \> 87.50\% \> 7.69\% \> 4.81\% \\
Deep            \> 53.12\% \> 40.62\% \> 6.25\% \\
Nonassignable   \> 11.21\% \> 49.14\% \> 39.66\% \\
\end{tabbing}
\end{quote}

With the extremely low p-value coming to $P < 10^{-28}$
 there is no doubt
that the different stimuli have a clear effect on the participant reaction and
that the results will be replicable.
The expectations are met with the continuation being highest for shallow anaphora
and lowest for nonassignable anaphora while
the abort reaction proves to have the opposite tendency.

One unpredicted feature in the results is
deep anaphora having a relatively close percentage of continuation and meta reactions and
nonassignable anaphora sharing similar values for meta reactions and abort reactions.
This suggests there is a inner division in the data that can be described and simulated more closely.

\subsubsection{Stimulus x Reaction specific ratings}

Looking at ratings for specific groups of stimuli, results are following:

\begin{quote}
\begin{tabbing}
\hspace{4cm} \= \hspace{3cm} \= \hspace{3cm} \= \= \hspace{3cm} \kill % set up tab stops
\textbf{anaphora type} \> \textbf{continuation} \> \textbf{meta} \> \textbf{abort} \\
Shallow \> 2.73 \> 3.12 \> 2.60 \\
Deep \> 2.43 \> 2.62 \> 2.75 \\
Nonassignable \> 2.45 \> 3.04 \> 2.98 \\
\end{tabbing}
\end{quote}

Interestingly for shallow and nonassignable anaphora,
the ratings do not follow the expected course.
The unfulfiled expectation lies in the meta reaction having worse ratings than the abort reactions.
The only anaphora type the expectation were met for is the deep one.
Yet the significance of this result is low and
more control over the conversation outside of the stimulus needs to be acquired
to be able to rely on these results.
To make conclusions about the meta reaction being indicative of
a less acceptable experience in participant
would be hurried.

\subsection{Comments}

While annotating, participants comments were collected
to gain an extra metric on the reflection of the stimuli.
While the shallow anaphora was not mentioned a single time
and generally went unnoticed,
the deep anaphora got 11 mentions, which makes for
slightly less than a fifth of conversations with this stimulus.
Finally the nonassignable anaphora was mentioned 69 times in the participant comments.
This makes over half of its occurences were commented upon.
All the comments expressed confusion about mapping the anaphora to a possible preceding referent.
This result follows the expectations.

\subsection{Conversation style}

As mentioned above, there was an attempt
to expose different participants to different conversation styles.
This came out of the neccesity to create a default way for the chatbot to communicate.
First style is the inquisitive, where the chatbot is prompted to ask questions.
An elaboration on it is a style of conversation,
where the chatbot is at one time prompted to ask questions and
at another to simply comment on the topic at hand.
There were 183 conversations collected with the chatbot communicating in an inquisitive style
and 128 conversations where the chatbot communicated in the second, more relaxed style.
The relaxed style has been expected to have a better score as was confirmed
with following result:

\begin{quote}
\begin{tabbing}
\hspace{4cm} \= \hspace{3cm} \kill % set up tab stops
\textbf{conversation style} \> \textbf{rating} \\
Inquisitive \> 2.63\\
Relaxed \> 2.45 \\
\end{tabbing}
\end{quote}

However the Kruskal-Wallis test p-value of 0.10 suggests too small effect
to consider this result replicable.
To find statistically significant effect on conversation styles
the conversation needs to be controlled in a more nuanced way by the dialog system.

The relaxed style however better rated on average also suffered from some issues
in terms of generating natural answers.
Sometimes the chatbot would simply repeat the participants response without adding anything new to it.
For example in following conversation lack of topical input lead to aborting the conversation:

\begin{quote}
\begin{itemize}[label={}, leftmargin=0pt, itemsep=0.5em]
\item \textbf{conversation id: 164}
\item ...
\item \utterance{b:}{To chápu, co bys chtěl v histologii zlepšit?}{I understand, what would you like to improve about histology}
\item \utterance{h:}{chtela bych ji nemit proste}{I would just like to not have it}
\item \utterance{b:}{To chápu, chtěla bys ji prostě nemít."}{I understand, you would just like to not have it.}
\item \textit{aborted}
\end{itemize}
\end{quote}


\subsection{Zero anaphora}

Since Czech is a prodrop language (),
it is expected that where the stimulus anaphora stands in subject position,
there will be tendency of it being dropped.
A response with zero anaphora \footcite{zero} would then be expected to be more acceptable
than an equivalent response with an explicit one.

During annotation there have been found to be only 16 conversations with zero anaphora stimulus.
This is partially due to the prompting of the stimulus,
the few-shot prompt contained no example of a zero anaphora and
partially because a lot of the stimuli contained an object positioned anaphora.

Coincidently the stimuli containing zero anaphora were split equally between shallow and deep anaphora stimuli.
The nonassignable anaphora stimuli mostly consisted of a hardcoded response that already contained the anaphora
therefore it is not a part of the zero anaphora collection of conversations.

\begin{quote}
\begin{tabbing}
\hspace{4cm} \= \hspace{3cm} \kill % set up tab stops
\textbf{reaction type} \> \textbf{\# occurences}\\
Continuation \> 14 \\
Meta \> 2 \\
Abort \> 0  \\
\end{tabbing}
\end{quote}

This result very clearly shows that the zero-anaphora stimuli were mostly going by unnoticed
and have a higher level of acceptability.
The same goes for rating with following result:

\begin{quote}
\begin{tabbing}
\hspace{4cm} \= \hspace{3cm} \kill % set up tab stops
\textbf{anaphora type} \> \textbf{rating}\\
Zero \> 2.19 \\
Nonzero \> 2.73 \\
\end{tabbing}
\end{quote}

Only shallow and deep anaphora type stimuli were used for this calculation.
The zero anaphora score is a lot better considering the range of averages
that the data has produced so far.
This shows future designs need to take zero-anaphora into account more
in order to achieve stronger contrast between stimuli.

The data contains conversations where stimulus was
inappropriately generated with a nonzero anaphora like here:

\begin{quote}
\begin{itemize}[label={}, leftmargin=0pt, itemsep=0.5em]
\item \textbf{conversation id: 347}
\item ...
\item \utterance{h:}{...hráli jsme je přes aplikaci ale já mu k vánocům koupila pravou šachovnici, takže konečně můžeme hrát normálně.}{... we would play via an app but I bought him a chessboard for christmas so we can finally play normally.}
...
\item \utterance{b:}{Jakou máte ji, klasickou nebo něco speciálního?}{Which do you have her, classic or something special?}
\item \utterance{h:}{jak to myslíš?}{what do you mean?}
\item ...
\end{itemize}
\end{quote}

The meta reaction is likely caused by the depth of the anaphora,
but the bot response would have been more likely had the anaphora not been a part of it.

In some cases zero anaphora can also be inappropriate like here:

\begin{quote}
\begin{itemize}[label={}, leftmargin=0pt, itemsep=0.5em]
\item \textbf{conversation id: 357}
\item ...
\item \utterance{h:}{Nooo teď jsme jeli kolem břevnovského kláštera, nikdy jsem tam nebyla, tak bych zajela tam.}{Well we just drove past Břevnov monastery, I never visited, so I would go there.}
\item \utterance{b:}{To zní zajímavě! Co tě na Břevnovském klášteře láká nejvíce?}{That sounds interesting! What about Břevnov monastery is the most attractive for you?}
\item \utterance{h:}{To nevím právě, překvapilo me to jak je to velký.}{I don't know, it surprised me how big it is.}
\item \utterance{b:}{Máš rád objevování nových míst a památek?}{Do you like discovering new places and monuments?}
\item \utterance{h:}{Ráda bych na to měla více prostoru.}{I would like to have more space to do it.}
\item \utterance{b:}{To chápu! Jaké místo bys chtěla objevit jako první?}{I get that! Which place would you like to discover first?}
\item \utterance{h:}{No nevím ale třeba bys mi mohl doporučit nějaký hezký místo v Česku spjato i s nějakou novou moderní architekturou, možná nějaká vyhlídka.}{I don't know but maybe you could recommend a nice place in Czechia connected with some new modern architecture, maybe a lookout tower.}
\item \utterance{b:}{Slyšela jsi o historii a zajímavostech?}{Have you heard about history and trivia?}
\item \textit{aborted}
\end{itemize}
\end{quote}

The participant commented after the conversation that the question did not fit in with the context
which is why they aborted.
That is reasonable as the response was generated based on the monastery topic from deeper in the conversation.
Perhaps if there was an explicit anaphora here, the mapping would have been easier
as dropping in a nonsubject position is not preferred in Czech.
Regardless, the hard topical shift that disrupts the coherence
caused by the chatbotwould remain in the conversation.

\section{Qualitative analysis}

2
\subsubsection{Pro-coherence perception}
\paragraph{Remaps}
    100, 316, 53, 257 will map if can
    126, 49, 330, 157, 175, 236 remap
    259 epic remap
    36 jaké další knihy ti doporučili ? general third person is accepted
    276, 275 nonassignable interpreted as third person participant reference
    243 misunderstood reference
    limiting context for deep anaphora can lead to anaphora being open to a remap
\paragraph{Indirect anaphora -- Pseudoanaphora}
    hardcoded phrase with intended nonassignable anaphora is sometimes assignable by association
    106, 271p, 212, 248, 260 pseudoanaphora
    149 unmapped pseudoanaphora (mediace při rozvodu)
    spectrum between pseudoanaphora and a newly mentioned entity anaphorized
\paragraph{Sequential coherence}
    123 sequentially interesting

2
\subsection{Deep anaphora}
    too simplified a notion
    remaps
    253 perfect
    30 clear mapping due to verb
\paragraph{Inappropriate pragmatics}
    127, 145 repeating a question - topical/pragmatic incoherence
        limiting context for deep anaphora tends to generate inappropriate pragmatics
    162, 114 stimulus understood but topic shift not accepted

\paragraph{Incomplete control over conversation - topic status}
    96, 168, 192, 203, 240, 245 not moving away from topic
    337 anaphora refers to hyperonymum of previous utterance entities
    312 vague anaphora, but on topic

    deep anaphora design has multiple scenarios it tends to go in:
    deep with generation issue
    deep with acceptance
    deep within same topic, other within same topic

    deep anaphora acceptance depends on
    whether the topic shift is annotated, accepted and
    whether there is enough information to perform mapping

1
\subsection{Nonassignable anaphora}
mostly fine

\paragraph{Returning to entities that havent been topicalized}
    162 graph according to Daneš

    b: Jaký je T1?
    h: R2 a R3
    b: Jaký je T2?
    h: R2a R2b R2c
    dodělat !!

    entity of q anaphora returned to is not topicalized at any point
    as opposed to 145 where entity of q anaphora is topicalized
    ...

\paragraph{acceptance}
    142 changed topic upon nonassignable
    279 continuing despite nonassignability

1
\paragraph{Evaluation of shallow, deep, nonassignable categorization}
deep anaphora (if accepted) tends to be seen as an abrupt shift to previous topic
nonassignable anaphora (if accepted) tends to be seen as a cataphore
    cataphore is unimportant in conversation, it is a text phenomena
    what is important is there is trust (or tolerance)
		It is hard to just change the anaphora - always more factors, some stronger in their influence on coherence:

1
\subsection{Outside of stimuli}

\subsubsection{Topical coherence}
    264p topic incoherence
    190 didnt get to stimulus, not enough topic in conversation?

\subsubsection{Metacommunication}
    317 metacommunication
    270 double anaphora -> meta
    274 almost meta
    101 meta but ended shortly after, coherence would not recover

1
\section{Conclusion}
