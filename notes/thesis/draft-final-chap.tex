
\section{Data collection}

\subsection{Experimental designs}

Data collection took place with the use of six chatbots.
They each represent a possible combination of a 3x2 design.
The two design variables are anaphora type and conversation style.

The anaphora types are as discussed above a shallow, deep and unassignable.
The conversation styles were two. One, inquisitive,
where the instruction part of the prompt
would always instruct the LLM to pose a question.
The other, relaxed, would interleave this question instruction with
an instruction to comment on the conversation in a nonquestion way.
The conversation style variable was deployed for purely explorative purposes.

All the chatbots had the same persona instruction except their names.
The first utterance in the conversation would be the chatbots and
would be hardcoded to contain greeting and self introduction.

They would be instructed to chat for several turns and then the stimulus would come.
After the stimulus the chatbot would continue for a couple more turns and then say goodbye.
The questionnare would then appear to the participant.
That is unless the participant aborted earlier, in which case the questionnare would appear immediately.
In some cases participants simply left the user interface leaving no extra information.
Other then that, participants were asked to grade the conversation on scale 1 to 5
where 1 was most acceptable and 5 was least acceptable.

Timing of the prompt would differ based on the anaphora type.
For shallow type the chatbot simply conversates for two turns and thereafter it starts tracking entities.
At first recognized entity the prompt fires off
prompting an utterance about the exctracted entity and anaphorizing it.
With deep anaphora similar thing happens,
but the chatbot waits several more entity occurences
before also prompting an utterance about the extracted entity and anaphorizing it.
Lastly the nonassignable anaphora has the same timing method as the shallow anaphora,
but the prompt itself is hardcoded to contain an anaphoric reference which is
very unlikely to be assignable.

In the experiment the participant would be instructed to end the conversation,
if the chatbots communication wasnt natural.
The concrete czech word used was "přirozená komunikace".

\subsection{Participants}
The data has been collected in two waves.
The pilot wave generated 50 conversations
the participants for which were recruited from the networks of the papers author.
In the attached full data
the identificators of these conversations have
the letter p for pilot appended at the end.
The second wave generated 325 conversations
the participants for which were recruited from the students of the Charles University.
They were instructed to take two conversations with a week pause in the middle.
However since for this experiment explorative in nature contrast and timing were not critical
they were free to revisit the chatbot however they liked.
There is no attempt made to indicate unique participants in the conversation as
this would not bring any observable results.
If such approach should be taken in the future,
more control has to be gained over the course of the converstaion
as is shown below.
No changes were made to the design for the second wave other then changing the structure of the qustionnare.
While in the pilot wave was tasked to make a comment on the conversation or
explain why they aborted it if they did so,
the second wave was shown the entire conversation
and was asked to mark and comment on specific chatbot responses they found odd or interesting.

2
\subsection{Annotation}

All 375 conversations were visited to confirm whether
the stimulus expected based on the conversation design.
Other than that the participant reaction was annotated to be either
continuation, metacommunication or aborting of the conversation.
These three types of participant reaction to stimulus
represent the amount of understanding the reference.
In simple terms, continuation means
the stimulus went under the participants radar
or has been accepted.
Metacommunication, an utterance which somehow addresses the ongoing communication,
shows the participant not understanding,
but perhaps attributing the misunderstanding to themselves or
believing the issue can be resolved.
Lastly the conversation is aborted when the participant looses trust
in the ongoing conversation being coherent now or in the future.
Then the illusion of social interaction falls
and the conversation the participant is having dissolves into
typing text into a computer program and getting text in return.
The immersivness necessary for simulating a conversation is here no longer achieved.

\subsection{Inter--annotator agreement}

To make sure that annotating the data is possible,
inter-annotator agreement has been measured on 37 conversations which gives 10% of the data.
Three annotators annotated each conversation for anaphora type and
whether the participant reaction to stimulus is metacommunicative.
Metacommunication is the only reaction type that needs manual annotation since
aborting the conversation is
observable in the transcript and marked in the data automatically
and any reaction that is not aborting and metacommunication has to be a continuation.

\subsubsection{Stimulus}

For anaphora type Cohen's Kappa was calculated for each of the three annotator pairs
with following results:

\begin{itemize}
    \item Cohen's Kappa for Annotator 1 and Annotator 2: 0.89
    \item Cohen's Kappa for Annotator 1 and Annotator 3: 0.88
    \item Cohen's Kappa for Annotator 2 and Annotator 3: 0.92
\end{itemize}

Coming to the average Pairwise Cohen's Kappa of 0.90
the inter-annotator agreement appears to be high enough
that this annotation categorization to be used and trusted.

Each of the conversations that caused disagreement is interesting in its own right:


\textbf{Conversation ID: 145}

\begin{itemize}[label={}, leftmargin=0pt, itemsep=0.5em]
    \item ...
    \item \utterance{b:}{Jaké dárky plánuješ koupit?}{Which gifts do you intend to buy?}
    \item \utterance{h:}{ještě nevím, potřebuji pro taťku nějaký dárek a pro mamku}{don't know yet, need for dad a gift and for mom}
    \item \utterance{b:}{To může být výzva, ale určitě najdeš něco skvělého.}{That can be a challenge, but surely you'll find something great.}
    \item \utterance{h:}{jo, taky si myslím}{yeah, i also think}
    \item \utterance{b:}{Jaké koníčky má tvůj táta a mamka?}{Which hobbies does your dad a mom have?}
    \item \utterance{h:}{nevím, neznám je}{don't know, don't know them}
    \item \utterance{b:}{To je škoda, můžeš se jich zeptat.}{That is a shame, you can ask them.}
    \item \utterance{h:}{bydlí moc daleko}{they live too far}
    \item \utterance{b:}{Bot: To je těžké, ale co si zjistit jejich zájmy?}{That is tough, but how about finding out what their interests are?}
    \item \utterance{h:}{taky nevím§}{also don't know}
    \item \utterance{b:}{Možná bys mohl zjistit, co mají rádi.}{maybe you could look into, what they like}
    \item \utterance{h:}{ok}{ok}
    \item \utterance{b:}{To je dobrý nápad, jak to uděláš?}{That's a good idea, how will you do it?}
    \item \utterance{h:}{zeptáím se tety, ta určitě bude vědět}{I'll ask my aunt, she'll surely know}
    \item \utterance{b:}{Jaké mají oni koníčky?}{Which hobbies do they have?}
    \item \utterance{h:}{kdo?}{who?}
    \item ...
\end{itemize}

This conversation potentially has a deep anaphora in it,
since the first occurence of the mappable entity \textit{dad and mom} is deep in the conversations history
at the stimulus time.
There are however other entites that are mappable
that are mentioned in the meantime like \textit{aunt}
or the general group of \textit{close ones} mentioned before the snippet starts.
This points at the potentially missing category of \textit{ambivalent anaphora}.
Another issue with this stimulus is also that it askes a question that is already being discussed.
This can be seen as a both topical and sequentical coherence disruption.
Why would the same question with the same reference in it be posed?
That also makes the anaphora that much more vague and leads to a metacommunication \textit{who?}.

\textbf{Conversation ID: 150}

\begin{itemize}[label={}, leftmargin=0pt, itemsep=0.5em]
\item ...
\item \utterance{h:}{Běhám 400m překážek a dělám sedmiboj}{I run 400m with obstacles and do heptathlon}
\item \utterance{b:}{To je náročné! Jak dlouho už se tomu věnuješ?}{That's demanding! How long have you been doing this?}
\item ... 7 turns (14 utterances) ...
\item \utterance{h:}{Jsem po zánětu v patě, tak snad se to nevrátil, hodně to uvolňuji a cvičím na to}{I had a heel inflammation, hopefully it won't return, I release and excersise it a lot}
\item \utterance{b:}{Jaké disciplíny tvoří on? Jakou máš mezi nimi nejoblíbenější?}{Which disciplines does it consist of? Which is you favorite among them?}
\item \utterance{h:}{Kdo?}{Who?}
\item ...
\end{itemize}

Here the mappable entity \textit{heptathlon} is so far deep in the conversation that
the annotator who would not agree with calling a deep anaphora stimulus
perhaps did not even notice it or insisted that this is just too deep.
While that is technically against the annotators instruction based on the above description of stimuli
but it points at the interesting problematic of anaphora depth.

The other two conversations that had an annotation disagreement about them were IDs 151 and 153.
Both of them have an imperfect generation in them that excludes them from the stimuli containing data.

\subsubsection{Metacommunication}

Inter-annotator agreement regarding the metacommunication has produced following results:

\begin{itemize}
\item Cohen's Kappa for Annotator 1 and Annotator 2: 0.94
\item Cohen's Kappa for Annotator 1 and Annotator 3: 0.88
\item Cohen's Kappa for Annotator 2 and Annotator 3: 0.94
\end{itemize}

With the average Pairwise Cohen's Kappa coming to 0.92,
it can be considered reliable to annotate metacommunication
in the collected data.

The two conversations that had disagreement about them in terms of
presence of metacommunication in the participant reaction to stimulus
are both interesing in their own right.

\textbf{Conversation ID: 164}

\begin{itemize}[label={}, leftmargin=0pt, itemsep=0.5em]
\item ...
\item \utterance{b:}{Jakou písničku z ní máš nejraději?}{Which song from here do you like the best?}
\item \utterance{h:}{z rádia?}{from the radio?}
\item ...
\end{itemize}

This reaction can be seen to contain an ellided metacommunicative device like \textit{do you mean}.
Since it is not present explicitely and the participant reaction is topically anchored in the conversation,
the case is ambivalent.

\textbf{Conversation ID: 173}

\begin{itemize}[label={}, leftmargin=0pt, itemsep=0.5em]
\item ...
\item \utterance{b:}{Tak to by mě zajímalo, jak na to pak ona zareagovala.}{Then I wonder, what was her reaction to that.}
\item \utterance{h:}{Ano, pokazilo}{Yes, it went wrong}
\item ...
\end{itemize}

It is unclear where the participant is coming from in their response.
For this reason, it is also unclear whether the participant reaction is a metacommunication.
There is a good chance the participant is somehow commenting on the chatbot breaking,
in which case such utterance would indeed be considered to address the ongoing communication.

\par
In conclusion the annotator agreement is
strong enough for both stimulus type and metacommunication reaction and
quantitative analysis can be performed on the measured metrics.

\section{Quantitative Analysis}

\subsection{Expectations}

In the performed quantitative analysis,
following expectations were used as reference to the found results.

For user rating the rating was expected to be ordered like so from best to worse:
shallow, deep, nonassignable.

As far as aborting the conversation goes,
The amount of aborted conversations were expected to be ordered
from least to most als:
shallow, deep, nonassignable.

The reactions were expected to map onto the stimuli like so:

\begin{tabbing}
\hspace{4cm} \= \hspace{4cm} \= \kill % set up tab stops
Shallow \> Continue \\
Deep \> Meta \\
Nonassignable \> Abort \\
\end{tabbing}

As far as conversation style, the inquisitive variant was expected to have a worse score
since every next question shifts the topic forwards, which should generate a topical progression
that sooner or later becomes incoherent as Hrbáček demonstrates ().

1
\subsection{Result}

rating, reaction, comment

\paragraph{Rating overall}
Ratings between 1 and 5:

\begin{itemize}
\item Rating 1: 22 times
\item Rating 2: 62 times
\item Rating 3: 138 times
\item Rating 4: 37 times
\item Rating 5: 6 times
\item No rating: 19 times
\end{itemize}
Ratio stimulus shallow:deep:nonassignable:other 104:64:116:85
Ratio stimulus with rating shallow:deep:nonassignable:other 102:58:105:71

2
\subsubsection{Design rating}
number rating is very noisy, other things happen in conversation that affect it

\begin{itemize}
\item Rating shlw 2.7549019607843137
\item Rating deep 2.5344827586206895
\item Rating nass 2.9523809523809526
\end{itemize}

Surprisingly shallow anaphora has worse score than deep anaphora.

\begin{itemize}
\item Kruskal-Wallis Test p-value: 0.020462703785719412
\item Effect Size (Eta Squared): 0.029688178453053583
\item Comparison: deep vs nonassignable, p-value: 0.0048713423950883464
\item Comparison: deep vs shallow, p-value: 0.1582978409422613
\item Comparison: nonassignable vs shallow, p-value: 0.136917898603309
\end{itemize}

Only significant for deep vs nonassignable.

\subsubsection{Design rating with abandonment interpretation}

\paragraph{Abandoned directly after stimulus}

Some conversations do not have a rating (participant clicked off of the experiment webpage)
These conversations can be interpreted as worse possible reaction - 5.

We can further split these into a situation where the participant clicked off of the experiment
right after the stimulus - this only happened in case of nonassignable anaphora design.

\begin{itemize}
\item Ratio stimulus null rating shallow:deep:nonassignable 2:6:11
\item Rating incl null direct shlw 2.7549019607843137
\item Rating incl null direct deep 2.5344827586206895
\item Rating incl null direct nass 3.063063063063063
\end{itemize}

\begin{itemize}
\item Rating diff nonull vs null direct shlw 0.0
\item Rating diff nonull vs null direct deep 0.0
\item Rating diff nonull vs null direct nass -0.11068211068211031
\end{itemize}

\paragraph{Abandoned anytime}

Or we can interpret as worse rating if participant clicked off at any time:

\begin{itemize}
\item Rating incl null anytime shlw 2.798076923076923
\item Rating incl null anytime deep 2.765625
\item Rating incl null anytime nass 3.146551724137931
\end{itemize}

\begin{itemize}
\item Rating diff nonull vs null anytime shlw -0.0431749622926092
\item Rating diff nonull vs null anytime deep -0.2311422413793105
\item Rating diff nonull vs null anytime nass -0.1941707717569785
\end{itemize}

The difference shows that the shallow anaphora had by far the least people just leaving.

\begin{itemize}
\item Kruskal-Wallis Test p-value: 0.017698134990819445
\item Effect Size (Eta Squared): 0.02871385062422946
\item Comparison: deep vs nonassignable, p-value: 0.01597013599420015
\item Comparison: deep vs shallow, p-value: 0.652654604765214
\item Comparison: nonassignable vs shallow, p-value: 0.018874560457834404
\end{itemize}

Here the results are more significant but still not for deep vs shallow.
There are many factors in the conversation data and
stimulus does not seem to have a strong enough effect on the rating.
If better more seamless conversation could be simulated,
perhaps stronger effect of the stimulus on rating could be observed.
The small difference can also be attributed to the fact that participants tend to
avoid extremes in their grading.

1
\subsubsection{Reaction to stimulus}

\begin{itemize}
\item Reaction ratio shlw continuation:meta:abort 91:08:05
\item Reaction ratio deep continuation:meta:abort 34:26:04
\item Reaction ratio nass continuation:meta:abort 13:57:46
\end{itemize}

\begin{itemize}
\item Reaction percentage ratio shlw 87.50% : 7.69% : 4.81%
\item Reaction percentage ratio deep 53.12% : 40.62% : 6.25%
\item Reaction percentage ratio nass 11.21% : 49.14% : 39.66%
\end{itemize}

\begin{itemize}
\item Chi2 Statistic: 139.0566579414896
\item P-value: 4.4935956067223523e-29
\item Degrees of Freedom: 4
\item Expected Frequencies:
\item [[50.53521127 31.09859155 56.36619718]
\item [33.32394366 20.50704225 37.16901408]
\item [20.14084507 12.3943662  22.46478873]]
\end{itemize}

Significant difference.

1
\subsubsection{Stimulus x Reaction specific ratings}

\begin{itemize}
\item     | cntn | meta | abrt |
\item shlw | 2.73 | 3.12 | 2.60 | stimulus x stimulus reaction match True
\item deep | 2.43 | 2.62 | 2.75 | stimulus x stimulus reaction match True
\item nass | 2.45 | 3.04 | 2.98 | stimulus x stimulus reaction match True
\end{itemize}

\subsubsection{Comments}

\begin{itemize}
\item Types of comments:
\item Stimulus mentioned in comments 80 times
\item shlw mentioned 00 times in comments - goes unnoticed
\item deep mentioned 11 times in comments - sometimes mentioned
\item nass mentioned 69 times in comments - mentioned in over half of the cases
\end{itemize}

1
\subsubsection{Conversation style}

\begin{itemize}
\item Relaxed style convos 128
\item Inquisitive style convos 183
\item Inquisitive style rating 2.63
\item Relaxed style rating 2.45
\item Relaxed style addressed in comments: 1x positive 1x negative
\item Inquistive style addressed in comments: 1x negative
\end{itemize}

Kruskal-Wallis Test p-value: 0.10266771117843472\
Effect Size (Eta Squared): 0.009445412958117978

Very small effect.
To find statistically significant effect on conversation styles
the conversation needs to be controlled in a more nuanced way by the dialog system.

relaxed style is often reduced to repetition
67 relaxed style not appreciated

1
\subsubsection{Zero anaphora}

\footcite{zero}

Number of zero anaphora conversations: 16\
stimulus zero anaphora shallow:deep:nonassignable 08:08:00\
Rating zero anaphora continuation:meta:abort 14:02:00

Rating zero anaphora 2.1875\
Rating non zero anaphora 2.7291666666666665

Zero anaphora seems to have much better scores.
Anaphorization has to incorporate zero anaphora better in the future.\
357 here zero anaphora does not work\
347 shouldve dropped
!! pouze subjektové nuly !!

\section{Qualitative analysis}

2
\subsubsection{Pro-coherence perception}
\paragraph{Remaps}
    100, 316, 53, 257 will map if can
    126, 49, 330, 157, 175, 236 remap
    259 epic remap
    36 jaké další knihy ti doporučili ? general third person is accepted
    276, 275 nonassignable interpreted as third person participant reference
    243 misunderstood reference
    limiting context for deep anaphora can lead to anaphora being open to a remap
\paragraph{Indirect anaphora -- Pseudoanaphora}
    hardcoded phrase with intended nonassignable anaphora is sometimes assignable by association
    106, 271p, 212, 248, 260 pseudoanaphora
    149 unmapped pseudoanaphora (mediace při rozvodu)
    spectrum between pseudoanaphora and a newly mentioned entity anaphorized
\paragraph{Sequential coherence}
    123 sequentially interesting

2
\subsection{Deep anaphora}
    too simplified a notion
    remaps
    253 perfect
    30 clear mapping due to verb
\paragraph{Inappropriate pragmatics}
    127, 145 repeating a question - topical/pragmatic incoherence
        limiting context for deep anaphora tends to generate inappropriate pragmatics
    162, 114 stimulus understood but topic shift not accepted

\paragraph{Incomplete control over conversation - topic status}
    96, 168, 192, 203, 240, 245 not moving away from topic
    337 anaphora refers to hyperonymum of previous utterance entities
    312 vague anaphora, but on topic

    deep anaphora design has multiple scenarios it tends to go in:
    deep with generation issue
    deep with acceptance
    deep within same topic, other within same topic

    deep anaphora acceptance depends on
    whether the topic shift is annotated, accepted and
    whether there is enough information to perform mapping

1
\subsection{Nonassignable anaphora}
mostly fine

\paragraph{Returning to entities that havent been topicalized}
    162 graph according to Daneš

    b: Jaký je T1?
    h: R2 a R3
    b: Jaký je T2?
    h: R2a R2b R2c
    dodělat !!

    entity of q anaphora returned to is not topicalized at any point
    as opposed to 145 where entity of q anaphora is topicalized
    ...

\paragraph{acceptance}
    142 changed topic upon nonassignable
    279 continuing despite nonassignability

1
\paragraph{Evaluation of shallow, deep, nonassignable categorization}
deep anaphora (if accepted) tends to be seen as an abrupt shift to previous topic
nonassignable anaphora (if accepted) tends to be seen as a cataphore
    cataphore is unimportant in conversation, it is a text phenomena
    what is important is there is trust (or tolerance)
		It is hard to just change the anaphora - always more factors, some stronger in their influence on coherence:

1
\subsection{Outside of stimuli}

\subsubsection{Topical coherence}
    264p topic incoherence
    190 didnt get to stimulus, not enough topic in conversation?

\subsubsection{Metacommunication}
    317 metacommunication
    270 double anaphora -> meta
    274 almost meta
    101 meta but ended shortly after, coherence would not recover

1
\section{Conclusion}
