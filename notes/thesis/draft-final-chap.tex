


\chapter{Quantitative Analysis}

\section{Expectations}

In the performed quantitative analysis,
following results were expected to be found.

The user rating was expected to be ordered like so from best to worse:
shallow, deep, nonassignable.

As far as aborting the conversation goes,
The amount of aborted conversations were expected to be ordered
from least to most as:
shallow, deep, nonassignable.

The reactions were expected to map onto the stimuli like so:

\begin{quote}
\begin{tabbing}
\hspace{4cm} \= \hspace{4cm} \= \kill % set up tab stops
Shallow \> Continue \\
Deep \> Meta \\
Nonassignable \> Abort \\
\end{tabbing}
\end{quote}

As far as conversation style, the inquisitive variant was expected to have a worse score
since every next question shifts the topic forwards, which should generate a topical progression
that sooner or later becomes incoherent as Hrbáček demonstrates \footcite[p.~30]{hrbacek1994}.

\section{Result}

\subsection{Ratings}

The collected ratings are of the following distribution:

\begin{quote}
\begin{tabbing}
\hspace{4cm} \= \hspace{4cm} \= \kill % set up tab stops
\textbf{rating} \> \textbf{times} \\
Rating 1 \> 22 \\
Rating 2 \> 62 \\
Rating 3 \> 138 \\
Rating 4 \> 37 \\
Rating 5 \> 6 \\
No rating \> 19 \\
\end{tabbing}
\end{quote}

The average rating of the collected conversations
which have been rated by participants is 2.77.
Since the missing ratings are participants who clicked off of the conversation
the absence of their rating can be interpreted as the worst rating.
Then the average rating of all the conversations is 2.97.

The instruction to the participants was to rate the conversation
using the same system as the grading in czech public school system.
That may explain the majority of ratings being a 2 or a 3.
The ratings in the collected data can only be compared against each other and
for them to give a better picture of the stimulus effect,
a more complex experiment design would need to be deployed
containing an extra "standard" conversation for reference and its rating for each participant.

\subsection{Annotation result}

The annotation result is as following:

\begin{quote}
\begin{tabbing}
\hspace{4cm} \= \hspace{4cm} \= \kill % set up tab stops
\textbf{anaphora type} \> \textbf{times} \\
Shallow \> 104 \\
Deep \> 64 \\
Nonassignable \> 116 \\
Other \> 85
\end{tabbing}
\end{quote}

Ignoring the conversation that lack a rating the result is following:

\begin{quote}
\begin{tabbing}
\hspace{4cm} \= \hspace{4cm} \= \kill % set up tab stops
\textbf{anaphora type} \> \textbf{times} \\
Shallow \> 102 \\
Deep \> 58 \\
Nonassignable \> 105 \\
Other \> 71
\end{tabbing}
\end{quote}

There is less conversation with the deep anaphora stimulus than those of other stimuli.
This is because the deep anaphora stimulus requires the most difficult operation by the dialog system
and is therefore the most likely to fail.
The annotation result also confirms its necessity
given almost a fifth of the conversations were marked as \textit{other} and
thereby excluded from the stimulus reactions quantitative analysis.
Conversation designs success in confronting the participant with stimuli is following:

\begin{quote}
\begin{tabbing}
\hspace{4cm} \= \hspace{4cm} \=  \hspace{4cm} \= \kill % set up tab stops

        \textbf{anaphora} \> \textbf{sucess} \> \textbf{fail} \\

        {} \> 256 \> 113 \\
        Shallow \> 85 \> 42 \\
        Deep \> 63 \> 61 \\
        Nonassignable \> 108 \> 10 \\

    \end{tabbing}
    \end{quote}

Overal approximately two thirds of the stimuli went through with success.
The slight differences in success and absolute numbers of stimuli are given by the fact,
that some conversation designs ended up producing a stimuli different from the originally intended.


\subsection{Design rating}

The rating for the various conversation types came out to be following:

\begin{quote}
\begin{tabbing}
\hspace{4cm} \= \hspace{4cm} \= \kill % set up tab stops
\textbf{anaphora type} \> \textbf{rating} \\
Shallow \> 2.75 \\
Deep \> 2.53 \\
Nonassignable \> 2.95 \\
\end{tabbing}
\end{quote}

Opposed to expectations shallow anaphora stimulus type has worse score than deep anaphora.
The p-value from Kruskal-Wallis test being 0.02 is sufficiently low, but
when values stimulus types are compared post-hoc via Dunn test,
result come to:

\begin{quote}
\begin{tabbing}
\hspace{4cm} \= \hspace{4cm} \= \kill % set up tab stops
\textbf{anaphora types} \> \> \textbf{p-value} \\
Deep \> Nonassignable \> 0.005 \\
Deep \> Shallow \> 0.158 \\
Nonassignable \> Shallow \> 0.136 \\
\end{tabbing}
\end{quote}

The only sufficiently low p-value is between deep and nonassignable anaphora.
The results only partially correspond with the expectations.
Number rating is however noisy as other things happen in conversation that affect it.
The chatbot contributions to the conversation outside of the stimulus
has to be under control and produce natural responses in order to really be able
to view the ratings as reflecting the stimuli
hence the high p-values.

\subsubsection{Design rating with abandonment interpretation}

Some conversations do not have a rating because participant clicked off of the experiment webpage.
Since the participants were instructed to leave the conversation if it is not \textit{natural},
these conversations can be interpreted as the worse possible reaction - 5.

We can further split these into a situation where
the participant clicked off of the experiment
right after the stimulus.
This only happened in case of nonassignable anaphora design.
Its new average score comes to 3.06.

If the participant left the conversation at any time after the stimulus,
this can be because the stimulus disturbed the conversation coherence
and while it was not serious enough at the stimulus time for the participant
to end the conversation,
the coherence never recovered.
If we interpret conversations abandoned at any point after stimulus as rated with the worse possible rating,
we get following results:

\begin{quote}
\begin{tabbing}
\hspace{4cm} \= \hspace{4cm} \= \kill % set up tab stops
\textbf{anaphora type} \> \textbf{rating} \\
Shallow \> 2.80 \\
Deep \> 2.77 \\
Nonassignable \> 3.15 \\
\end{tabbing}
\end{quote}

Shallow anaphora that is expected to have the best result still gets a slightly worse rating than the deep one.
The differences between these ratings and the ones where abandoned conversations are excluded are:

\begin{quote}
\begin{tabbing}
\hspace{4cm} \= \hspace{4cm} \= \kill % set up tab stops
\textbf{anaphora type} \> \textbf{difference} \\
Shallow \> -0.04 \\
Deep \> -0.23 \\
Nonassignable \> -0.19 \\
\end{tabbing}
\end{quote}

The difference shows that the shallow anaphora had by far the least number of abandoned conversations.
Updated significance check provides a slightly different picture with
Kruskal-Wallis test p-value being a similar value, but
the post-hoc Dunn test showing more significant differences:

\begin{quote}
\begin{tabbing}
\hspace{4cm} \= \hspace{4cm} \= \kill % set up tab stops
\textbf{anaphora types} \> \> \textbf{p-value} \\
Deep \> Nonassignable \> 0.016 \\
Deep \> Shallow \> 0.653 \\
Nonassignable \> Shallow \> 0.019 \\
\end{tabbing}
\end{quote}

Here the results are more significant but still not enough for deep vs shallow anaphora.
In conclusion the shallow anaphora was not rated as expected in relation to the other stimuli, but
the difference does not seem to be significant and
in accordance with the expectations has the lowest number of abandoned conversations.
There are many factors in the conversation data and
stimulus does not seem to have a strong enough effect on the rating.
If better more seamless conversation could be simulated,
perhaps stronger effect of the stimulus on rating could be observed.
The small difference can also be attributed to the fact that participants tend to
avoid extremes in their grading.

\subsection{Reaction to stimulus}

A quantitative metric that is expected to show better the acceptability and understandability
of the various stimulus designs is the direct conversational reaction to them.
The annotated data came to the following result:

\begin{quote}
\begin{tabbing}
\hspace{4cm} \= \hspace{3cm} \= \hspace{3cm} \= \hspace{3cm} \= \kill % set up tab stops
\textbf{anaphora type} \> \textbf{continuation} \> \textbf{meta} \> \textbf{abort} \\
Shallow         \> 91 \> 8 \> 5 \\
Deep            \> 34 \> 26 \> 4 \\
Nonassignable   \> 13 \> 57 \> 46 \\
\end{tabbing}
\end{quote}

Which percentage wise gives the following:

\begin{quote}
\begin{tabbing}
\hspace{4cm} \= \hspace{3cm} \= \hspace{3cm} \= \= \hspace{3cm} \kill % set up tab stops
\textbf{anaphora type} \> \textbf{continuation} \> \textbf{meta} \> \textbf{abort} \\
Shallow         \> 87.50\% \> 7.69\% \> 4.81\% \\
Deep            \> 53.12\% \> 40.62\% \> 6.25\% \\
Nonassignable   \> 11.21\% \> 49.14\% \> 39.66\% \\
\end{tabbing}
\end{quote}

With the extremely low p-value coming to p < 0.0001
 there is no doubt
that the different stimuli have a clear effect on the participant reaction and
that the results will be replicable.
The expectations are met with the continuation being highest for shallow anaphora
and lowest for nonassignable anaphora while
the abort reaction proves to have the opposite tendency.

One unpredicted feature in the results is
deep anaphora having a relatively close percentage of continuation and meta reactions and
nonassignable anaphora sharing similar values for meta reactions and abort reactions.
This suggests there is a inner division in the data that can be described and simulated more closely.

\subsubsection{Stimulus x Reaction specific ratings}

Looking at ratings for specific groups of stimuli, results are following:

\begin{quote}
\begin{tabbing}
\hspace{4cm} \= \hspace{3cm} \= \hspace{3cm} \= \= \hspace{3cm} \kill % set up tab stops
\textbf{anaphora type} \> \textbf{continuation} \> \textbf{meta} \> \textbf{abort} \\
Shallow \> 2.73 \> 3.12 \> 2.60 \\
Deep \> 2.43 \> 2.62 \> 2.75 \\
Nonassignable \> 2.45 \> 3.04 \> 2.98 \\
\end{tabbing}
\end{quote}

Interestingly for shallow and nonassignable anaphora,
the ratings do not follow the expected course.
The unfulfiled expectation lies in the meta reaction having worse ratings than the abort reactions.
The only anaphora type the expectation were met for is the deep one.
Yet the significance of this result is low and
more control over the conversation outside of the stimulus needs to be acquired
to be able to rely on these results.
To make conclusions about the meta reaction being indicative of
a less acceptable experience in participant
would be hasty.

\subsection{Comments}

While annotating, participants comments were collected
to gain an extra metric on the reflection of the stimuli.
While the shallow anaphora was mentioned once due to a generation issue
and generally went by unnoticed,
the deep anaphora got 11 mentions, which makes for
slightly less than a fifth of conversations with this stimulus.
Finally the nonassignable anaphora was mentioned 69 times in the participant comments.
This makes over half of its occurences were commented upon.
All the comments expressed confusion about mapping the anaphora to a possible preceding referent.
This result follows the expectations.

\subsection{Conversation style}

As mentioned above, there was an attempt
to expose different participants to different conversation styles.
This came out of the neccesity to create a default way for the chatbot to communicate.
First style is the inquisitive, where the chatbot is prompted to ask questions.
An elaboration on it is a style of conversation,
where the chatbot is at one time prompted to ask questions and
at another to simply comment on the topic at hand.
There were 183 conversations collected with the chatbot communicating in an inquisitive style
and 128 conversations where the chatbot communicated in the second, more relaxed style.
The relaxed style has been expected to have a better score as was confirmed
with following result:

\begin{quote}
\begin{tabbing}
\hspace{4cm} \= \hspace{3cm} \kill % set up tab stops
\textbf{conversation style} \> \textbf{rating} \\
Inquisitive \> 2.63\\
Relaxed \> 2.45 \\
\end{tabbing}
\end{quote}

However the Kruskal-Wallis test p-value of 0.10 suggests too small effect
to consider this result replicable.
To find statistically significant effect on conversation styles
the conversation needs to be controlled in a more nuanced way by the dialog system.

The relaxed style however better rated on average also suffered from some issues
in terms of generating natural answers.
Sometimes the chatbot would simply repeat the participants response without adding anything new to it.
For example in following conversation lack of topical input lead to aborting the conversation:

\begin{quote}
\begin{itemize}[label={}, leftmargin=0pt, itemsep=0.5em]
\item \textbf{conversation id: 164}
\item ...
\item \utterance{b:}{To chápu, co bys chtěl v histologii zlepšit?}{I understand, what would you like to improve about histology}
\item \utterance{h:}{chtela bych ji nemit proste}{I would just like to not have it}
\item \utterance{b:}{To chápu, chtěla bys ji prostě nemít.}{I understand, you would just like to not have it.}
\item \textit{aborted}
\end{itemize}
\end{quote}


\subsection{Zero anaphora}

Since Czech is a prodrop language\footcite{pevskova2019slavic},
it is expected that where the stimulus anaphora stands in subject position,
there will be tendency of it being dropped.
A response with zero anaphora \footcite{zero} would then be expected to be more acceptable
than an equivalent response with an explicit one.

During annotation there have been found to be only 16 conversations with zero anaphora stimulus.
This is partially due to the prompting of the stimulus,
the few-shot prompt contained no example of a zero anaphora and
partially because a lot of the stimuli contained an object positioned anaphora.

Coincidently the stimuli containing zero anaphora were split equally between shallow and deep anaphora stimuli.
The nonassignable anaphora stimuli mostly consisted of a hardcoded response that already contained the anaphora
therefore it is not a part of the zero anaphora collection of conversations.

\begin{quote}
\begin{tabbing}
\hspace{4cm} \= \hspace{3cm} \kill % set up tab stops
\textbf{reaction type} \> \textbf{\# occurences}\\
Continuation \> 14 \\
Meta \> 2 \\
Abort \> 0  \\
\end{tabbing}
\end{quote}

This result very clearly shows that the zero-anaphora stimuli were mostly going by unnoticed
and have a higher level of acceptability.
The same goes for rating with following result:

\begin{quote}
\begin{tabbing}
\hspace{4cm} \= \hspace{3cm} \kill % set up tab stops
\textbf{anaphora type} \> \textbf{rating}\\
Zero \> 2.19 \\
Nonzero \> 2.73 \\
\end{tabbing}
\end{quote}

Only shallow and deep anaphora type stimuli were used for this calculation.
The zero anaphora score is a lot better considering the range of averages
that the data has produced so far.
This shows future designs need to take zero-anaphora into account more
in order to achieve stronger contrast between stimuli.

The data contains conversations where stimulus was
inappropriately generated with a nonzero anaphora like here:

\begin{quote}
\begin{itemize}[label={}, leftmargin=0pt, itemsep=0.5em]
\item \textbf{conversation id: 347}
\item ...
\item \utterance{h:}{...hráli jsme je přes aplikaci ale já mu k vánocům koupila pravou šachovnici, takže konečně můžeme hrát normálně.}{... we would play via an app but I bought him a chessboard for christmas so we can finally play normally.}
...
\item \utterance{b:}{Jakou máte ji, klasickou nebo něco speciálního?}{Which do you have her, classic or something special?}
\item \utterance{h:}{jak to myslíš?}{what do you mean?}
\item ...
\end{itemize}
\end{quote}

The meta reaction is likely caused by the depth of the anaphora,
but the bot response would have been more likely had the anaphora not been a part of it.

In some cases zero anaphora can also be inappropriate like here:

\begin{quote}
\begin{itemize}[label={}, leftmargin=0pt, itemsep=0.5em]
\item \textbf{conversation id: 357}
\item ...
\item \utterance{h:}{Nooo teď jsme jeli kolem břevnovského kláštera, nikdy jsem tam nebyla, tak bych zajela tam.}{Well we just drove past Břevnov monastery, I never visited, so I would go there.}
\item \utterance{b:}{To zní zajímavě! Co tě na Břevnovském klášteře láká nejvíce?}{That sounds interesting! What about Břevnov monastery is the most attractive for you?}
\item \utterance{h:}{To nevím právě, překvapilo me to jak je to velký.}{I don't know, it surprised me how big it is.}
\item \utterance{b:}{Máš rád objevování nových míst a památek?}{Do you like discovering new places and monuments?}
\item \utterance{h:}{Ráda bych na to měla více prostoru.}{I would like to have more space to do it.}
\item \utterance{b:}{To chápu! Jaké místo bys chtěla objevit jako první?}{I get that! Which place would you like to discover first?}
\item \utterance{h:}{No nevím ale třeba bys mi mohl doporučit nějaký hezký místo v Česku spjato i s nějakou novou moderní architekturou, možná nějaká vyhlídka.}{I don't know but maybe you could recommend a nice place in Czechia connected with some new modern architecture, maybe a lookout tower.}
\item \utterance{b:}{Slyšela jsi o historii a zajímavostech?}{Have you heard about history and trivia?}
\item \textit{aborted}
\end{itemize}
\end{quote}

The participant commented after the conversation that the question did not fit in with the context
which is why they aborted.
That is reasonable as the response was generated based on the monastery topic from deeper in the conversation.
Perhaps if there was an explicit anaphora here, the mapping would have been easier
as dropping in a nonsubject position is not preferred in Czech.
Regardless, the hard topical shift that disrupts the coherence
caused by the chatbot would remain in the conversation.

\chapter{Qualitative observation}

Visiting each conversation during annotation
gave a detailed insight on the nature of the data,
success of generating the correct stimuli and
creating the sought after situations.
The following observations about
the experiment design, the various stimuli and conversation coherence in general
were made.

\section{Pro-coherence perception}

Most noticeably the data shows how participants communicate with cooperativity in mind
and assume a lot in order to maintain the conversation flow.
The quality with which the chatbot could generate responses was high enough,
that the illusion of social interaction would hold up even if minor issues did come up.

\subsection{Remaps}

In following conversation, the participant maps an anaphora on a possible candidate
even though the hardcoded response was meant to be nonassignable:

    \begin{quote}
    \begin{itemize}[label={}, leftmargin=0pt, itemsep=0.5em]
    \item \textbf{conversation id: 100}
    \item ...
    \item \utterance{h:}{mám jednu mourovatou doma}{I have a tabby one at home}
    \item \utterance{b:}{Jak se jmenuje tvá mourovatá kočka?}{What is the name of your tabby cat?}
    \item \utterance{h:}{to nevím}{I don't know}
    \item \utterance{b:}{Jak dlouho už ji máš doma?}{How long have you had it at home?}
    \item \utterance{h:}{nemám vlastně}{I don't actually}
    \item \utterance{b:}{Jak ses k ní tedy dostal?}{So how did you get it?}
    \item \utterance{h:}{ráda peču}{I like to bake}
    \item \utterance{b:}{Jaké oblíbené dobroty obvykle pečeš?}{Which treats do you usually bake?}
    \item \utterance{h:}{dorty}{cakes}
    \item \utterance{b:}{Tak to by mě zajímalo, jak na to ona potom zareagovala.}{Well I'd like to know how did she react to it afterwards.}
    \item \utterance{h:}{myslíš tu kočku?}{You mean the cat?}
    \item ...
    \end{itemize}
    \end{quote}


This conversation along with several others is the reason annotation was necessary for this type of stimulus.
The participant tolerance was sometimes very high, even if the anaphora was incongruent
with the assumed mapping or even if
some of the reference contained in the hardcoded stimulus does not map perfectly
like in following conversation:

    \begin{quote}
    \begin{itemize}[label={}, leftmargin=0pt, itemsep=0.5em]
    \item \textbf{conversation id: 257}
    \item ...
    \item \utterance{b:}{Stále je co zlepšovat, přemýšlel jsi o trenérovi?}{There is always room to improve, did you consider a trainer?}
    \item \utterance{h:}{na dvou hodinách už jsem byl s trenérem}{I already did two lessons with a trainer}
    \item \utterance{b:}{Tak to by mě zajímalo, jak na to ona potom zareagovala.}{Well I'd like to know how did she react to it afterwards.}
    \item \utterance{h:}{že musím lépe používat nohy, pohyb vychází z kyčle}{that I have to use my legs better, the movement start in the hip}
    \item ...
    \end{itemize}
    \end{quote}

On this regard, a general third person seems to be acceptable,
though from the annotation point of view is should be nonassignable as can be seen here:

    \begin{quote}
    \begin{itemize}[label={}, leftmargin=0pt, itemsep=0.5em]
    \item \textbf{conversation id: 36}
    \item ...
    \item \utterance{b:}{Co tě nejvíce zaujalo v této knížce?}{What did you find the most interesting in this book?}
    \item \utterance{h:}{Nádherné fotografie minerálů pod mikroskopem.}{Beautiful photos of minerals under a microscope.}
    \item \utterance{b:}{Jaké další knihy o minerálech ti doporučili?}{What other books about mineral did they recommend?}
    \item \utterance{h:}{Většinou mi nikdo nic nedoporučuje, hledám si knihy sám.}{Usually nobody recommends me anything, I look for my books alone.}
    \item ...
    \end{itemize}
    \end{quote}

In some cases the nonassignable anaphora stimulus has the potential
to be interpreted as a third person reference to the participant.
Such cases seem to have been tolerated by the participants as well
though in human-human conversation these would probably cause metacommunication.

    \begin{quote}
    \begin{itemize}[label={}, leftmargin=0pt, itemsep=0.5em]
    \item \textbf{conversation id: 276}
    \item ...
    \item \utterance{b:}{Co tě dneska tak unavilo?}{What exhausted you today?}
    \item \utterance{h:}{konverzace s klučičím pohlavím}{conversations with the male gender}
    \item \utterance{b:}{To zní náročně, o čem jste si povídali?}{That sounds demanding, what did you talk about?}
    \item \utterance{h:}{o škole}{about school}
    \item \utterance{b:}{Tak to by mě zajímalo, jak na to ona potom zareagovala.}{Well I'd like to know how did she react to it afterwards.}
    \item \utterance{h:}{docela dobře}{pretty well}
    \item ...
    \end{itemize}
    \end{quote}

In case of the deep anaphora stimulus, one of the big reasons annotation was necessary
was also the risk of a remap.
Generating the deep anaphora response, the LLM is only shown the appropriate section of the conversation.
This exposes the generated and anaphorized response
to the risk of another mapping candidate to have been mentioned in the meantime like in the following conversation:

    \begin{quote}
    \begin{itemize}[label={}, leftmargin=0pt, itemsep=0.5em]
    \item \textbf{conversation id: 259}
    \item ...
    \item \utterance{h:}{ano, mám - zaklínač 3, kingdomcome, nebo dota 2}{yes I do - Wither 3, kingdomcome or dota 2}
    \item \utterance{b:}{Zaklínač 3 má nádherně propracovaný svět a silný příběh.}{Withcer 3 has an elaborate world and a strong story.}
    \item \utterance{h:}{přesně tak, je to úplně jiný svět než ten náš, takže je to skvělá relaxace.}{exactly, it is a completely different world from ours, so it is a great relaxation.}
    \item \utterance{b:}{Do které části herního světa Zaklínače 3 se nejraději vracíš?}{To which part of the Witcher 3 world do you like to return the most?}
    \item \utterance{h:}{Jako které místo se mi tam líbilo? Novigrad je moc hezký}{As in which place there I liked? Novigrad is very pretty.}
    \item \utterance{b:}{Kterého hrdinu v ní hraješ nejraději?}{Which hero do you play most often in her? (dota 2 is feminine in Czech)}
    \item \utterance{h:}{No tak tam není moc široký výběr hrdinů, Geralta}{Well there is not a broad selection of heroes, Geralt}
    \item ...
    \end{itemize}
    \end{quote}

Since multiple videogames were mentioned,
the participant interpreted the question to be about the most recetnly mentioned one,
\textit{Witcher 3} which has also been topicalized by the chatbot.
The chatbot was prompted to ask about the \textit{dota 2} where also the question makes more sense.

Due to stimulus generation issues, sometimes the reference is misunderstood.
In following conversation the stimulus reference the the \textit{african continent}
but the participant answers in a more broad way.
They even mention this misunderstanding later in a comment.
This means during the conversation, the anaphora is nonassignable and general enough,
that the participant continues the conversation even if it entails a topic shift.

\begin{quote}
\begin{itemize}[label={}, leftmargin=0pt, itemsep=0.5em]
\item \textbf{conversation id: 243}
\item ...
\item \utterance{b:}{Dnes jsem se dozvěděla, že existuje více než 7 000 jazyků.}{Today I learned there is over 7000 languages.}
\item \utterance{h:}{to je hodně zajímavé, už jsem to taky slyšela. Mám pocit, že třetinou z nich se mluví na africkém kontinentu}{that is very interesting, I also heard about this. I think that third of them are spoken on the african continent.}
\item \utterance{b:}{Máš nějakou oblíbenou zemi nebo jazyk na něm?}{So you have a favorite country or language on him? (continent is masculine in Czech)}
\item \utterance{h:}{mám docela ráda anglii a angličtinu, co ty?}{I like england and english, what about you?}
\item ...
\end{itemize}
\end{quote}


\subsection{Associative anaphora}

    An anaphora which lacks an explicit candidate mapping does not necessarily need to be nonassignable.
    Such associative anaphora \footcite{assocAnaph} also occurs in the collected data:

    \begin{quote}
    \begin{itemize}[label={}, leftmargin=0pt, itemsep=0.5em]
    \item \textbf{conversation id: 248}
    \item ...
    \item \utterance{h:}{Byla jsem v nemocnici za pacienty}{I went to the hospital to see the patients}
    \item \utterance{b:}{To muselo být zajímavé. Jaká byla tvoje nejcennější zkušenost?}{That must have been interesting. What was you most valuable experience?}
    \item \utterance{h:}{Je jich hodne}{There was a lot}
    \item \utterance{b:}{To zní dobrodružně, co tě tam dneska konkrétně zaujalo?}{That sounds adventurous, what specifially interested you today?}
    \item \utterance{h:}{Zvetsena slezina}{enlarged spleen}
    \item \utterance{b:}{Znamenalo to pro něj nějaké specifické komplikace?}{Did it for him entail some complications?}
    \item \utterance{h:}{No…je to priznak lymfomu}{Well... it is a symptom of lymphoma}
    \item ...
    \end{itemize}
    \end{quote}

    Here the chatbot response contains an anaphora which
    cannot be mapped onto a specific mentioned entity in the previous text
    but is understandable as there is clearly a person at hand in the situation the participant describes.
    The participant has then no issues continuing the conversation and
    moving the topic forward by answering the chatbots question.

    A spectrum could be shown of anaphoric devices
    that stems from a associative anaphora which is clearly related to a mentioned entity
    though lacks an explicit antecedent and
    a newly mentioned entity which then has the potential to retrospectively be pronounced a cataphora
    if it is lexically realized later on in the conversation.

\subsection{Sequential coherence}

Without directly provoking it via experiment design,
some conversations show a pro-coherence perception on the participant side
in terms of the sequental structure \footcite{schegloff1996turn} of the conversation:


    \begin{quote}
    \begin{itemize}[label={}, leftmargin=0pt, itemsep=0.5em]
    \item \textbf{conversation id: 123}
    \item ...
    \item \utterance{h:}{Boj s rakovinnými buňkami znáš?}{Do you know the fight with cancer cells?}
    \item \utterance{b:}{Jak se rozsivky konkrétně podílejí na boji s ní?}{How diatoms specifically participate in the fight against it?}
    \item \utterance{h:}{Ano.}{Yes.}
    \item ...
    \end{itemize}
    \end{quote}

Here the participant answer hints at a misinterpretation of the chatbot response.
While the chatbot is instructed to ask open-ended questions to keep the conversation going,
the participant seems to understand the question in the chatbot response
as a request for confirmation which is also given.
This is likely because the chatbot question adds new topical content with its question
which allows for interpreting the question as demonstrating a level of knowledge about the topic
which the participant inquires about in the first question.

\section{Topical (in)coherence}

    %too simplified a notion
    %remaps
    %lexical surroundings - verb
    %perfect

    The deep anaphora stimulus type had the highest error rate.
    The annotation showed that only 64 conversations of this type were collected.
    That is approximately half of how many conversations were collected of the shallow and nonassignable anaphora.
    This relative failure has mainly been caused by the remaps discussed above.
    In its ideal form it looked like this:

    \begin{quote}
    \begin{itemize}[label={}, leftmargin=0pt, itemsep=0.5em]
    \item \textbf{conversation id: 253}
    \item ...
    \item \utterance{h:}{Začínám se připravovat na zkouškové období na vejšce, protože už se kvapem blíží}{I am starting to prepare for the examp period in my university, because it is coming up quickly.}
    \item \utterance{b:}{Jaké předměty ti dají nejvíc zabrat?}{Which subjects are the hardest?}
    \item \utterance{h:}{Asi demografická analýza}{I guess demographical analysis.}
    \item \utterance{b:}{Co tě na demografické analýze nejvíc baví, nebo naopak stresuje?}{What is the most fun aspect or the most stressful about demographical analysis?}
    \item \utterance{h:}{Nejvíce mě stresuje, že je tam opravdu hodně učiva. Baví mě tabulky v excelu}{What stresses me the most is that there is really a lot of material. I like the excel tables.}
    \item \utterance{b:}{Na co se pak těšíš nejvíc, až zkouškové skončí?}{What do you look forward to the most, when the exam period is over?}
    \item \utterance{h:}{Na ten klid a žádnej stres}{The peace and no stress}
    \item \utterance{b:}{Na co ještě jiného, kromě klidu, se těšíš?}{Is there anything else other than peace you are looking forward to?}
    \item \utterance{h:}{Na to, že budu moct zase trávit víc času se svými přáteli}{To being able to spend more time with my friends again}
    \item \utterance{b:}{Jak dlouho už ji studuješ?}{How long have you been studying it?}
    \item \utterance{h:}{druhým rokem}{second year}
    \item ...
    \end{itemize}
    \end{quote}

    The chatbot question goes back to an entity that has only been mentioned several turns ago
    and the conversation topic has somewhat drifted apart since then.
    The participant would in this case be able to map the anaphora with the help of its lexical environment.
    Especially the verb of the question containing the anaphora makes or breaks the understanding.
    This is clear in following conversation, where the verb is repeated:

    \begin{quote}
    \begin{itemize}[label={}, leftmargin=0pt, itemsep=0.5em]
    \item \textbf{conversation id: 30}
    \item ...
     \item \utterance{h:}{vcera me bolela hlava}{I had a headache yesterday}
     \item \utterance{b:}{Jak dlouho už tě to trápí?}{How long has that bothered you?}
     \item \utterance{h:}{krk jen dnes a hlava dva dny}{throat only today and head two days}
     \item \utterance{b:}{Bolí tě i něco jiného, nebo jen hlava a krk?}{Does anything else hurt other than head and throat?}
     \item \utterance{h:}{nene}{no no}
     \item \utterance{b:}{Co si myslíš, že by ti mohlo pomoct?}{What do you thing could help you?}
     \item \utterance{h:}{snad léky z lékarny}{Perhaps medicine from the pharmacy}
     \item \utterance{b:}{Co jsi dělal, když tě začala ona bolet?}{What did you do when she started hurting? (note: head is feminine in Czech)}
     \item \utterance{h:}{vzala si paralen}{I took a paralen (note: paralen is a pill)}
    \item ...
    \end{itemize}
    \end{quote}

    This mapping is so obvious due to the repetition that
    the inappropriate nonzero anaphora does not distract the participant.

\subsection{Hard shift refusal}

    Various reactions to the deep anaphora stimulus type were recorded.
    The data shows sometimes mapping the anaphora does not warrant acceptance of the response.

    \begin{quote}
    \begin{itemize}[label={}, leftmargin=0pt, itemsep=0.5em]
    \item \textbf{conversation id: 114}
    \item ...
    \item \utterance{b:}{To je těžké říct, ale miluju Gaudího. Co ty?}{That is hard to say, but I love Gaudí. What about you?}
    \item \utterance{h:}{Já také, co se ti na Gaudím líbí?}{Me too, what do you like about Gaudí?}
    \item \utterance{b:}{Líbí se mi jeho kreativita a přírodní inspirace. Co tě na něm oslovuje?}{I love his creativity and inspiration in nature. What talk to you about him?}
    \item ... 2 turns (4 utterances) ...
    \item \utterance{h:}{Vinylových desek má až moc :( nenapadá tě něco jiného?}{He has way too many vinyls :( can't you think of something else?}
    \item \utterance{b:}{Jeho organické tvary a barvy mě fascinují. Jak se ti líbí ta?}{His organic shapes and color fascinate me. How to you like her?}
    \item \utterance{h:}{Cože? O čem to mluvíš?}{What? What are you talking about?}
    \item ...
    \end{itemize}
    \end{quote}

    In this conversation the stimulus suffers from a generation issue and
    causes a metacommunicative reaction
    and later is commented upon:

    \begin{quote}
    \utterance{comment:}{Chatbot nepochopil změnu tématu, pravděpodobně reagoval na romantismus, ale já se s ním chtěla bavit o dárcích. Navíc, nedokázal strukturovat otázku.}
    {Chatbot misunderstood the topic shift, probably reacted to romantism, but I want to talk about gifts. Besides he couldn't structure the question.}
    \end{quote}

    Other than the ideal case where deep anaphora supports the conversation coherence by
    following up on a previously mentioned topic,
    deep anaphora as it was programmed here would sometimes abruptly shift to an abandoned topic.

\subsection{Incomplete control over conversation}

    One other frequent reason conversations that were
    meant to contain a deep anaphora stimulus would fail
    was insufficient topical distance from the anaphorized entity.
    Since convform does not relate tracked entities to each other
    conversations like following could not be considered to contain a deep anaphora
    because the anaphorized entity would last occur too recently:

    \begin{quote}
    \begin{itemize}[label={}, leftmargin=0pt, itemsep=0.5em]
    \item \textbf{conversation id: 192}
    \item ...
    \item \utterance{b:}{Pravěk je zajímavý! Co tě na něm nejvíc fascinovalo?}{Prehistoric era is fascinating! What fascinated you most about it?}
    \item \utterance{h:}{Mamuti}{Mammoths}
    \item \utterance{b:}{Mamuti jsou fascinující! Četl jsi o jejich vyhynutí?}{Mammoths are fascinating! Did you read about their extinction?}
    \item \utterance{h:}{Ještě jsem je viděl}{I saw them still}
    \item \utterance{b:}{Opravdu? Kde jsi je viděl a jaký to byl zážitek?}{Really? Where did you see them and how was that experience?}
    \item ... 4 turns (8 utterances) ...
    \item \utterance{b:}{Mamuti a jejich adaptace na ledové podmínky mě fascinují.}{Mammoths and their adaptation to icy conditions fascinate me.}
    \item \utterance{h:}{V těch knihovnách?}{In those libraries?}
    \item \utterance{b:}{Odkud sis vybral materiály k jejich výzkumu?}{Where did you pick the materials for their research?}
    \item ...
    \end{itemize}
    \end{quote}

\subsection{Nontopicalized entities}

    One of the requirements for an entity
    to be able to be mapped onto by a deep anaphora
    might be that it has been topicalized.
    One such case is found in the data:

    \begin{quote}
    \begin{itemize}[label={}, leftmargin=0pt, itemsep=0.5em]
    \item \textbf{conversation id: 162}
    \item ...
    \item \utterance{b:}{Jaké hry máš nejraději hrát s kamarády?}{Which games do you like to play with your friends the most?}
    \item \utterance{h:}{Asi Sea of Thieves, Minecraft, nebo střílečky}{Sea of Thieves, Minecraft or shooters I guess}
    \item \utterance{b:}{Jakou hudbu nejraději posloucháš při hraní her?}{Which music do you prefer to listen to while playing these games?}
    \item \utterance{h:}{Ne vždycky u toho poslouchám hudbu, ale když ano, tak buď rap, nebo doporučení od dvou kanálů na Youtube}{I don't alway listen to music, but when I do it's either rap or recommendations of two Youtube channels}
    \item \utterance{b:}{Jaké kanály na YouTube ti dělají hudební doporučení?}{Which channels on Youtube give you recommendations?}
    \item \utterance{h:}{Asi je neznáš, ale xkito a MrSuicideSheep}{You probably don't know them but xkito and MrSuicideSheep}
    \item \utterance{b:}{Jaké stavební projekty jsi v něm nejvíc dokončil?}{Which building projects did you finish the most in him (note: Minecraft is masculine in Czech)}
    \item \textit{aborted}
    \end{itemize}
    \end{quote}

    The participant aborts the conversation after a question is asked that contains an anaphora
    which due to its lexical environment can only be mapped to \textit{minecraft} from a previous participant response
    as it is the only entity that has to do with \textit{building projects}.
    This entity however has only been in the rhema position and mentioned long ago from the perpsective of topical progression.
    Other issues contributing to the participant aborting the conversation are lack of the topic shift annotation
    and lack of closure of the interrupted topic.
    There has however also been an annotator disagreement on this conversation
    as one of the annotators considered this conversation to contain a nonassignable anaphora.
    While that was slightly against the annotator manual,
    because formally there is indeed a mapping candidate present in the conversation text,
    this supports the participants decision to stop the conversation.
    In terms of Daneš \footcite{danevs1989functional} the conversation topic progression is as follows:

    \newcommand{\topicannot}[3]{
        \textbf{#1} #2 \\
    }

    \begin{quote}
    \begin{itemize}[label={}, leftmargin=0pt, itemsep=0.5em]
    \item \textbf{topical progression of conversation id: 162}
    \item ...
    \item \topicannot{b:}{Which T1?}
    \item \topicannot{h:}{R1a, R1b , or R1c.}
    \item \topicannot{b:}{Which T2?}
    \item \topicannot{h:}{Either R2a or R2b.}
    \item \topicannot{b:}{Which T3(<-R2b)?}
    \item \topicannot{h:}{T3a and T3b.}
    \item \topicannot{b:}{Which T4(<-R1b)?}
    \item \textit{aborted}
    \end{itemize}
    \end{quote}

    Here T4 is tied to R1b which never stood in the topic position.
    More conversations of this type  need to be gathered
    to understand the role of topicalization in conversation coherence.

\subsection{Unexpected refusal}

Sometimes stimuli that were considered to be mappable and coherent
cause metacommunicative reactions - the participants misunderstood.

In the following conversation the stimulus was mappable:

\begin{quote}
\begin{itemize}[label={}, leftmargin=0pt, itemsep=0.5em]
\item \textbf{conversation id: 96}
\item ...
\item \utterance{b:}{Co by tě na tom předmětu víc zaujalo?}{What would interested you the most in that course?}
\item \utterance{h:}{kdyby ho učil pes}{If it was given by a dog}
\item \utterance{b:}{Jaké bys od psa čekal vyučovací metody?}{Which teaching methods would you expect from a dog?}
\item \utterance{h:}{štěkání v morseovce}{barking morse code}
\item \utterance{b:}{A co bys dál od psa očekával v hodině?}{And what else would you expect from a dog in the lesson?}
\item \utterance{h:}{mazlení. co ty?}{cuddling. what about you?}
\item \utterance{b:}{Jakého bys chtěl mít jako učitele?}{Which would you like to have as a teacher?}
\item \utterance{h:}{jak to myslíš?}{how do you mean?}
\item ...
\end{itemize}
\end{quote}

The question itself was however ambivalent in some way as the participant comment suggests:

\begin{quote}
\utterance{comment:}{nenapsal jaké plemeno}{didn't write which race}
\end{quote}

In following conversation, zero anaphora referring to the topic at hand
caused metacommunication:

\begin{quote}
\begin{itemize}[label={}, leftmargin=0pt, itemsep=0.5em]
\item \textbf{conversation id: 312}
\item ...
\item \utterance{h:}{Mám, jmenuje se: She is the man}{I do, it is called: She is the man}
\item \utterance{b:}{To je skvělý film! Co tě na něm nejvíc baví?}{That is a great movie! What do you like about it the most?}
\item \utterance{h:}{Je to romantická komedie, takže mě baví to, že je to spíše na oddech.}{It is a romcom so I like it more to relax.}
\item \utterance{b:}{Jaké typy máš nejraději?}{Which types do you like best?}
\item \utterance{h:}{Typy čeho?}{Types of what?}
\item \utterance{b:}{Typy filmů nebo seriálů, jaký žánr preferuješ?}{Types of movies or series, which genre do you prefer?}
\item ...
\end{itemize}
\end{quote}

Referring to a the current hypertopic via a zero anaphora has the potential to cause a spectrum of reaction.

\subsection{Unexpected acceptance}

Sometimes participants would accept a hardly or not at all assignable anaphora.

    \begin{quote}
    \begin{itemize}[label={}, leftmargin=0pt, itemsep=0.5em]
    \item \textbf{conversation id: 279}
    \item ...
    \item \utterance{b:}{Kterou koledou začínáte zkoušky?}{Which carol do you start your rehearsals with?}
    \item \utterance{h:}{Každou zkoušku začínáme jinak}{We start each rehersal differently}
    \item \utterance{b:}{Tak to by mě zajímalo, jak na to pak ona zareagovala.}{Then I wonder, how she reaction to that.}
    \item \utterance{h:}{To mě taky}{Me too}
    \item \utterance{b:}{Z jakého důvodu měníte pořadí hned po každém zkouškovém dni?}{For which reason do you change the order after every rehearsal day?}
    \item \textit{aborted}
    \end{itemize}
    \end{quote}

This conversation has a gramatically feminine mapping candidate
and somehow the chatbot remark though hardcoded makes sense to the participant.

In the following conversation the participant seems to want to continue in the conversation
regardless of the incoherent response from the chatbot and so they simply change the topic:

    \begin{quote}
    \begin{itemize}[label={}, leftmargin=0pt, itemsep=0.5em]
    \item \textbf{conversation id: 142}
    \item ...
    \item \utterance{h:}{a co jsi dělal ty? }{and what did you do?}
    \item \utterance{b:}{Dělal jsem si radost četbou a učením nových věcí. Ty máš nějaké oblíbené předměty?}{I was enjoying myself by reading and learning new things. Do you have any favorite subjects?}
    \item \utterance{h:}{biofyzikální chemii}{biophysical chemistry}
    \item \utterance{b:}{To zní zajímavě! Co tě na ní nejvíc baví?}{That sounds interesting! What do you like most about it?}
    \item \utterance{h:}{jako baví mě farmakologie}{like I enjoy pharmacology}
    \item \utterance{b:}{Tak to by mě zajímalo, jak na to pak ona zareagovala.}{Then I wonder, how she reaction to that.}
    \item \utterance{h:}{a ty studuješB }{and you studyB}
    \item ...
    \end{itemize}
    \end{quote}

\subsection{Nonfactual implication}

    The chatbot would sometimes hallucinate information that has not been said.
    Here the chatbot implies that the participant mentioned going to work that day:

    \begin{quote}
    \begin{itemize}[label={}, leftmargin=0pt, itemsep=0.5em]
    \item \textbf{conversation id: 264p}
    \item \utterance{b:}{Ahoj, já jsem Vladimatik.}{Hi, I am Vladimatik.}
    \item \utterance{h:}{Ahojky }{Hello}
    \item \utterance{b:}{Jaký jsi měl den?}{How was your day?}
    \item \utterance{h:}{Docela to šlo. Co ty?}{It was ok. What about you?}
    \item \utterance{b:}{Taky se mám dobře, díky za optání.}{I am also well, thanks for asking.}
    \item \utterance{h:}{To je dobře }{That's good.}
    \item \utterance{b:}{Taky ráda něco podnikám po práci, co tě dneska bavilo?}{I also like to do things after work, what did you enjoy today?}
    \item \utterance{h:}{Já jsem dneska nebyla v práci a navíc mě dnes nic nebavilo…}{I wasn't at work today and besides I enjoyed nothing today...}
    \item ...
    \end{itemize}
    \end{quote}

    Implying things that were not mentioned leads to a special type of metacommunicative reaction
    which can contain negated repetition of the incorrect statement.

\section{Metacommunication}

One of the stimulus reaction types is metacommunication.
This is a response which somehow addresses the course of the interaction itself.
It is not trivial and was therefore annotated.
In its simplest form metacommunicative reactions would be
a question that contains a repetition of the element causing confusion like here:

    \begin{quote}
    \begin{itemize}[label={}, leftmargin=0pt, itemsep=0.5em]
    \item \textbf{conversation id: 317}
    \item ...
    \item \utterance{b:}{Tak to by mě zajímalo, jak na to pak ona zareagovala.}{Then I wonder, how she reaction to that.}
    \item \utterance{b:}{kdo ona?}{who she?}
    \item ...
    \end{itemize}
    \end{quote}

In other cases a more general metacommunicative formulas are deployed which express general confusion.
This happeden in following case for example since
there were multiple unmapped anaphoras
preventing a simple metacommunication.

    270 double anaphora -> meta ¨
    \begin{quote}
    \begin{itemize}[label={}, leftmargin=0pt, itemsep=0.5em]
    \item \textbf{conversation id: 270}
    \item ...
    \item \utterance{b:}{To je skvělý plán, co nejraději v posilovně cvičíš?}{That is a great plan, what do you like to excersise in the gym?}
    \item \utterance{h:}{nejraději cvičím záda. myslím, že budu dělat cviky na záda zrovna dneska}{I like the back the most. I think I will do back excersise today}
    \item \utterance{b:}{Které konkrétně na něj mají vliv?}{Which exactly have influence on him?}
    \item \utterance{h:}{jak to myslíš, napiš mi prosím otázku líp}{what do you mean, write the question better please}
    \item ...
    \end{itemize}
    \end{quote}

Metacommunication signals a conversation coherence disruption but they give a chance to fix the issue at hand.
The clarification or correction may not suffice like in the following case:

    \begin{quote}
    \begin{itemize}[label={}, leftmargin=0pt, itemsep=0.5em]
    \item \textbf{conversation id: 101}
    \item ...
    \item \utterance{b:}{To jsi naprosto pravdu! Jaké knižní překvapení tě naposledy potkalo?}{You are completely right! Which book surprise did you get last?}
    \item \utterance{h:}{Asi brandon sanders, i když sci-fi vlastně vůbec nečtu}{I guess brandon sanders even though I don't actually read sci-fi at all.}
    \item \utterance{b:}{Tak to by mě zajímalo, jak na to pak ona zareagovala.}{Then I wonder, how she reaction to that.}
    \item \utterance{h:}{Kdo?}{Who}
    \item \utterance{b:}{Omlouvám se, myslel jsem na postavy z jeho knih. Co se ti na Sandersonovi líbilo?}{Sorry, I meant the characters from his books. What did you like about Sanderson?}
    \item \textit{abort}
    \end{itemize}
    \end{quote}

\section{Conclusion}

The inter-annotator agreement shows that the anaphora stimulus categories are valid.
However a deeper look into the data on both quantitative and qualitative levels shows
that there are further subgroups of coherence disruptions and anaphora types
which the categorization did not cover.
This was to a degree intentional.
Using a simplified model to understand the studied area via describing the models weaknesses
indeed brought an insight and managed to illuminate further steps.

The general research question has been to confirm that the coherence would be increasingly more disrupted
with each anaphora type - shallow, deep and nonassignable.
In most measurable ways this proved to be true.
This was not a surprising result as rather than answering the question
the goal of the research has been to gather concrete data supporting this claim
and confirming the usefulness of the conversation simulation methodology in the process.
Participant reactions to the different types of stimuli differed greatly,
the more disruptive sitmuli had more participants abandoning the conversation and
the participant comments show that while the shallow anaphora stimulus went by practically unnoticed,
the deep and nonassignable ones had increasingly more participants make a remark on them.
The only metric that did not quite succeed showing the predicted outcome was global rating of the conversation.
The results seem to go in the expected direction, but are not always statistically significant and
in some cases the deep anaphora conversations have a better average score than the shallow anaphora ones.
This is generally because the rating covers the entire conversation the participants had
and many other things happen in the conversation other than the stimulus.
For future research it is necessary to get the general course of the conversation more under control.

On a similar token, the general conversation style variable did not show any significant results.
Same cause can be attributed to this.
More control over the chatbot has to be gained
to program such that might generate better or worse participant ratings.

One aspect where the participant rating give an interesting result is the zero anaphora.
Conversation containing it have much better average rating that ones where it is explicit.
The data confirms an already knonw fact that Czech is a prodrop language.
This must be used in the future to increase the contrast in the observed phenomena
by decreasing noise from this nonacceptability source.

The data shows a strong pro-coherence perception.
The participants are often willing to map the anaphora to whichever candidate is at hand
even if it is only implicit and associative.

The deep anaphora stimulus type has been an interesting look into the inner workings of conversation.
The chatbot meant to ellicit this type of data was relatively unsuccesful.
Improvements to be made in that regard mainly lie in more thorough real-time topic annotation
and letting the post stimulus prompt access the intended reference.
That way the acceptance of the explanation in case of metacommunicative reaction can also be observed.

In general it has been observed that the deep anaphora acceptance depends on several aspects.
The topic shift needs to be annotated in the discourse,
there needs to be enough information to perform mapping,
there needs to be no critical ambivalence to perform mapping and
an ntity that has not been previously topicalized is less likely to be mapped onto.
